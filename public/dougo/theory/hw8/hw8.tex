\documentclass{article}
\usepackage{amstex}
\usepackage{amssymb}
\usepackage{theorem}
\usepackage{fancyheadings}

\setlength{\parindent}{0pt}	% Don't indent paragraphs.

% Reduce margins.
\addtolength{\voffset}{-1in}
\addtolength{\textheight}{2in}
\addtolength{\hoffset}{-1.25in}
\addtolength{\textwidth}{2.5in}

\pagestyle{fancy}
\setlength{\headrulewidth}{0pt}

\lhead{COM 3350 Assignment 7}
\rhead{Doug Orleans (\texttt{dougo@@ccs.neu.edu})}

\theorembodyfont{\rmfamily}		% don't use italics
\theoremstyle{break}			% put a line break after the header
\newtheorem{exercise}{Exercise}
\theoremstyle{plain}
\newtheorem{subexercise}{}[exercise]
\renewcommand{\thesubexercise}{\alph{subexercise}.}
\newenvironment{answer}{\begin{quotation}\noindent}{\end{quotation}}

\newcommand{\sipser}{\textit{Sipser}}
\newcommand{\encoding}[1]{\ensuremath{\langle#1\rangle}}
\newcommand{\qed}{~\ensuremath{\square}}
\newcommand{\Th}{\ensuremath{\operatorname{Th}}}
\newcommand{\Nat}{\ensuremath{\mathcal{N}}}
\newcommand{\tred}{\ensuremath{\leq_{\text{T}}}}
\newcommand{\ATM}{\ensuremath{A_{\textsf{TM}}}}
\newcommand{\defin}[1]{\textbf{\textit{#1}}}
\newcommand{\sym}[1]{\ensuremath{\mathtt{#1}}}
\newcommand{\symbf}[1]{\textbf{\sym{#1}}}
\newcommand{\set}[1]{\ensuremath{\{#1\}}}
\newcommand{\enumlang}[1]{\set{\sym{#1}}}
\newcommand{\union}{\cup}
\newcommand{\setname}[1]{\textit{#1}}
\renewcommand{\liminf}{\ensuremath{\lim_{n\to\infty}}}
\newcommand{\TM}{\textsf{TM}}
\newcommand{\NTM}{\textsf{NTM}}
\newcommand{\DFA}{\textsf{DFA}}
\newcommand{\NFA}{\textsf{NFA}}
\renewcommand{\P}{\textrm{P}}
\newcommand{\NP}{\textrm{NP}}
\newcommand{\coNP}{\textrm{coNP}}
\newcommand{\emptystring}{\ensuremath{\boldsymbol\varepsilon}}
\newcommand{\ceil}[1]{\ensuremath{\left\lceil#1\right\rceil}}

%\renewcommand{\AND}{\ \keyword{and}\ }
%\newcommand{\ENDWHILE}{\untab\addtocounter{programline}{-1}}
%\newcommand{\ENDIF}{\untab\untab\addtocounter{programline}{-1}}
%\newcommand{\ACCEPT}{\keyword{accept}\ }
%\newcommand{\REJECT}{\keyword{reject}\ }
%\renewcommand{\keyword}[1]{\textbf{#1}} 
%\renewcommand{\prognumstyle}{\textrm} 
%\NumberProgramstrue 
%\normalbaroutside 

\begin{document}

\begin{exercise}[\sipser, 9.16]
Define the function
$\textit{majority}_n:\enumlang{0,1}^n\to\enumlang{0,1}$ as
\begin{displaymath}
\textit{majority}_n(x_1,\ldots,x_n)=\left\{
\begin{array}{ll}
\sym 0 & \sum x_i < n/2 \\
\sym 1 & \sum x_i \geq n/2
\end{array}
\right.
\end{displaymath}
Thus the \textit{majority}$_n$ function returns the majority vote of
the inputs.  Show that \textit{majority}$_n$ can be computed with
\begin{subexercise}
$O(n^2)$ size circuits.
\end{subexercise}
\begin{answer}
Construct a cascading array of gates clusters:
\vspace{2in}\\
This has $n/2$ rows of at most $n$ AND/OR pairs, thus $O(n^2)$
gates.\qed
\end{answer}

\begin{subexercise}
$O(n\log n)$ size circuits.
\end{subexercise}
\begin{answer}
Build a circuit inductively, with $T(n)$ gates.  Base case: with one
input, no gates are required ($T(1)=0$); the output is the majority
vote of the input.  Induction step: with $n$ inputs, divide them into
four sets $A, B, C$, and $D$, with each set containing $\ceil{n/4}$
inputs (possibly sharing at most one input if $n$ is not a multiple of
four).  Build six sub-circuits $AB, AC, AD, BC, BD$, and $CD$ that
compute the majority function for the union of each pair of sets,
e.g.~$AB$ computes $\textit{majority}(A\cup B)$; by the induction
hypothesis, each sub-circuit requires $T(\ceil{n/2})$ gates.  The
majority function is then computed by the circuit $AB*CD+AC*BD+AD*BC$
(requiring five gates).  The total number of gates is
$T(n)=6T(\ceil{n/2})+5$, and the solution to this recurrence relation
is $O(n\lg n)$.\qed
\end{answer}
\end{exercise}

\begin{exercise}
\begin{subexercise}[\sipser, 10.4]
Show that the parity function with $n$ inputs can be computed by a
branching program that has $O(n)$ nodes.
\end{subexercise}
\begin{answer}
Construct a branching machine with a series of pairs of nodes:
\vspace{2in}\\
This has $n$ pairs of nodes, thus $O(n)$ nodes.\qed
\end{answer}

\begin{subexercise}[\sipser, 10.5]
Show that the majority function with $n$ inputs can be computed by a
branching program that has $O(n^2)$ nodes.
\end{subexercise}
\begin{answer}
Construct a branching machine with a cascading array of nodes:
\vspace{2in}\\
This has $n/2$ rows of at most $n$ nodes, thus $O(n^2)$ nodes.\qed
\end{answer}
\end{exercise}

\begin{exercise}[\sipser, 10.7]
Show that BPP $\subseteq$ PSPACE.
\end{exercise}
\begin{answer}
A probabilistic Turing machine is just a \NTM\ with specific
probabilities at each nondeterministic step; any language recognized
by a probabilistic Turing machine is also recognized by a regular
\NTM\ with the same states and transitions, as long as the error
probability $\epsilon$ is strictly less than 1, i.e.~there exists at
least one accepting branch (the probability of this branch is
$2^{-k}$, which can never be 0).  BPP is the set of all languages with
error probability less than $\frac 13$, so BPP $\subseteq$ NP
$\subseteq$ PSPACE.\qed
\end{answer}

\begin{exercise}[\sipser, 10.11]
Let $M$ be a probabilistic polynomial time Turing machine and let $C$
be a language where, for some fixed $0<\epsilon_1<\epsilon_2<1$,
\begin{enumerate}
\renewcommand{\theenumi}{\alph{enumi}}
\item $w\not\in C$ implies Pr[$M$ accepts $w$] $\leq\epsilon_1$, and
\item $w\in C$ implies Pr[$M$ accepts $w$] $\geq\epsilon_2$.
\end{enumerate}
Show that $C\in$ BPP.
\end{exercise}
\begin{answer}
Let $\epsilon$ be the minimum of $\epsilon_1$ and $1-\epsilon_2$; then
\begin{enumerate}
\renewcommand{\theenumi}{\alph{enumi}}
\item $w\not\in C$ implies Pr[$M$ rejects $w$]
      $>1-\epsilon_1\geq 1-\epsilon$, and
\item $w\in C$ implies Pr[$M$ accepts $w$] $\geq\epsilon_2\geq 1-\epsilon$.
\end{enumerate}
So $M$ is a probabilistic polynomial time Turing machine that
recognizes $C$ with error probabilty $\epsilon$.
Since $\epsilon<\frac 12$, because either $\epsilon_1<\frac 12$ or
$\epsilon_2>\frac 12$, Lemma~10.5 holds: $M$ has an equivalent
probabilistic polynomial time Turing machine $M_2$ with an error
probability of $2^{-\lg 3}=\frac 13$.  Thus $C\in$ BPP.\qed
\end{answer}

\begin{exercise}[\sipser, 10.12]
Show that, if P = NP, then P = PH.
\end{exercise}
\begin{answer}
If P = NP, then \setname{SAT} $\in$ P; then a polynomial time oracle
\TM\ for a language in $\P^{\setname{SAT}}$ could decide
$\setname{SAT}$ in polynomial time rather than asking its oracle,
i.e.~$\P^{\setname{SAT}}$ = P = NP.  Then by Problem~9.9, NP = coNP.
If this is true, then $\Sigma_1$P = $\Pi_1$P = P.  Now, consider an
alternating machine $M$ for a language $L$ in $\Sigma_i$P for some
$i>1$; if $\Pi_{i-1}$P = P, then everything after the first run of
existential steps in $M$ can be replaced by a polynomial set of
deterministic steps, i.e.~$M$ is equivalent to a polynomial time \NTM\
and $L\in$ NP = P.  Thus, by induction on $i$, PH $\subseteq$ P; also,
P = $\Sigma_0$P $\subseteq$ PH, therefore P = PH.\qed
\end{answer}

\begin{exercise}
Prove or disprove: The circuit family for \textit{parity}$_n$ requires
$\Theta(n)$ gates.
\end{exercise}
\begin{answer}
Clearly, there must be at least one wire for each input, and they all
must eventually lead to the output wire; otherwise, flipping a bit
that was not connected to the output wouldn't properly flip the
computed parity.  Assume there is a solution that has $o(n)$ gates.
By the pigeonhole principle, there must be at least one gate that has
$n/o(n)=\omega(1)$ wires leading directly from the input to the input
of the gate, with no intervening gates, and whose output wire
eventually leads to the circuit's output; we can choose an $n$ big
enough so that this gate must have at least three inputs.  This gate
will produce the same output whether the input has exactly one \sym 1
or exactly two \sym 1s (namely \sym 0 for an AND gate or \sym 1 for an
OR gate); but these two cases have different parity.  Thus any circuit
implementing \textit{parity}$_n$ must have $\Omega(n)$ gates; since
there exists an implementation that uses $O(n)$ gates, the bound can
be tightened to $\Theta(n)$ gates.\qed
\end{answer}
\end{document}
