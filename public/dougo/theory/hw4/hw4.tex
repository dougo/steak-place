\documentclass{article}
\usepackage{amstex}
\usepackage{theorem}
\usepackage{fancyheadings}
\usepackage{program}

\setlength{\parindent}{0pt}	% Don't indent paragraphs.

%\addtolength{\voffset}{-0.25in}
%\addtolength{\textheight}{0.5in}
\addtolength{\hoffset}{-0.5in}	% Reduce left and right margins by 1/2 inch.
\addtolength{\textwidth}{1in}

\pagestyle{fancy}
\setlength{\headrulewidth}{0pt}

\lhead{COM 3350 Assignment 3}
\rhead{Doug Orleans (\texttt{dougo@@ccs.neu.edu})}

\theorembodyfont{\rmfamily}		% don't use italics
\theoremstyle{break}			% put a line break after the header
\newtheorem{exercise}{Exercise}
\theoremstyle{plain}
\newtheorem{subexercise}{}[exercise]
\renewcommand{\thesubexercise}{\alph{subexercise}.}
\newenvironment{answer}{\begin{quotation}\noindent}{\end{quotation}}

\newcommand{\sipser}{\textit{Sipser}}
\newcommand{\encoding}[1]{\ensuremath{\langle#1\rangle}}
\renewcommand{\qed}{~\ensuremath{\square}}
\newcommand{\Th}{\ensuremath{\operatorname{Th}}}
\newcommand{\Nat}{\ensuremath{\mathcal{N}}}
\newcommand{\tred}{\ensuremath{\leq_{\text{T}}}}
\newcommand{\ATM}{\ensuremath{A_{\textsf{TM}}}}
\renewcommand{\true}{\textsc{TRUE}}
\renewcommand{\false}{\textsc{FALSE}}
\newcommand{\defin}[1]{\textbf{\textit{#1}}}
\newcommand{\setname}[1]{\textit{#1}}
\renewcommand{\liminf}{\ensuremath{\lim_{n\to\infty}}}

\renewcommand{\AND}{\ \keyword{and}\ }
\newcommand{\ENDWHILE}{\untab\addtocounter{programline}{-1}}
\newcommand{\ENDFOR}{\untab\addtocounter{programline}{-1}}
\newcommand{\ENDIF}{\untab\untab\addtocounter{programline}{-1}}
\newcommand{\ACCEPT}{\keyword{accept}\ }
\newcommand{\REJECT}{\keyword{reject}\ }
\newcommand{\RETURN}{\keyword{return}\ }
\renewcommand{\keyword}[1]{\textbf{#1}} 
\renewcommand{\prognumstyle}{\textrm} 
\NumberProgramstrue 
\normalbaroutside 

\renewcommand{\a}{\stepcounter{subexercise}\textbf{\thesubexercise}\ }

\begin{document}

\begin{exercise}
\begin{subexercise}[\sipser, 7.11]
Call graphs $G$ and $H$ \defin{isomorphic} if the nodes of $G$ may be
reordered so that it is identical to $H$.  Let
$\setname{ISO}=\{\encoding{G,H}|~\text{$G$ and $H$ are isomorphic
graphs}\}$.  Show that $\setname{ISO}\in$ NP.
\end{subexercise}
\begin{answer}
The certificate is the reordering of the nodes of $G$, which is
$O(\encoding{G})$ in size and can be verified in $O(\encoding{G})$
time (with two tapes).\qed
\end{answer}

\begin{subexercise}[\sipser, 7.16(b)]
Let $G$ represent an undirected graph and let
\begin{align*}
\setname{LPATH} = \{\encoding{G,a,b,k}
|~&G \text{ contains a simple path of}\\
&\text{length at least $k$ from $a$ to $b$}\}.
\end{align*}
Show that \setname{LPATH} is NP-complete.  You may assume the
NP-completeness of \setname{UHAMPATH}, the Hamiltonian path problem
for undirected graphs.
\end{subexercise}
\begin{answer}
\setname{LPATH} is in NP; the certificate is the path.  
$\encoding{G,a,b}\in\setname{UHAMPATH}$ if and only if
$\encoding{G,a,b,|G|}\in\setname{LPATH}$; thus
$\setname{UHAMPATH}\leq_{\text{P}}\setname{LPATH}$, and since
\setname{UHAMPATH} is NP-complete, so is \setname{LPATH}.\qed
\end{answer}
\end{exercise}

\begin{exercise}[\sipser, 7.22]\label{neq}
Let $\phi$ be a 3cnf-formula.  An \defin{$\not=$-assignment} to the
variables of $\phi$ is one where each clause contains two literals
with unequal truth values.  In other words an $\not=$-assignment
satisfies $\phi$ without assigning three true literals in any clause.
\begin{subexercise}
Show that the negation of any $\not=$-assignment to $\phi$ is also an
$\not=$-assignment.
\end{subexercise}
\begin{answer}
Each clause in an $\not=$-assignment to $\phi$ contains either one or
two true literals.  Thus each clause in the negation contains
either one or two false literals, and hence either two or one true
literals; therefore, the negated $\not=$-assignment is also an
$\not=$-assignment.\qed
\end{answer}

\begin{subexercise}
Let $\not=$\setname{SAT} be the collection of 3cnf-formulas that have
an $\not=$-assignment.  Show that we obtain a polynomial time
reduction from \setname{3SAT} to $\not=$\setname{SAT} by replacing
each clause $c_i$
\begin{displaymath}
(y_1\vee y_2\vee y_3)
\end{displaymath}
by the two clauses
\begin{displaymath}
(y_1\vee y_2\vee z_i)\text{ and }(\overline{z_i}\vee y_3\vee b)
\end{displaymath}
where $z_i$ is a new variable for each clause $c_i$ and $b$ is a
single additional new variable.
\end{subexercise}
\begin{answer}
Let $f$ be the construction above; it is clearly polynomial time
computable.  We must show that, for every 3cf $\phi$,
\begin{displaymath}
\phi\in\setname{3SAT} \iff f(\phi)\in\setname{$\not=$SAT}.
\end{displaymath}

\pagebreak
\noindent First the $\implies$ direction:
\begin{quotation}
If $\phi\in$ \setname{3SAT}, then there is an assignment $\rho$ that
satisfies $\phi$; for every clause $c_i$ in $\phi$, $\rho$ makes at
least one of $y_1$, $y_2$, or $y_3$ true.  We can extend $\rho$ with
assignments to $b$ and each $z_i$ to construct a new assignment
$\rho'$: let $b=0$, and assign each $z_i$ according to the following
table:

\begin{center}
\begin{tabular}{ccc|c}
$y_1$	&$y_2$	&$y_3$	&$z_i$ \\
\hline \hline
0&0&1	&1 \\
0&1&0	&0 \\
0&1&1	&1 \\
1&0&0	&0 \\
1&0&1	&1 \\
1&1&0	&0 \\
1&1&1	&0 \\
\end{tabular}
\end{center}

\noindent We can see that $\rho'$ is an $\not=$-assignment for
$f(\phi)$: if $y_1=y_2$, then $z_i\not=y_1$, and if $y_3=b=0$, then
$\overline{z_i}=1$.  Thus $f(\phi)\in$ $\not=$\setname{SAT}.
\end{quotation}

\noindent Now the $\Longleftarrow$ direction:

\begin{quotation}
\noindent If $f(\phi)\in$ $\not=$\setname{SAT}, then there is some
assignment $\rho'$ that is an $\not=$-assignment for $f(\phi)$.
First, assume that $b=0$ in $\rho'$.  This implies that for each
clause $c_i$ in $\phi$, either $y_3=1$, or $z_i=0$ and either $y_1=1$
or $y_2=1$.  In any of these cases, $c_i=1$.  Thus we can let $\rho$
be $\rho'$ after removing the assignments for $b$ and each $z_i$, and
it will satisfy $\phi$.  Now, assume that $b=1$ in $\rho'$, and
consider $\overline{\rho'}$; by part (a), it is also an
$\not=$-assignment, and $b=0$ in it, so we can again apply the above
reasoning to construct $\rho$ that satisfies $\phi$ by removing the
assignments for $b$ and each $z_i$ from $\overline{\rho'}$.
Therefore, we can always construct an assignment that satisfies
$\phi$, and so $\phi\in\setname{3SAT}$.
\end{quotation}

\noindent Thus $f$ is a polynomial time reduction from \setname{3SAT} to
$\not=$\setname{SAT}.\qed
\end{answer}

\begin{subexercise}
Conclude that $\not=$\setname{SAT} is NP-complete.
\end{subexercise}
\begin{answer}
$\not=$\setname{SAT} is in NP; the certificate is the $\not=$-assignment.
By part (b), $\setname{3SAT}\leq_{\text{P}}\not=$\setname{SAT};
since \setname{3SAT} is NP-complete, $\not=$\setname{SAT} is thus
NP-complete as well.\qed
\end{answer}
\end{exercise}

\begin{exercise}[\sipser, 7.23]
A \defin{cut} in an undirected graph is a separation of the vertices
$V$ into two disjoint subsets $S$ and $T$.  The size of a cut is the
number of edges that have one endpoint in $S$ and the other in $T$.
Let
\begin{displaymath}
\setname{MAX-CUT} = \{\encoding{G,k}|~G
\text{ has a cut of size $k$ or more}\}.
\end{displaymath}
Show that \setname{MAX-CUT} is NP-complete.  You may assume the result
of Exercise~\ref{neq}.
\end{exercise}
\begin{answer}
\setname{MAX-CUT} is in NP; the certificate is a cut of size $k$ or
more.  To show that \setname{MAX-CUT} is NP-complete, we give a
polynomial-time reduction from \setname{$\not=$SAT} to it.  Let $\phi$
be a 3cf with $n$ variables and $k$ clauses.  Construct an undirected
graph $G$ consisting of $n$ variable gadgets and $k$ clause gadgets:
\begin{itemize}
\item The gadget for a variable $x$ is a collection of $3k$ nodes
labeled with $x$ and another $3k$ nodes labeled with $\overline{x}$,
together with an edge from every $x$ node to every $\overline{x}$
node.  In other words, the gadget is the complement of two
$3k$-cliques.
\item The gadget for a clause $(a \vee b \vee c)$ is a triple of edges
connecting one triple of nodes labeled $a$, $b$, and $c$.  No two
clause gadgets have any nodes in common.
\end{itemize}
$G$ has $6kn$ nodes and $(3k)^2n + 3k$ edges, and every node has
either $3k$ or $3k+2$ incident edges.  Now, given an
$\not=$-assignment for $\phi$, we can construct a cut for $G$: put
each node into $S$ or $T$ if its label has a value of $0$ or $1$,
respectively.  Since every node labeled $x$, in one set, is connected
to $3k$ nodes labeled $\overline{x}$, in the other set, all of the
variable gadget edges are in the cut; also, since every triangle has
exactly one node in one set and two in the other, two edges of every
clause gadget are in the cut.  Thus, the cut is of size $(3k)^2n+2k$.
In other words, if $\encoding{\phi}\in$ \setname{$\not=$SAT}, then
$\encoding{G,(3k)^2n+2k}\in$ \setname{MAX-CUT}.  To see that the
inverse also holds, notice that since a triangle can contribute at
most two edges to a cut, $(3k)^2n+2k$ is the maximum possible size
of a cut of $G$; thus, any cut of this size must include all of the
variable gadget edges and two edges from every clause gadget, and
therefore must correspond to an $\not=$-assignment for $\phi$.
Therefore, if $\encoding{G,(3k)^2n+2k}\in$ \setname{MAX-CUT}, then
$\encoding{\phi}\in$ \setname{$\not=$SAT}.  Hence, we have shown that
$\not=$\setname{SAT} $\leq_{\text{P}}$ \setname{MAX-CUT}, which means
that \setname{MAX-CUT} is NP-complete.\qed
\end{answer}

\begin{exercise}[\sipser, 7.30]
Let $\setname{MAX-CLIQUE}=\{\encoding{G,k}|~$ the \emph{largest} clique
of $G$ has $k$ vertices\}.  Whether \setname{MAX-CLIQUE} is in NP is
unknown.  Show that if P $=$ NP, then \setname{MAX-CLIQUE} is in P,
and a polynomial time algorithm exists that, for a graph $G$, finds
one of its largest cliques.
\end{exercise}
\begin{answer}
If P $=$ NP, then \setname{CLIQUE} $\in$ P; thus, we can iterate $k$
from 1 to $|G|$, and find the largest $k$ for which
$\encoding{G,k}\in$ \setname{CLIQUE}.  This is a linear repetition of
a polynomial time algorithm, and thus also polynomial time,
i.e.~\setname{MAX-CLIQUE} is in P.  To find one of its largest
cliques: [idea from Lars Hansen]

\begin{programbox}
|clique| \gets G
\FOREACH |vertex| \in G
	|newclique| = |clique| - \{|vertex|\}
	\IF \encoding{|newclique|,k} \in \setname{MAX-CLIQUE}
	\THEN	|clique| = |newclique|
	\ENDIF
\ENDFOR
\RETURN |clique|
\end{programbox}

We test each node in the graph to see if removing it changes the max
clique size.  If not, then it's not in the max clique, so we can
discard it.  After examing all the nodes this way, the nodes we have
not discarded will be exactly the set of nodes in the max clique.  (If
there are more than one of them, then this algorithm will find the
``last'' max clique.)  Again, this is a linear repetition of a
polynomial time algorithm (i.e.~line~4), thus the algorithm is
polynomial time.\qed
\end{answer}

\pagebreak
\begin{exercise}[\sipser, 7.34]\label{3color}
A \defin{coloring} of a graph is an assignment of colors to its nodes
so that no two adjacent nodes are assigned the same color.  Let
\begin{align*}
\setname{3COLOR} = \{\encoding{G}|~
&\text{the nodes of $G$ can be colored with three colors such that}\\
&\text{no two nodes joined by an edge have the same color}\}.
\end{align*}
Show that \setname{3COLOR} is NP-complete.
\end{exercise}
\begin{answer}
\setname{3COLOR} is in NP; the certificate is the coloring.

\ldots
\end{answer}

\begin{exercise}
Define \setname{2COLOR} analogously to \setname{3COLOR} in
Exercise~\ref{3color}.  Show that \setname{2COLOR} is in P.
\end{exercise}
\begin{answer}
Since there are only two colors to choose from, every connected node
in the graph is uniquely constrained as to which color it may be
(namely, the opposite of its neighbors).  Thus, we can attempt to
color each connected component of the graph via a simple
depth-first-traversal (regarding the graph as undirected), alternating
between two colors for the node marking.  If a traversal ever reaches
a node which is already colored the same color as the previous node
visited, then reject.  If all nodes in the graph can be colored this
way, then accept.  This algorithm is linear-time in the number of
nodes in the graph, thus \setname{2COLOR} is in P.\qed
\end{answer}
\end{document}
