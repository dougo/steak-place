\documentclass{article}
\usepackage{tex2page}
\usepackage{amsmath}

\newcommand{\defn}[1]{\textbf{#1}}
\newcommand{\code}[1]{\texttt{#1}}
\newcommand{\parm}[1]{\textit{#1}}

\title{Ph.D. Thesis Proposal: \\
Still Needs A Title}
\author{Doug Orleans}

\begin{document}

%\begin{htmlonly}
%This document is also available in
%\htmladdnormallink{postscript}{../proposal.ps} form.
%\end{htmlonly}

\maketitle

%\tableofchildlinks

\begin{center}
DRAFT --- DO NOT DISTRIBUTE
\end{center}

\begin{abstract}
\textbf{Thesis}: The benefits of incremental programming can be
expanded beyond those provided by traditional object-oriented
inheritance and dynamic dispatch by expressing behavior as
parameterized bundles of decision point branches.
\end{abstract}

\section{Introduction}

One of the chief benefits of object-oriented programming is the
support for \defn{incremental programming} \cite{mezini-thesis}, that
is, the construction of new program components by specifying how they
differ from existing components: a subclass can extend or override the
behavior of methods that exist in the superclass.  Incremental
programming allows for clean separation of concerns, because the
differences of the new component can be specified separately from the
existing component, but also allows for reuse, by not requiring the
existing component to be copied or modified.
[I think there's a more important point to be made here, but I
can't quite put my finger on it.  Something to do with how components
are combined?  Also, should I explain why separation of concerns and
reuse are desirable, or is that obvious?]

When a subclass method extends or overrides a superclass method,
essentially a new branch is added to a decision point in the program
execution: when a message matching the methods' signature is sent to
an object, if the object is an instance of the subclass, then execute
the subclass method, otherwise run the superclass method.  Without
support for incrementally adding this branch to the decision point, it
would have to be coded explicitly as a \code{switch} statement or
something analogous, and in order to extend the branch to handle new
kinds of objects, the \code{switch} statement would have to be copied
or modified.
[Again, I think the don't-need-to-modify angle is sort of a red
herring.  Compare with pasting pieces of code together that happen to
live in different files.]
The widespread success of object-oriented programming is largely due
to this improvement over procedural programming, replacing
\code{switch} statements that dispatch on different kinds of objects
with separately-specified branches.
[do I need to support this claim?]
[Quote from Mira's thesis, page 79:
Avoiding the ``case''-like style of procedural programming with
respect to dispatching kind-specific behavior is an essential factor
for the qualitative progress in reusability attributed to the
object-oriented paradigm.]

This idea can be extended to all of the decision points in a program,
not just those that distinguish different kinds of objects.  This
paper proposes a new model of programming, in which the behavior of a
program can be specified as collections of decision point branches.
Just as the behavioral atoms of procedural programming were split into
the subatomic particles of methods in object-oriented programming,
methods can be split into still smaller quantum units of behavior.
This is achieved by allowing more precise expression of the condition
determining when a branch should apply to a decision point.  A method
is a branch whose condition involves the run-time type of the receiver
object of the message.  Branch conditions should be able to involve
other data as well: the state of the receiver object and the other
message arguments; the message itself; the enclosing branch from which
the message was sent; and global or thread-local properties of the
system at the time of the message send.

Many behavioral design patterns \cite{design-patterns} make these
sorts of conditions expressible as extensible decision points by
converting the decision into a dispatch based on receiver-type,
because that's the only kind of dispatch that most object-oriented
languages provide.  The State pattern, for example, involves making a
class for each different state that an object can be in, and sending a
message to a state object to decide what behavior to execute based on
the state object's class.  However, it would be much more natural to
express the state-based dispatch directly as a set of branches with
the desired state conditions, without the conceptual and practical
overhead of creating new classes and maintaining state objects.

Ideally the condition governing the applicability of a branch should
be allowed to be any arbitrary computation involving any element of
the program state at the decision point, although practical
considerations such as tractability and efficiency limit this
somewhat.  Still, this goal can be approached by reifying decision
points as program entities, and specifying branch conditions as
logical combinations of predicate expressions over these decision
point entities.  The details of these entities, as well as a prototype
condition expression language and mechanisms for determining how to
choose between multiple applicable branches, are given in
section~\ref{section:fred}.

Once behavior has been split into these quantum units, they need to be
collected again into larger components, just as methods are organized
into classes.  The main reason methods are grouped together into
classes is data abstraction: they need to share access to private data
values (namely, the fields of the class).  This can also be achieved
using lexical scoping and environments with dynamic
extent. [Citation?]  In this system, an environment is not only a set
of bindings of names to types, values, and messages, but also a set of
branches that may apply to message sends made in the context of the
environment.  In addition, components can be made more reusable by
parameterization: a component should map input environments to output
environments, exporting both name bindings and new branch definitions,
which can refer to parts of the input environment parameter in both
the condition predicate and the branch body.  Some ideas for the
design of such components are presented in
section~\ref{section:bundles}.

More precise branch condition expression allows for finer granularity
between behavioral units in a program, which in turn allows concerns
to be more easily separated.  In particular, concerns that would need
to be tangled together with or scattered across other concerns using
traditional object-oriented design can be modularized well.  This is
also the goal of aspect-oriented programming \cite{AOP}.
Section~\ref{section:AOP} discusses how many of the features of
AspectJ \cite{AspectJ} can be thought of in terms of decision point
branches.

The paper concludes with a discussion of other related work in
section~\ref{section:related-work}, followed by a summary of the
contributions of my work and future research goals in
section~\ref{section:contributions}.

\section{Fred: A Prototype Language of Decision Point Branches}
\label{section:fred}

In order to experiment with the idea of expressing behavior as
decision point branches, I have developed a small prototype language
called Fred, implemented as a set of procedures and macros in MzScheme
\cite{MzScheme}.  I rely on a number of features of MzScheme to avoid
having to reproduce them in the language design of Fred, such as
first-class procedures, lexical scoping, and a structure mechanism.

There are three basic kinds of entities in Fred: \defn{messages},
\defn{branches}, and \defn{decision points}.  The programmer never
creates decision points by hand, only messages and branches.  Messages 
and branches are defined with the special forms \code{define-msg} and
\code{define-branch}; decision points can be inspected in the code of a
branch through the special variable \code{dp} with accessor functions
such as \code{dp-msg} and \code{dp-args}.  A simple example will serve 
to illustrate how these pieces can be put together:

\begin{verbatim}
(define-msg fact)
(define-branch (and (eq? (dp-msg dp) fact)
                    (= (car (dp-args dp)) 1))
  1)
(define-branch (eq? (dp-msg dp) fact)
  (let ((x (car (dp-args dp))))
    (* x (fact (- x 1)))))
\end{verbatim}

The first expression creates a new unique message entity and binds it
to the name \code{fact} in the current environment.  Messages are
implemented as procedures so that a message can be sent to a list of
arguments simply by applying the message to the arguments.

The second expression creates a new branch to handle the base case of
the factorial function and adds it to the current
environment\footnote{Actually this is not implemented yet; all
branches are stored in a single global table.}.  The first subexpression
of the \code{define-branch} special form is a condition predicate that
is evaluated at every decision point; if it evaluates to true, then
the remaining subexpressions in the form are evaluated.  In this example,
if the message of the decision point is equal to \code{fact} and the
first element of the argument list is equal to 1, then the value 1 is
returned as the value of the decision point.  In other words, sending
the message \code{fact} to the value 1 evaluates to 1.

The third expression creates a new branch to handle the alternative
case; if the message \code{fact} is sent to any argument \code{x},
then \code{fact} is sent to \code{x-1} and the result is multiplied by
\code{x}.  Note that the predicates of both branches evaluate to true
when the message \code{fact} is sent to 1; when two or more branches
apply to a given decision point, the branch whose predicate is most
specific has higher precedence.  Specificity is determined by logical
implication: if a predicate $P_1$ implies another predicate $P_2$,
i.e. $P_2$ is always true when $P_1$ is true, then $P_1$ is more
specific than $P_1$.  In this example, the first branch is more
specific than the second, because \code{(and X Y)} always implies
\code{X}.

Explicitly applying accessors to the decision point can be tedious,
so there is some syntactic sugar available to make branch definition a 
bit more concise:

\begin{verbatim}
(define-method fact (x) & (= x 1)
  1)
(define-method fact (x)
  (* x (fact (- x 1))))
\end{verbatim}

The \code{define-method} special form creates a branch whose predicate
compares the message of a decision point to the given message, as well
as providing names for the arguments to be bound to.  It also defines
the message if it is not already defined.  Further syntactic sugar
allows up to one test per parameter to be moved into the formals
specifier, as long as the formal argument is named as the first
operand of the test expression:

\begin{verbatim}
(define-method fact ((= x 1))
  1)
\end{verbatim}

For a longer example, consider a library to implement \defn{cords}, a
data structure for strings that optimizes the concatenation operation
by storing a tree of fragments rather than copying arrays of
characters into a single array for every
concatenation~\cite{cords}~\cite{cords-in-AspectJ}.  I will start with
the basic structure and behavior and show how features can be added to
the library incrementally without modifying any code.

First, we define three data types, \code{cord}, \code{flat-cord}, 
and \code{concat-cord}, using MzScheme's \code{define-struct} form.  
\code{flat-cord} is just a wrapper around Scheme strings, while
\code{concat-cord} has cords as left and right children; they both
inherit from the abstract base class \code{cord}:

\begin{verbatim}
(define-struct cord ())
(define-struct (flat-cord struct:cord)
  (string))
(define-struct (concat-cord struct:cord)
  (left right))
\end{verbatim}

The \code{define-struct} form generates procedures for creating
structure instances and accessing the fields, as well as a predicate
procedure for testing whether an entity is an instance of the
structure type; the name of the predicate is formed by appending a
question mark to the type name.  A type predicate also returns true
for all instances of subtypes; this must be declared to Fred's
implication system so that it can figure out when one predicate
implies another:

\begin{verbatim}
(declare-implies 'flat-cord? 'cord?)
(declare-implies 'concat-cord? 'cord?)
\end{verbatim}

Now we can start defining behavior over these types.  First, the
concatenation operation, which can handle both cords and strings for
either argument, and produces a cord:

\begin{verbatim}
(define-method concat ((string? l) r)
  (concat (make-flat-cord l) r))
(define-method concat ((cord? l) (string? r))
  (concat l (make-flat-cord r))
(define-method concat ((cord? l) (cord? r))
  (make-concat-cord l r))
\end{verbatim}

Then we can define a length operator for the two kinds of cords:

\begin{verbatim}
(define-method len ((flat-cord? x))
  (string-length (flat-cord-string x)))
(define-method len ((concat-cord? x))
  (+ (len (concat-cord-left x)) (len (concat-cord-right x))))
\end{verbatim}

as well as an indexed reference operator:

\begin{verbatim}
(define-method ref ((flat-cord? x) (integer? i))
  (ref (flat-cord-string x) i))
(define-method ref ((concat-cord? x) (integer? i))
  (ref (concat-cord-left x) i))
(define-method ref ((concat-cord? x) (integer? i))
  & (>= i (len (concat-cord-left x)))
  (ref (concat-cord-right x) (- i (len (concat-cord-left x)))))
\end{verbatim}

Note that the \code{ref} operator is split into two branches, one for
each branch of the tree.  The predicate of the second branch is more
specific than that of the first branch, so it has precedence when they 
are both applicable.

We can add new subtypes to \code{cord} just as easily as in a
traditional object-oriented language.  For example, to optimize the
substring operation, we can add a \code{substring-cord} type with
new branches for the existing operations:

\begin{verbatim}
(define-struct (substring-cord struct:cord)
  (base offset length))
(declare-implies 'substring-cord? 'cord?)

(define-method len ((substring-cord? x))
  (substring-cord-length x))
(define-method ref ((substring-cord? x) (integer? i))
  (ref (substring-cord-base x) (+ i (substring-cord-offset x))))
\end{verbatim}

We can also add new subtypes that aren't implemented as structure
types; for example, we can optimize the case of concatenating an empty
cord to another cord, by defining an \code{empty-cord} predicate:

\begin{verbatim}
(define (empty-cord? x)
  (and (cord? x) (= (len x) 0)))
(declare-implies 'empty-cord? 'cord?)

(define-method concat ((empty-cord? l) (cord? r))
  r)
(define-method concat ((cord? l) (empty-cord? r))
  & (not (empty-cord? l))
  l)
\end{verbatim}

Note that the extra condition in the predicate of the second branch is
required to ensure the two branches don't overlap (when concatenating
two empty cords); neither branch is more specific than the other, so
this would result in a ``message ambiguous'' error.  Another way to
avoid this would be to add a third branch to handle the overlap case
explicitly:

\begin{verbatim}
(define-method concat ((empty-cord? l) (empty-cord? r))
  l)
\end{verbatim}

Now suppose we want to optimize the cords library by keeping the tree
structure balanced, so that the \code{ref} operator doesn't degenerate 
to linear search.  This involves two things: keeping track of the
depth of the tree, and re-balancing the tree after a concatenation if
the depth is too big.  We could modify the existing code to add these
changes, but to achieve better separation of concerns, we should use
the principle of incremental programming and implement the
modification by expressing the new behavior as additions to the
existing behavior.  First, instead of modifying the data structures to 
add a \code{depth} field, we can create a new table and provide
accessors that acts the same as field accessors would:

\begin{verbatim}
(define *depth-table* (make-hash-table 'weak))

(define (compound-cord? x)
  (or (concat-cord? x) (substring-cord? x)))
(declare-implies 'concat-cord? 'compound-cord?)
(declare-implies 'substring-cord? 'compound-cord?)

(define-method set-depth! ((compound-cord? x) (integer? d))
  (hash-table-put *depth-table* x d))
(define-method depth ((compound-cord? x))
  (hash-table-get *depth-table* x))
(define-method depth ((flat-cord? x))
  0)
\end{verbatim}

In order to add the \code{depth} field to multiple types at once, we
make a new predicate that acts like a union type-- again, without
actually needing to implement a data structure for the type.
[How can this type be extended, though?  If we need to add another new type?]

Now we need to extend the behavior of \code{concat} to update the
depth field and balance the tree if needed:

\begin{verbatim}
(define-around (eq? (dp-msg dp) concat)
  (let ((c (invoke-next-branch)))
    (set-depth! c (max (depth (concat-cord-left c))
		       (depth (concat-cord-right c))))
    (ensure-balanced c)))
(define-method ensure-balanced ((concat-cord? x))
  & (> (depth x) *max-depth*)
  (balance x))
(define-method ensure-balanced ((concat-cord? x))
  x)
\end{verbatim}

The \code{invoke-next-branch} procedure invokes the branch that has
the next highest precedence after the current branch.  However, this
branch is an \defn{around} branch, a special kind of branch that has
higher precedence than all non-around branches.  Otherwise, because
its condition is more general than the other branches that are
applicable to \code{concat} message sends, it would have the lowest
precedence.  

[Jumping Aspect problem; dp-previous, dp-within]

\section{Bundles}
\label{section:bundles}

The previous section describes a language for defining behavior as a
number of independent branches; this section will briefly outline some
ideas for collecting these branches, along with the types, messages,
and data that they refer to, into modular components.  I will call
these components \defn{bundles}, since they are collections of
branches.

As explained in the introduction, a bundle should map input
environments to output environment, where environments consist of both
name bindings and branches.  There are two operations necessary:
bundle definition, which creates a bundle as a set of definitions and
expressions parameterized over the input environment and producing an
output environment; and bundle invocation, which provides an input
environment, executes the definitions and expressions inside the
bundle, and merges the output environment with the current
environment.

Bundle definition can be implemented as procedure definition, where
the return value of the procedure is the output environment.  Bundle
invocation is then just procedure application, with an additional step
for merging the environments.

[Example]

[Composition of bundles]

\section{Aspect-Oriented Programming}
\label{section:AOP}

A bundle is very similar to an aspect in AspectJ.  Decision points,
predicates, and branches are analogous to join points, pointcuts, and
advice.  Abstract pointcuts serve the purpose of parameterizing
aspects; the actual values of the pointcuts are provided as concrete
pointcuts in aspects that inherit from the abstract aspects.

However, composition of bundles is more flexible than inheritance of
aspects.  An aspect in AspectJ is also a Java class, and can have
instances; aspect instances can hold field values, and aspect methods
can dispatch on aspect instance types.  Thus it is natural to use
aspect inheritance to implement reuse of these fields and methods.
But if inheritance must also be used for parameterizing aspects with
concrete pointcuts, there is the possibility for conflict.  

[pointcut designators <-> predicate expressions]

\section{Related Work}
\label{section:related-work}

Branches are directly inspired by Ernst et al's predicate
dispatching~\cite{predicate-dispatch}.  In that system, each method has a
predicate expression over the argument values, where the predicate
expression language allows logical combinations of ``is-a'' tests or
arbitrary test expressions in the base language.  Method precedence is
based on logical implication, where ``is-a'' atoms are compared based
on the subtype relation.  Branch conditions generalize their method
predicates by being expressions over decision points, which include
the argument values but also the message value and calling context.
Also, there is no method combination (a la \code{around} branches) in
their system; there is no way to customize the method precedence
relation.

Bundles are inspired by Flatt and Felleisen's units~\cite{units}.
Units are reusable modules, parameterized by sets of import bindings
and producing sets of export bindings.  Units are linked together
statically into compound units by connecting the imports and exports
of other units together.  Bundles generalize units by expanding the
imports and exports to environments that include sets of branches in
addition to variable bindings.  I have not yet designed composite
bundles but I plan to use the same external-linkage strategy as that
of compound units.

Mezini's Rondo language~\cite{Rondo} was designed to address
context-dependent variations while supporting incremental programming.
A Rondo program consists of a set of \defn{adjustments}, which
encapsulate sets of classes that extend other classes (in a
generalized sense, without subtyping) when certain conditions hold.
Rondo and Fred share the same goals, but approach them from different
angles; Rondo is still class-oriented, while Fred is method-oriented.
Focusing on methods as the quantum unit of behavior allows a
finer-grained separation of concerns than focusing on classes.  Also,
adjustments are not parameterized, although successor systems such as
Adaptive Plug and Play Components with Lieberherr~\cite{APPC} add the
notion of adapters, which link paramaterized components with flexible
mappings.  I regard my system as a lower-level set of building blocks
on top of which more complex structures such as adapters can be layered.

Aksit's composition filters~\cite{composition-filters} provide a
similar kind of functionality as bundles of branches.  A composition
filter is a set of message filters that is attached to a class; all
messages sent to instances of that class are first handled by the
filters, which may perform some action (such as delegating the message
to some other code) based on predicate expressions being satisfied.
This approach is class-oriented, like Rondo, and the grouping of
filters into attachments to classes is not as flexible as being able
to group arbitrary sets of branches into bundles.  A recent
development in composition filters research is the ability to attach
filters to sets of arbitrary objects, which may be equivalent to
bundles of branches; more investigation is needed.

\section{Contributions and Research Goals}
\label{section:contributions}

[Extensible predicates.  More espressive branch combination.]
[More flexible adapters, with a higher level syntax.  One-to-many mapping.]
[Efficiency.]

\bibliography{proposal}
\bibliographystyle{alpha}

\end{document}
