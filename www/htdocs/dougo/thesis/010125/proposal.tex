\documentclass[12pt]{article}
%\documentclass[openbib]{article}
\usepackage{html}

% Grammar environment

\newenvironment{grammar}{
  \renewcommand{\:}{\goesto{}}
  \renewcommand{\|}{$\vert$}
  \tt \frenchspacing
  \begin{tabbing}
    %\qquad\quad \= 
    \qquad \= $\vert$ \= \kill
  }{\unskip\end{tabbing}}

% Commands for grammars
\newcommand{\arbno}[1]{#1\hbox{\rm*}}  
\newcommand{\atleastone}[1]{#1\hbox{$^+$}}

\newcommand{\goesto}{$\longrightarrow$}

% non-terminal in a grammar
\newcommand{\nt}[1]{{\noindent\hbox{\rm$\langle$#1$\rangle$}}}


% Reduce margins by 1/2 inch.
\addtolength{\voffset}{-0.5in}
\addtolength{\textheight}{1in}
\addtolength{\hoffset}{-0.5in}
\addtolength{\textwidth}{1in}

\title{Ph.D. Thesis Proposal: \\
The Design and Implementation of Aspect-Oriented Languages}
\author{Doug Orleans}

\begin{document}

\begin{htmlonly}
This document is also available in
\htmladdnormallink{postscript}{../proposal.ps} form.
\end{htmlonly}

\maketitle

%\tableofchildlinks

\begin{center}
DRAFT --- DO NOT DISTRIBUTE
\end{center}

\section{Introduction}

\defn{Aspect-oriented programming} (AOP) \cite{AOP} is a programming
methodology gaining much attention in the software engineering field.
It has the potential to increase code reuse, improve software
maintainability, and shorten development time, by enabling the
separation of concerns that are not easily separated using traditional
methodologies such as object-oriented programming.  AOP is not tied to
any particular programming language, but most current work is focused
on the development of AspectJ \cite{AspectJ}, a general-purpose
aspect-oriented extension of Java \cite{JLS}.  Java is a wise choice for a
language to extend; its popularity will make it easier to spread the
idea of AOP, and it's a simple enough language to facilitate such
extension.  However, there are two main reasons to consider
aspect-oriented extensions of other languages: to bring the benefits
of AOP to users of other languages, and to make sure that the
development of the AOP methodology isn't being shaped too much by the
properties of Java.  By thinking about AOP in the context of other
languages, we might come up with fresh insights into the general
nature of AOP, which might even pave the way for a formal semantics of
AOP.  Also, designing and implementing aspect-oriented extensions to
other languages could lead to new design and implementation techniques
that could apply to AspectJ or alternate aspect-oriented extensions to
Java.

What languages should be considered for extension?  There are lots to
choose from.  Rather than using specific languages, however, I plan to
examine particular language \emph{features}, separately and in
combination.  Considering a language feature individually will limit
the variables involved, and will lead to a clearer understanding of
the impact of that feature on an aspect-oriented language.  Then when two
features whose impact has been quantified are combined, the impact of
the combination itself can be clearly analyzed as well.  A number of
aspect-oriented languages can then be constructed on demand from this
building-block set of language features.

What language features, then, should be considered?  My strategy is to
concentrate on features which are generalizations of features in Java.
In particular, the features I plan to study are multiple dispatch,
intercessory reflection, and object-basedness.  (These terms are
defined and explained in later sections.)  The main advantage of
choosing generalizations of features of Java is that then any program
written in AspectJ can be reimplemented in the more general
aspect-oriented languages that will result.

My implementation strategy is to develop prototype languages by
embedding them in Scheme \cite{R5RS}.  Scheme is well-suited for
language development due to its syntax extension mechanisms.  For
example, a simple CLOS-like language with a meta-object protocol
\cite{AMOP} called Tiny CLOS \cite{tiny-clos} has been embedded in
Scheme.  In some cases it might be worthwhile to modify the AspectJ
compiler \code{ajc} directly to handle a new language feature, but
most of the time it will probably be better to have all the
experiments working in the same environment, that of Scheme; in
particular, that will make it easier to combine language features.

\section{Aspect-Oriented Programming}

% Explain AOP ideas and benefits.

Aspect-oriented programming is a way to achieve the benefits of
separation of concerns when a concern is not easily separated using
procedural or object-oriented decomposition.  For example, suppose you
want to add tracing statements to your program, for debugging: every
method execution should print a message before it starts and after it
finishes.  The functionality of printing out messages, deciding what
level of detail should be printed, keeping track of indentation, etc.,
can be encapsulated into a tracing module.  But you would still have
to add a call to the beginning and end of the body of every method in
your system, in order to invoke the tracing module.  These calls are
scattered throughout the code, rather than being organized in one
place; moreover, the body of every method is now tangled with these
extra calls, which can distract from the purpose of that particular
method.  With AOP, instead of placing these calls in the body of every
method, you would instead define an \defn{aspect} which contained
instructions for placing the calls, and combine the aspect with your
program using an \defn{aspect weaver}.  The aspect could be written in
a special-purpose tracing aspect language, or in a general-purpose
aspect language (such as AspectJ).  The former is easier for the
aspect writer, since a program in a domain-specific language is
usually more concise than in a general-purpose language, but more work
for the aspect language implementor, especially if the aspect writer
needs to do something that isn't possible in the aspect language and
the language needs to be redesigned.

The benefits of this kind of separation of concerns (known in the AOP
literature as \defn{cross-cutting} concerns \cite{AOP}) are the same
as any separation of concerns: it makes software easier to understand, 
easier to change, and easier to reuse and adapt \cite{MDSOC}.  It does 
these by reducing scattering and tangling, by localizing all code
relevant to one concern into one program structure.

How does AOP differ from other kinds of separation of concerns?  One
good litmus test for whether a language is aspect-oriented or not is
whether the language has \defn{quantification} and \defn{obliviousness}
\cite{ob-quant}.  Quantification is the ability to quantify a piece of 
code over other program elements (either static or dynamic); in other
words, you can say ``whenever condition C arises, perform action A.''
What can be specified in condition C depends on the degree of
quantification in the AO language; it is generally on the order of
``whenever method M is executed'', but it could be as fine grained as
``whenever variable v is accessed'' \cite{SOP} or ``whenever line L is
executed'' \cite{come-from}.  Obliviousness is the ability to quantify 
code over programs that have been written oblivious to this code.  In
other words, a program does not have to explicitly cooperate with an
aspect that refers to the program; in the example above, the code
that's being traced does not need to be modified or otherwise written
in any particular way in order to enable the tracing aspect to be
implemented.  A certain degree of obliviousness is essential for any
code to be reusable, but ideally AOP can be used with code that was
not intended to be reused at all.

\section{AspectJ}

% Overview of AspectJ syntax & semantics.

AspectJ \cite{AspectJ} is a general-purpose aspect language that is an
extension of Java \cite{JLS}.  An aspect definition is like a class
definition with some extra parts: \defn{pointcuts}, \defn{advice}, and 
\defn{introductions}.  A pointcut defines where to weave, and advice 
and introductions define what to weave.

A pointcut specifies a set of \defn{join points}, which are points in
the dynamic execution of a Java program.  For example,
\code{executions(void f())} is a pointcut containing all executions
of methods named \code{f} with no arguments returning void.
\code{instanceof(C)} is a pointcut containing all join points that
occur when the currently-executing object (i.e. the value of
\code{this}) is an instance of class \code{C}.  Pointcut
designators can be compounded with boolean operators; thus
\code{instanceof(C) \&\& executions(void f())} is the intersection of
the previous two pointcuts.  Wildcards can be used in pointcut
specifiers as well: \code{executions(* f(..))} contains all
executions of methods named \code{f}, regardless of argument types
or return types.

A piece of advice consists of a pointcut, a block of code, and an
instruction for when to execute the code relative to the join points
in the pointcut: \code{before} them, \code{after} them, or instead
of them (\code{around}).  In the case of a piece of \code{around}
advice, the block of code can call the special function
\code{proceed()} to execute the join point.  The special variable
\code{thisJoinPoint} is also available; it contains an object
describing the current join point and can be queried with methods such 
as \code{getExecutingObject()} and \code{getSignature()}.

An introduction lexically inserts fields and methods into the text of
a particular class.  It can also add an interface to a class's
\code{implements} list.

Here is a simple example of aspect definition, for our tracing example 
described above (a simplified version of the Trace example from the
AspectJ primer \cite{AspectJ}):
\begin{verbatim}
aspect Trace {
  void traceEntry(Signature s) { System.out.println("Entering " + s); }
  void traceExit(Signature s)  { System.out.println("Exiting " + s); }
  abstract pointcut classes();
  pointcut methods(): classes() && executions(* *(..));
  pointcut constructors(): classes() && executions(new(..));
  before(): methods() || constructors() {
    if (trace) traceEntry(thisJoinPoint.getSignature());
  }
  after(): methods() || constructors() {
    if (trace) traceExit(thisJoinPoint.getSignature());
  }
  introduction * { boolean trace; }
}
aspect TraceMyClasses extends Trace {
  pointcut classes(): instanceof(Foo || Bar || Baz || Garply);
}
\end{verbatim}
Here, tracing can be turned on per-object by setting the \code{trace}
field of that object to \code{true}.  This aspect would be woven into
the rest of the code by invoking the AspectJ compiler:
\begin{verbatim}
% ajc Trace.java TraceMyClasses.java Foo.java Bar.java Baz.java Garply.java
\end{verbatim}
which produces \code{.class} files for the five classes.  (\code{ajc}
can also be run with the \code{-preprocess} option, generating new
\code{.java} files into a separate directory.)  Note also that the
Trace aspect can be reused with a different set of classes by defining 
a new sub-aspect that defines the \code{classes()} pointcut differently.

\section{Multiple Dispatch}

% Explain multiple dispatch vs. single dispatch, outline AOP for MultiJava.

A typical object-oriented language involves message sends and method
invocations.  A message is sent to an object, and a method is chosen
to be invoked based on the dynamic (run-time) type of the object the
message was sent to (the receiver).  In Java, the static
(compile-time) types of the argument expressions also plays a part in
the method selection, but not the dynamic types.  For example, suppose
these classes are defined:
\begin{verbatim}
class P {
  void f(P p) { System.out.println("PP"); }
  void f(C c) { System.out.println("PC"); }
}
class C extends P {
  void f(P p) { System.out.println("CP"); }
  void f(C c) { System.out.println("CC"); }
}
\end{verbatim}
and suppose we have the following variables declarations:
\begin{verbatim}
P p = new P(); C c = new C(); P c2 = new C();
\end{verbatim}
Then the expressions \code{p.f(p)}, \code{p.f(c)}, \code{c.f(p)},
and \code{c.f(c)} would all print the expected things, namely
\code{PP}, \code{PC}, \code{CP}, and \code{CC}, respectively.
However, while \code{c2.f(p)} would print \code{CP}, because the
receiver \code{c2} is an object of class \code{C}, \code{p.f(c2)}
would print \code{PP}, because the static type of the \code{c2}
variable is class \code{P}.

This behavior can surprise even the experienced Java programmer
sometimes.  A more symmetric solution is to use
\defn{multiple dispatch}, where the dynamic types of all the arguments 
play an equal part in method selection, instead of Java's
single-dispatch mechanism.  In essence, a message is sent to a
sequence of objects, rather than a message with a sequence of
arguments being sent to a single object.  CLOS \cite{CLtL} and Cecil
\cite{Cecil} are two languages that have multiple dispatch; MultiJava
\cite{MultiJava} extends Java with multiple dispatch.  MultiJava
allows methods to dynamically dispatch on other arguments in addition
to the receiver; an argument can have both a static type and a dynamic
type (called its \defn{specializer}).  In MultiJava, we could rewrite
the above example thusly:
\begin{verbatim}
class P {
  void f(P p) { System.out.println("PP"); }
  void f(P@C c) { System.out.println("PC"); }
}
class C extends P {
  void f(P p) { System.out.println("CP"); }
  void f(P@C c) { System.out.println("CC"); }
}
\end{verbatim}
Then both \code{p.f(c)} and \code{p.f(c2)} would print \code{PC}, and
both \code{c.f(c)} and \code{c.f(c2)} would print \code{CC}.

%AspectJ makes the distinction between message send and method
%invocation: the pointcut \code{r(): receptions(void f(P))} includes join
%points when the \code{f} message is received (with static argument
%type \code{P}), while \code(e(): executions(void f(P))} includes join
%points when a particular method is selected and executed.  The
%difference can be seen in a super-send: if the method \code{C.f(P p)}
%were to call \code{super.f(p)}, no message send happens and no method
%selection is done; the method \code(P.f(P p)} in the superclass is
%invoked directly.  The join point is a member of \code{e()} but not
%\code{r()}.  

The main thing that multiple dispatch adds that is important for an
AOP language is more kinds of method signatures that can be mentioned in
pointcut designators.  In the above example, \code{executions(f(P))}
would include all four methods, but there should be a way to include
just the methods that specialize on \code{C}, such as
\code{executions(f(@C))}.  An alternate approach would be to
generalize \code{instanceof(C)} to mean any join point where the
receiver or any of the arguments are of class \code{C}; this extends
the notion of ``currently-executing object'' to the set of objects that
the message is sent to.  Perhaps a better solution would be to add a
new pointcut specifier \code{specializers(C)}, which included
join-points where any of the arguments were of (dynamic) type
\code{C}, or \code{specializers(C1,...,Cn)} where the sequence
of argument types matches \code{C1,...,Cn}.

\section{Intercessory Reflection}

% Explain Java MOP vs. Smalltalk/CLOS MOP.  Discuss Smalltalk aspect
% weaver, outline AOP for tiny-clos.

A language with \defn{reflection} allows programmatic manipulation of
program elements as data.  In an object-oriented language, if
the reflective data is also object-oriented, it's known as a
\defn{meta-object protocol (MOP)}.  For example, in Java, classes,
methods, and fields are objects, of type \code{Class}, \code{Method},
and \code{Field}, respectively; \code{o.getClass()} returns object
\code{o}'s class, and \code{c.getMethods()} and \code{c.getFields()}
return arrays of class \code{c}'s methods and fields, respectively.
Also, \code{m.invoke()} invokes method object \code{m}, and
\code{f.get()} returns the value stored in field object \code{f}.
This is known as \defn{introspective} reflection.
Java also allows new classes to be created at run-time, with
\code{ClassLoader.defineClass()}, but classes can't be redefined, and 
methods and fields can't be added or changed.  Moreover, the Java
programmer cannot create a new \emph{kind} of class, a subclass of the
\code{Class} class, in order to override methods of the default class
behavior.  This ability is known as \defn{intercessory} reflection.
Smalltalk \cite{Smalltalk} and CLOS \cite{AMOP} have intercessory
MOPs, and OpenJava \cite{OpenJava} extends Java with an intercessory
MOP, although OpenJava's MOP is only compile-time, making it more like
a macro-expander than full-fledged reflection.

The main impact of an intercessory MOP on an aspect language is the
ability to modify classes at run-time.  If a new method is added to a
class that matches a pointcut defined in an aspect that's ``in play'', 
should the advice be hooked into that new method?  Or is aspect
attachment a one-time transformation?  AspectJ already has to answer
a related question: since it only generates code at compile-time, it
can't anticipate new classes being defined at run-time.  B\"ollert
\cite{AOP-ST} designed an aspect weaver for Smalltalk that 
replaces woven classes with generated subclasses, but doesn't detect
modifications to already-woven classes.  While it's possible to do
this sort of on-demand weaving with a generative approach, an
alternate way of implementing AOP in a language with an intercessory
MOP is to make new kinds of classes and methods that call the
appropriate advice at each join point.  I've developed some
meta-objects in Tiny CLOS \cite{tiny-clos} to do this.

Tiny CLOS has multiple dispatch in the form of \defn{generic
functions}.  Methods are attached to generic functions, rather than
being attached to classes.  For example, the methods named \code{f} in 
the previous section would all be methods attached to the generic
function \code{f} and take two arguments rather than one:
\begin{verbatim}
(define-generic f)
(define-method (f (this <p>) (p <p>)) (display "PP"))
(define-method (f (this <p>) (c <c>)) (display "PC"))
(define-method (f (this <c>) (p <p>)) (display "CP"))
(define-method (f (this <c>) (c <c>)) (display "CC"))
\end{verbatim}
I define a subclass of the class \code{<generic>} called
\code{<aspectizable-generic>}; when an instance of it is applied to a
list of arguments, all advice that is applicable is run before, after,
or (in the case of \code{around} advice) instead of the corresponding
points in the method selection and execution algorithm (with the
continuation being passed to \code{around} advice).

In order to determine which advice is applicable, each piece of
advice's pointcut needs to be compared to the current join point.
Rather than try to duplicate AspectJ's pattern-matching style of
pointcut specifiers, however, I decided to generalize pointcuts into
arbitary boolean functions.  To wit, a pointcut is a generic function
that takes an aspect instance and a join point object (a descendant of
the \code{<join-point>} class, such as \code{<execution-join-point>}) and
returns true or false saying whether or not the pointcut contains the
join point.  (The reason it takes an aspect instance is so that
pointcuts can be overridden by aspect subclasses; this has the side
effect that the data stored as part of an aspect can play a part in
the pointcut determination.  I haven't yet come up with an example where
a pointcut was relative to an aspect instance, but it seems
potentially useful.)  For example, the \code{classes()} and
\code{methods()} pointcuts from the Trace example given earlier could
be written like so:
\begin{verbatim}
(define-generic classes 'generic-class <pointcut>)
(define-generic methods 'generic-class <pointcut>)
(define-method (methods (trace <trace>) (jp <join-point>)) #f)
(define-method (methods (trace <trace>) (jp <execution-join-point>))
  (classes trace jp))
(define-method (classes (trace <trace-my-classes>) (jp <join-point>))
  (memq (class-name (car (method-specializers (slot-ref jp 'method))))
        '(<foo> <bar> <baz> <garply>)))
\end{verbatim}
In other words, \code{(methods trace jp)} returns true if and only if
\code{trace} is an instance of the \code{<trace-my-classes>} aspect,
\code{jp} is an execution join point, and the name of the leftmost
specializer of the method being executed is one of the classes being
traced.  This simulates the \code{instanceof} pointcut designator in
AspectJ, by matching the ``receiver'' of the method; rather than
checking for class equality, however, only the names are compared,
which has the advantage of allowing this aspect to be defined
\emph{before} the classes to be traced are defined.  Also, the wildcard
matching of AspectJ could be achieved by comparing the name of the
class (or the generic function) to a regular expression.

Once it is determined that a piece of advice is applicable to a join
point, the advice is invoked.  A piece of advice is, like a pointcut,
a generic function; it takes an aspect instance and a join point (and
a continuation, in the case of \code{around} advice) and executes the
code in the advice.  For example, the \code{before} advice from the
tracing example could be written as:
\begin{verbatim}
(define-generic before-methods 'generic-class <before-advice>)
(define-method (before-methods (trace <trace>) (jp <join-point>))
  (if (slot-ref trace 'trace?)
      (trace-entry trace (get-signature jp))))
(slot-set! before-methods 'pointcut methods)
\end{verbatim}
Since a piece of advice is a generic function, it could be overridden
by sub-aspects, which isn't possible in AspectJ.

\section{Object-basedness}

% Discuss Self, Cecil, _A Theory of Objects_, BOB.

Languages such as Self \cite{Self}, Cecil \cite{Cecil}, and Obliq
\cite{Obliq} are object-oriented, but do not have classes; they are
known as \defn{object-based} languages, or sometimes
\defn{prototype-based}, although that refers to a particular
programming idiom that is not inherent in the language itself.
Objects inherit directly from other objects, and in Self and Obliq,
methods and fields are defined directly on objects themselves (in
Cecil, methods and fields are defined on generic functions).  When a
message is sent to an object, if there is not a matching method
defined on that object, the message is passed on to its parent
object.  Methods can be shared by multiple objects, by having the 
objects inherit from a parent object that holds the common methods.
The \emph{parent-of} relation thus serves the purpose of both
the \emph{class-of} and \emph{superclass-of} relations in class-based
languages.

Unifying these relations makes for a simpler language, without losing
expressiveness: classes can easily be simulated in an object-based
language.  (Note that the reverse is less easy: to simulate an
object-based language in a class-based language, every class would
essentially need to be a singleton class, and object creation would
involve class creation, which is typically expensive in a class-based
language---if it can even be done at run-time at all.)  An
object-based aspect-oriented language would similarly be simpler but
just as expressive.  For example, there would be no distinction
between aspect instances and sub-aspects.  Also, pointcuts and advice
would not need to be marked as static or non-static; static advice
just doesn't rely on fields defined on an aspect having different
values in the aspect's children objects.  The \code{instanceof(C)}
pointcut would just mean all join points where the receiver object has
\code{C} as an ancestor.

\section{Other language features}\label{other-features}

% first-class functions, dynamic typing, parametric polymorphism
% predicate dispatch

The three language features I have described above are the first three
that I have considered, but there are others worth investigating as
well.  Two features that are already inherent in my Tiny CLOS
implementation are \defn{first-class functions} (i.e., functions may
be assigned to variables, passed as arguments to other functions, and
returned from functions) and \defn{dynamic typing} (i.e., functions
and variables are not annotated with their type).  It remains to be
determined exactly what impact these features has on an
aspect-oriented language.  One side effect of first-class functions is
that functions (and classes) do not have an inherent name; they may be
bound to different names in different environments.  This might make
join point specification harder, because it would have to be relative
to a particular environment.  Dynamic typing also makes join point
specification harder, because there is less information about a
function that can be used to differentiate it from others.

Another feature that would be interesting to consider is
\defn{predicate dispatching} \cite{predicate-dispatch}.  It
generalizes multiple dispatch even further; rather than deciding what
method to invoke based on the dynamic types of the arguments, a
generic function can use any arbitrary predicate on the arguments to
choose.  For example, a generic function \code{pop} that takes a stack
argument could choose to run different methods depending on whether
the stack is empty or not.  This fits in well with the notion that
pointcuts are predicates; advice could just be methods on a generic
function whose predicates are the pointcuts.

Other features worth looking at are \defn{parametric polymorphism},
\defn{multiple inheritance}, and \defn{modules}.

\section{Related work}

% subject-oriented programming, composition filters, both have
% Smalltalk implementations.

B\"ollert \cite{AOP-ST} designed a general-purpose aspect
weaver for Smalltalk.  It can incrementally weave aspects into
classes, and remove them, at run-time.  In order not to interfere with
Smalltalk's development tools, which allow reflective access to the
source code at run-time, source-to-source weaving is avoided; instead
it uses Smalltalk's intercessory reflection to replace each woven
class with a new class and to replace a woven class's metaclass with a
new metaclass (to make sure that new instances are instances of the
new class).  It would be interesting to reproduce this approach in
Tiny CLOS or OpenJava, and to compare with my even-less-invasive
approach which does not involve replacing classes, but only adds a
small (fixed) number of new metaclasses.

Several areas of research have similar aims to aspect-oriented
programming, namely separation of concerns that are not easily
separated using procedural or object-oriented decomposition.  Some of
them, in particular \defn{subject-oriented programming (SOP)} \cite{SOP} and
\defn{composition filters} \cite{comp-filt}, have implementations in
Smalltalk as well as Java and C++.  SOP allows multiple class
definitions with the same name to be automatically merged into one
class; a \defn{subject} is a collection of partial class definitions
implementing a common concern, similar to an aspect.  SOP has mostly
been superceded by \defn{hyperspaces} \cite{MDSOC}, which is a more
general formulation of SOP allowing remodularization in multiple
concern dimensions simultaneously.  Currently the only hyperspaces
implementation is HyperJ, which is a set of tools for Java.
Composition filters allow message sends to be wrapped with multiple
layers of filters, which are similar to aspects.  Sina/st \cite{Sina} is
a language embedded in Smalltalk that implements composition filters
using reification of messages.

\section{Contributions}

The main contributions of my thesis research will be in an improved
understanding of aspect-oriented programming untethered to any
particular language.  So far, specific contributions I have already
tentatively identified include the idea of pointcuts being predicate
functions; the ability to use aspect instance data in pointcuts; and
the ability to override or extend advice in sub-aspects.  These
enhancements could all theoretically be applied back to AspectJ, but
they arose by looking at AOP in the context of languages other than
Java.

Specific implementation techniques are likely to be contributions of
the research as well.  In particular, the use of intercessory
reflection to implement aspect attachment is a flexible but clean
solution; the source-to-source translation approach used by AspectJ
can be seen as an optimization of intercession, but studying a more
dynamic form of intercession may lead to other ideas for optimizing
while still keeping the flexibility (allowing for things like
attaching and detaching aspects at run-time).

\section{Research Plan}

My research plan is to consider at least the three features detailed
above, plus probably three or four features from
section~\ref{other-features}.  Each feature should be considered in
isolation or as close to it as possible, by considering a language
only differing from the Java model with respect to that one feature.
I've already gotten somewhat ahead of my plan by developing AOP for
Tiny CLOS, which combines several features (namely multiple dispatch, 
intercession, dynamic typing, and higher-order functions).  So the
first step would be to develop an object model in Scheme that is
similar to Java, and then sart modifying that in different directions, 
one feature at a time, considering AOP for each.

Where feasible, I would also like to migrate some of my AOP
implementation ideas back to Java, or to extensions of Java such as
MultiJava \cite{MultiJava}, OpenJava \cite{OpenJava}, GJ (Generic
Java---Java with parametric polymorphism) \cite{GJ}, etc., by
modifying the AspectJ implementation (which is open-source).

In order to demonstrate the benefit of using AOP with each of these
language features, I need to provide some compelling examples of
programs that become easier to express using AOP.  To begin with, I
plan to use some of the examples from the AspectJ primer; I have
already started with the Trace example, and the Telecom simulation
seems like another good example.  However, I would also like to devise 
my own examples; one in particular that I've thought about is that of
multi-user resource control, where each user's resource allocation (such
as amount of memory, number of threads, or computation time) is
limited by one or more resource policies, encoded as an aspect.  My
working model for this example is the JRes toolkit \cite{JRes}, which
uses a specialized Java class loader to rewrite bytecode class
definitions in order to insert appropriate checks, in a somewhat
aspect-oriented manner.  I think this provides a good opportunity to
show the benefits of AOP by using an aspect weaver rather than a class 
loader.

\bibliography{thesis}
\bibliographystyle{alpha}

\end{document}
