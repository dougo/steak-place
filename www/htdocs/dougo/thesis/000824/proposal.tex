\documentclass{article}
%\documentclass[openbib]{article}
\usepackage{html}

% Grammar environment

\newenvironment{grammar}{
  \renewcommand{\:}{\goesto{}}
  \renewcommand{\|}{$\vert$}
  \tt \frenchspacing
  \begin{tabbing}
    %\qquad\quad \= 
    \qquad \= $\vert$ \= \kill
  }{\unskip\end{tabbing}}

% Commands for grammars
\newcommand{\arbno}[1]{#1\hbox{\rm*}}  
\newcommand{\atleastone}[1]{#1\hbox{$^+$}}

\newcommand{\goesto}{$\longrightarrow$}

% non-terminal in a grammar
\newcommand{\nt}[1]{{\noindent\hbox{\rm$\langle$#1$\rangle$}}}


% Reduce margins by 1/2 inch.
\addtolength{\voffset}{-0.5in}
\addtolength{\textheight}{1in}
\addtolength{\hoffset}{-0.5in}
\addtolength{\textwidth}{1in}

\title{Ph.D. Thesis Proposal: \\
A Next-Generation Collaborative Programming Language}
\author{Doug Orleans}

\begin{document}

\begin{htmlonly}
This document is also available in
\htmladdnormallink{postscript}{../proposal.ps} form.
\end{htmlonly}

\maketitle

\tableofchildlinks

\section{Introduction}

Computer programming is seldom a solitary effort---many projects are
just too big for one person to implement, and even smaller projects
can often be done faster with multiple developers.  The usual approach
to multi-developer programming is to use collaborative tools such as a
version control system to facilitate the exchange of source code.  An
alternate approach, however, is to use a programming language that
itself supports the concept of multiple programmers collaborating on a
project.  Several such languages exist, such as MOO\cite{MOO},
ColdC\cite{ColdC}, and LPC\cite{LPC-basic}\cite{LPC-interm}; however,
since they were written as descendants of MUD\cite{MUD}, a multi-user text
adventure game (and now referred to generically as \emph{muds}), they
were designed by hobbyists who had little background in programming
language design or theory, and thus are somewhat clumsy and ill-suited
for applications outside the domain of muds (and in some ways are even
inadequate for the mud domain itself).  Many ambitious projects have
been successfully implemented using them, however, even outside the
domain of muds, demonstrating that the idea of collaborative
programming languages is a useful one.  Little substantial progress
has been made in this area over the last seven years; this indicates
that some plateau had been reached, where the languages had matured to
an expressiveness that was suitable enough for widespread use.  I
believe, however, that there are still significant improvements to be
made in the design of collaborative programming languages.  For my
dissertation I will explore some of these possibilities, designing and
implementing a prototype language that is suitable for fleshing out
into a next-generation collaborative programming language.  Along the
way I will attempt to bring some academic rigor to the area, surveying
and comparing existing languages and developing semantics for
``featherweight'' subsets of them (in the manner of FJ\cite{FJ}) as a basis
for further study.  The rest of this paper presents a feature overview
of the most prominent collaborative programming languages, an
evaluation of their advantages and shortcomings, a description of the
basic prototype design for a new collaborative programming language,
and finally an outline of my plan for completing the dissertation.
Included as an appendix is a presentation of the formal syntax and
semantics for FM (Featherweight MOO).

\section{Feature overview}

In order to design a next-generation collaborative programming
language, I am studying current collaborative languages to see what
features I need to include and what needs to be improved.  What follows
is an overview of the basic features of some of the more widely-used
languages; my dissertation will include a more in-depth survey of the
features of these and other languages.
\subsection{Common features}
\label{features}
Before getting into the details of a particular collaborative
programming language, I will start with an overview of what the
languages I am studying have in common.  First of all, by definition,
they allow multiple programmers to collaborate.  In practice, this
means that they act as servers, accepting input from multiple network
sockets asynchronously, and allow code (i.e.~behavior) to be added or
modified dynamically based on this input.  (A digression: the features
I discuss may appear to be features of a programming language
\emph{environment}, rather than of a programming language; however, the
line between the two is often blurred, even in more mainstream
languages--- the interactive environment is fairly integral to the
Smalltalk\cite{Smalltalk} and Lisp\cite{Lisp} families of languages,
and the Java language specification\cite{JLS} covers the class loading
process in detail.  More importantly, however, a collaborative
programming language typically
\emph{reifies} the programmer as a structure that is
part of the semantics of the language, so features that usually are
part of the environment really \emph{are} part of the language.)  This
feature, asynchronous dynamic behavior modification, requires a number
of other features: \emph{type safety} and \emph{garbage collection},
so that a wild pointer or memory leak caused by one programmer can't
cause the whole server to crash; \emph{multitasking} and
\emph{resource control}, so that one programmer's code can't
monopolize the server by, for instance, running a tight infinite loop;
\emph{persistence}, so that the programmers don't have to re-upload all their
code when the server is shutdown and restarted; and \emph{reflection},
so that programmers can examine the code that's already on the server
(their own or others').

Another feature that collaborative programming languages usually have
is \emph{security}: a programmer owns code and data that he creates, and can
control access to it; however, while this can be useful for enforcing
abstraction barriers (much like access control in other languages,
e.g. the \texttt{private}, \texttt{protected}, and \texttt{public}
keywords in C++\cite{ARM} and Java), it is not strictly necessary for
collaboration, and is often not used at all.  Collaborative
programming languages are often (perhaps predominantly) used in
another setting than collaboration, though, which I will call
\emph{community programming}: programmers share code and data dynamically,
yet they may not always be collaborating in the sense of working
towards a common goal, and in fact may not trust each other at all; in
this situation, access control mechanisms are essential.

The collaborative programming languages I am studying share another
feature that is not essential to collaboration: they are all
single-dispatch classless object-oriented languages.  That is, they
have objects, which have fields and methods, and which may inherit
fields and methods from other objects.  These languages are probably
object-oriented because they were designed for the multi-user
simulation environment domain, and OO languages are well suited for
simulations; the fact that they are all also classless may be
coincidence, but it may also be due to the dynamicity properties of
the language: it's easier to modify a field or method on an object,
which is dynamically inherited by the object's descendant objects,
than it is to modify a field or method on a class and update all the
instances of the class.  Interestingly, recent versions of several of
these languages have added mechanisms that can simulate the
class-instance relation.

\subsection{MOO}

The MOO programming language is idiosyncratic in that it is a mud
language, and its terminology reflects that fact: users are called
\emph{players}, super-users are called \emph{wizards}, object
methods are called \emph{verbs} (because they are used to execute
the verbs in commands, such as \texttt{{\ttfamily\bfseries look} in
mirror}, \texttt{\textbf{get} key}, or \texttt{\textbf{put} book on
shelf}), and object fields are called \emph{properties}.  There are
also a number of aspects to the language that are irrelevant to a
general-purpose collaborative programming language: for example, all
objects have \texttt{name}, \texttt{location}, and \texttt{contents}
properties, and all user input is processed by default by a command
parser that tries to match words in the command to objects in the same
location as the player.  In general I will ignore mud-specific aspects
of the language in this paper, but occasionally they have an impact on
other parts of the language.

All program code in MOO lives in the verbs; there are no global
functions or static code blocks.  The MOO server has a built-in
command for programming a verb that accepts the code of the verb
terminated by a single period on a line by itself.  Typically
programmers can also evaluate single expressions or sequences of
statements with an \texttt{eval} command.  The syntax is somewhat
similar to C++ or Java, with semicolon-terminated statements and infix
arithmetic; object properties are referenced with a dot,
e.g.~\texttt{player.name}, while verbs are invoked with a colon,
e.g.~\texttt{player:name()}.  The server provides a set of
\emph{builtin functions} (analogous to system calls) which are
invoked as if they were global functions,
e.g.~\texttt{notify(}\textit{p}\texttt{,~"Hello,~world!")} sends the
string ``Hello,~world!'' to player \textit{p}'s connection.  Variables
are untyped and do not need to be declared ahead of time; using a
variable before it has been initialized results in a ``Variable not
found'' exception.  There are no formal named arguments to a verb;
instead a single special variable \texttt{args} holds the list of
actual arguments used in the verb invocation.  Scattering assignment
can be used (as in Perl) to assign the arguments to a list of
variables, e.g.~\texttt{\{a,~b,~c\}~=~args}.  Other special variables
include \texttt{this}, which is the receiver object of the
current verb invocation, just like in C++ or Java 
(\texttt{this} must be explicitly used when referring to properties
and verbs on the current object);
\texttt{caller}, which is what the value of \texttt{this} was when the
current verb was invoked; and \texttt{player}, which by default is the
player who entered the command currently executing (but the value of
\texttt{player} can be changed by a verb running with wizard
permissions---see the paragraph on security below).

As mentioned in the overview, MOO is a single-dispatch classless
object-oriented (sometimes referred to as object-based) language.
Objects inherit from each other, singly; thus, each object has a
parent, and when a verb or property is referenced on an object, if
it's not defined on that object, the object's parent is searched.  A
verb can call the builtin function \texttt{pass()} to re-send the
current verb call to the object's parent.  Properties use
copy-on-write inheritance; when a property is changed on a child
object it is copied to the child and then changed.  Each object has a
unique fixed integer associated with it, known as its \emph{object
number}, or \emph{objnum}; all objects can be referred to in code with
a literal objnum expression \texttt{\#}\textit{objnum}.  Objnums are
assigned at object creation time in increasing order starting with
\texttt{\#0} for the system object.  A shorthand exists for referring
to properties and verbs on the system object: \texttt{\$foo} is
equivalent to \texttt{\#0.foo}, and \texttt{\$bar(x,~y)} is equivalent
to \texttt{\#0:bar(x,~y)}.  This allows for a global namespace of
sorts; a useful object can be given a name by adding a property to
\texttt{\#0} with the object as its value, so that, e.g.,
\texttt{\$player} can be used to refer to the generic player object
that is used as the parent of all player objects.

Objects are not garbage-collected; objects must be deleted explicitly.
Objnums referring to deleted objects (as well as negative objnums) are
referred to as \emph{invalid}.  Deletion of objects leaves ``holes''
at these invalid objnums; there is a \texttt{renumber()} builtin
function that reassigns an object's number to be the lowest
nonnegative invalid objnum, but this is not used in practice (except
in special cases) because it does not fix existing references to the
object's old number (in properties or in verb code).  Instead, the
custom is to never actually delete an object, but to strip all verbs
and properties from it and put it onto a free list of ``garbage''
objects that can be reused when new objects are needed.

There are other, non-object values in MOO, such as integers, strings,
and lists; list values are immutable heterogenous arrays and are
garbage-collected, using a simple reference-counting scheme.  Lists
are often used for data structures, because objects are heavyweight,
both conceptually and in time and space overhead, due to the cost of
recycling and the allocation of builtin properties such as name and
location that are only useful for objects representing virtual
entities.  A recent experimental extension to the language adds a new
value called \emph{waifs}; this extension in essence treats normal
MOO objects as classes, with waifs acting as instance objects.  Each
waif value has a class (an objnum) and a list of values corresponding
to its class's list of defined properties.  Verbs are invoked on waifs
just as they are on objects, by dynamically looking up the class's
inheritance chain; waifs cannot define their own verbs or properties.
Waifs, like lists, are garbage-collected, but are mutable, like
objects, providing the best of both worlds, although the current
implementation imposes the limitation that circular waif structures
cannot be created, so that reference-counting garbage collection can
still work.

MOO is single-threaded, but has multitasking; each command received
from a player starts a task, and additional tasks may be created with a
\texttt{fork...endfork} statement.  Task switching is done
cooperatively: a task may surrender control to other tasks for a given
minimum number of seconds \textit{n} (which may be~0) by calling the
builtin function \texttt{suspend(}\textit{n}\texttt{)}; a suspended task may be
resumed before its time by calling the builtin function
\texttt{resume(}\textit{id}\texttt{)}, where \textit{id} is the task-id of the
suspended task (a unique integer).  The
\texttt{read(}\textit{p}\texttt{)} builtin function also
suspends the current task, resuming when a line of input has been read
from player \textit{p}'s connection.  In order to prevent a task from
monopolizing the server by not suspending, however, the server
interrupts a task if it has run for more than a certain amount of time
or number of instructions (\emph{ticks}); the default threshold
amounts are 5 seconds and 30,000 ticks (for command tasks), which can
be overridden by setting special properties on the system object
(\texttt{\#0}).  When a task is interrupted by the server, an
uncatchable exception is thrown and printed to the player who typed
the command that created the task (this behavior can also be
overridden by adding a special verb to the system object).  Usually, a
task that has a loop that might run a large number of times will call
the builtin function
\texttt{ticks\_left()} each time through the loop to determine whether
it needs to suspend.

In addition to time restrictions on a task, the amount of space used
by a player can be restricted; if a property named
\texttt{ownership\_quota} exists on a player, then when that player
calls the builtin function \texttt{create()} to make a new object, the
value of the property is decremented by one if it's positive, or else
a ``Resource limit exceeded'' exception is thrown.  The value is
likewise incremented whenever an object is destroyed with the
\texttt{recycle()} builtin function.  (If a ``garbage'' list is used
as described above, this value can also be adjusted whenever objects
are added to or removed from the list.)  The builtin function
\texttt{object\_bytes()} can also be used to measure the size of an
object, if a more finer-grained quota policy needs to be implemented.
There are also restrictions on the number of nested verb calls in a
task stack and the number of forked or suspended tasks owned by any
one player.

The heap of allocated objects is known as the \emph{database}, or
\emph{db}, although the db does not have much resemblance to a
conventional database; the db is kept entirely in memory in a single
process, and must be periodically checkpointed by calling the
\texttt{dump\_database()} builtin function, which saves
the db to disk in a format that can be reloaded if the system needs to
be restarted after being shut down (or crashing).  A version of the
MOO server called LPMOO\cite{LPMOO} is implemented on top of another mud server,
DGD\cite{DGD} (in its language, LPC), which itself provides more continuous
persistence, as well as smaller process size, by keeping all objects
on disk and loading them into memory only when needed; only the
working set of cached objects need be written to disk when
checkpointing, which can then be done more often since it's much
faster.

There are a number of builtin functions that provide reflective
capabilities.  For introspection, there are builtin functions to
determine an object's parent, its children, its defined properties or
verbs, or whether an object is a player object; there are builtin
functions to inspect a property or verb definition's name, owner, and
permission flags (see next paragraph), or about a builtin function, or
to retrieve a verb's code; and there are builtin functions to get the
list of player objects, the maximum object number, or the list of
currently queued tasks.  For invocation, there are builtin functions
to change an object's parent, to make an object a player or not a
player, to add or delete a property or verb from an object, to set a
property or verb definition's name, owner, or permission flags, or to
set the code of a verb; there are also builtin functions to call a
builtin function given its name in a string, or to evaluate an
arbitrary expression given in a string.  In addition, properties can
be referenced using a computed name with special syntax,
e.g.~\texttt{obj.(propname)} will read the property whose name is the
value of \texttt{propname} (which must be a string) on the object
\texttt{obj}; verbs can be invoked with a similar syntax,
e.g.~\texttt{obj:(verbname)()} will invoke the verb whose name is the
value of \texttt{verbname} on the object \texttt{obj}.  There is a
limited form of intercession, as well: whenever a builtin function
\textit{fun} is called, if there is a verb on the system object whose
name is \texttt{bf\_}\textit{fun}, it is invoked instead.  There is no
way to intercede on a property reference or a verb invocation,
however.

Security in MOO is accomplished with a simplified Unix-style
permission system.  Tasks always run with the permissions of a single
player, similar to the effective user id of a process in Unix.  Every
object, verb, and property has an owner and \texttt{r} and \texttt{w}
flags that determine if tasks running with permissions other than the
owner (or a wizard) can read or write them.  Each object also has an
\texttt{f} flag (``fertile'') that determines if tasks running with
permissions other than the owner may create children of the object.
Property ownership and permissions are inherited with copy-on-write
similar to property values: if the owner or the permission flags of a
property are changed on a descendant of the object that defines the
property, the values are copied to the descendant and then changed.
In addition, each property has a \texttt{c} flag, which determines
whether the ownership is changed on child objects to be the same as
the owner of the child object; this allows a user to create a child of
someone else's object, and still be able to modify the properties of
the child object that are defined on the parent.  If this flag is not
set, then the copy of the property on the child remains owned by the
owner of the parent property; this allows verbs defined on the parent
object, which are usually owned by the parent object owner, to modify
the properties on the object.  Tasks always run with the permission of
the owner of the current verb, rather than the permissions in effect
when the verb is invoked, so that a verb owned by a player can always
access properties owned by that player as well.  This is the reverse
of the normal Unix case; it is as if all programs in Unix had their
set-user-ID bit on.  Tasks running with wizard permissions, i.e.~verbs
owned by wizards, can call
\texttt{set\_task\_perms(}\textit{p}\texttt{)} to change the current
task's permissions to player \textit{p}; this is often used to run
with the caller's permissions, by setting them to the value of the
builtin function \texttt{caller\_perms()}.  Tasks running with wizard
permissions may also change the value of the special variable
\texttt{player}, in order to simulate a command being sent by another player 
than the one who actually sent it.  A number of builtin functions are
restricted to act only on objects, verbs, or properties owned by the
current permissions, unless the current permissions are those of a
wizard, such as \texttt{chparent()} and \texttt{recycle()}; others
may only be with wizard permissions, such as \texttt{dump\_database()}
or \texttt{shutdown()}.

\subsection{ColdC/Genesis}

ColdC (originally called C$--$) is a successor to MOO, and so shares
many of MOO's features, improving on some of them.  The main
innovation was the removal of all mud-specific features, creating a
true general-purpose collaborative programming language.  The ColdC
server (or \emph{driver}), Genesis (originally called Coldmud), does
not attempt to parse commands, nor does it maintain location and
contents; even security features have been almost completely removed.
The intention is that all of these features can (if desired) be
implemented on top of the base functionality that the language
provides.  Terminology is more in line with traditional OO languages:
users are called users and methods are called methods; fields are
called \emph{object variables}.

ColdC syntax is more like C++ than MOO is: both object variable and
method reference is done with dot (\texttt{foo.bar} or
\texttt{foo.bar()}), and object variables on the current object may be
referred to by simply naming them (i.e. no explicit \texttt{this}
needed); local variables must be declared at the top of a method
(though there is still no static typing); method parameters are named;
block structure can be provided with curly braces.  Invoking a method
on the current object can be written as \texttt{.foo()}; the dot is
still necessary to avoid ambiguity with functions (the ColdC term for
what MOO calls builtin functions).  There are no special variables;
instead, functions are used: \texttt{this()} returns the current receiver
object, \texttt{definer()} returns the object that defines the current method
(which is always an ancestor of \texttt{this()}), \texttt{sender()}
returns the object that invoked the current method, and
\texttt{caller()} returns the object that defines the method that
invoked the current method (an ancestor of \texttt{sender()}).  The
function \texttt{user()} simply returns whatever was passed to the
last call to \texttt{set\_user()} on the current object; the intention 
is similar to the \texttt{player} special variable in MOO, i.e.~to
keep track of the originating object for the current task, but the
driver does not do anything special to set or maintain it.

The ColdC object model has multiple inheritance; method inheritance
uses depth-first search up the parents list, from left to right,
skipping parents it has already searched due to inheriting
the same parent via multiple paths.  The \texttt{pass()} function
re-sends to the next method that would be found in the current method
invocation, i.e.~it might not always be an ancestor of
\texttt{definer()}.  Objects have unique numbers, as in MOO, which are 
referred to with the syntax \texttt{\#}\textit{objnum}; they also have
unique alphanumeric names, which are referred to with the syntax
\texttt{\$}\textit{name}.  Object names can be changed with the
\texttt{set\_objname()} function, but this should be done with care as 
methods with literal references to the old name will not be
updated.\footnote{I don't know why they aren't just stored as object
numbers!}

Similar to MOO, objects in ColdC are not garbage collected, but there
is a lightweight object data structure similar to the waifs extension
to MOO, called a \emph{frob}.  Unlike waifs, however, frobs are
immutable, which somewhat limits their utility.

ColdC has cooperative multitasking, similar to MOO.  Atomic mode can
be turned on with the function \texttt{atomic()}, which prevents any
other tasks from running until atomic mode is turned off; a task
running in atomic mode is also immune to the tick limit.  The creation
functions do not enforce any sort of quota policy, but one can be
implemented in ColdC using the \texttt{bind\_function()} function (see
the next paragraph below).  Genesis is disk-based, i.e.~it keeps all
objects on disk and only loads them into memory when needed; however,
tasks are not persistent and must be manually saved in a restartable
format if they need to stay around after the system is restarted.
Reflection in ColdC is similar to MOO, but with no intercessory
capabilities and no \texttt{eval()} function.

ColdC has no notion of task permissions such as MOO has; however, it
allows C++/Java style access control for methods, using the keywords
\texttt{public}, \texttt{protected}, and \texttt{private}, and it is a 
runtime error to call protected or private methods from outside the
object.  There are three additional access control keywords:
\texttt{root}, which means a method may only be called by the
\texttt{\$root} object; \texttt{driver}, which means a method may only
be called by the driver; and \texttt{frob}, which means a method may
only be called with a frob as the receiver.  All object variables are
considered to be private, i.e.~only the defining object may access its 
object variables.  Functions (e.g.~administrative functions like
\texttt{shutdown()}) can be
protected by using the \texttt{bind\_function(}\textit{fun}\texttt{,
}\textit{obj}\texttt{)} function to only allow the \textit{fun}
function to be called from methods defined on \textit{obj} (typically
\texttt{\$sys} or \texttt{\$root}).  This can be used to force access
to functions to go through methods, which can impose extra security or
resource control measures.  A method may also be set
\texttt{nooverride}, which means that no descendant object may define a 
method with the same name; this can prevent spoofing of secured
methods.

\subsection{LPC/DGD}

LPC is the language for LPMUD, a descendant of the original MUD that
evolved in parallel to MOO without much cross-breeding.  DGD
(Dworkin's Generic Driver) is the latest LPC driver, which, like
Genesis/ColdC, is a general-purpose collaborative programming language
because it doesn't include any mud-specific features.  Its syntax is
almost identical to C, including statically typed variables but not
including pointers (but objects, strings, arrays, etc. are passed by
reference, similar to Java).  The database consists of a Unix-style
hierarchical file system; each object corresponds to a file whose name
ends in \texttt{.c}, e.g.~\texttt{/system/driver.c}, which contains
the source code for the object variables and functions.  A literal
object reference is simply a string containing the source filename of
the object (without the \texttt{.c}).  Calls to functions in other
objects are done C++-style, i.e.~\texttt{obj->fun()}.  Driver
functions are called \emph{kfuns} (kernel functions) and are
called the same way as functions in ColdC or MOO.  The kfuns
\texttt{this\_object()}, \texttt{previous\_object()}, and
\texttt{previous\_program()} play the roles of ColdC's
\texttt{this()}, \texttt{sender()}, and \texttt{caller()},
respectively.  An object can (multiply) inherit from other objects
using \texttt{inherit }\textit{obj}\texttt{;} declarations.  A
function can re-send to a function on its parent using
\texttt{::}\textit{fun}\texttt{()}.  Directed re-send is accomplished
by providing a tag in the \texttt{inherit} declaration and using the
tag, e.g.~the declaration \texttt{inherit p2 "/foo/bar";} allows
re-sends to the object \texttt{"/foo/bar"} with the expression
\texttt{p2::}\textit{fun}\texttt{()}.

LPC objects are not garbage collected, but arrays and other data
structures are.  Something similar to waifs and frobs exists, called
\emph{clones}: instead of inheriting an object by making a new
object file with an \texttt{inherit} declaration, you can call the
kfun \texttt{clone\_object()}, which makes a new copy of an object.  A
clone cannot itself be cloned or inherited from, but is otherwise like 
a regular object.  When an object is modified and reloaded (with the
kfun \texttt{compile\_object()}), all of its clones are modified as
well, keeping the same values for variables whose definition did not
change.  Clones are assigned names in the file system, based on the
name of the object it was cloned from and a unique number, e.g.~a
clone of an object named \texttt{/foo/bar} will be named something
like \texttt{/foo/bar\#1234}.  Since clones are always accessible by
name with a literal object name expression, they are not garbage
collected.

LPC has cooperative multitasking, similar to MOO and ColdC.  The
resources (maximum stack depth and number of ticks) available to a
block of code can be specified with a
\texttt{rlimits(}\textit{depth}\texttt{, }\textit{ticks}\texttt{) \{}
\ldots \texttt{\}} construct.  DGD is disk-based, similar to Genesis.
Reflection is similar to MOO, although introspection is mainly done at 
the object level rather than the level of functions and object
variables.  (The exception is the kfun \texttt{function\_object()},
which determines which inherited object provides a given function.)
There is no \texttt{eval()} function.  Intercession is achieved through 
a number of hooks that are called on the driver object (similar to the
system object in MOO or ColdC); of particular note is
\texttt{call\_object()}, which is called whenever an object calls a
function on another object---neither MOO nor ColdC has intercession on 
method calls.  In addition, kfuns can be shadowed (and new ones added) 
in the root object, called the \emph{auto} object, similar to
\texttt{\#0:bf\_}\textit{fun} wrappers in MOO.

LPC's approach to security is similar to ColdC's: access to functions
and variables is controlled with keywords such as \texttt{private} and
\texttt{static} (similar to \texttt{protected} in other languages),
and all other access control must be implemented in the database
(with the intercessory driver hooks, as opposed to restricting access
to kfuns as is done in ColdC).  The keyword \texttt{nomask} serves the 
same purpose as ColdC's \texttt{nooverride}.

\section{Evaluation}
Now that I've surveyed a few existing collaborative programming
languages, I can begin identifying features that I want to keep and
features that I want to improve upon.  Of course, the general features
I identified as essential in section \ref{features} will need to go
into my language: dynamic behavior modification, type safety, garbage
collection, multitasking, resource control, persistence, and
reflection.  I will also need to at least provide a framework in which 
security capabilities can be implemented.

The classless object-oriented model seems to be a successful one, and
my own personal bias is towards a classless object model as well; as
Abadi and Cardelli observe in their book \emph{A Theory Of
Objects}\cite{theory-of-objects},
page~36:
\begin{quote}
The main insight of object-based languages is that class-based notions 
need not be assumed, but instead can be emulated by more primitive
notions.  Moreover, these more primitive notions can be combined in
more flexible ways than in a strict class discipline.
\end{quote}

One of the key barriers to this sort of flexible combination in MOO,
ColdC, and LPC is the lack of garbage collection for objects.  The
ability to access any object by number (in MOO) or name (in ColdC and
LPC) leads to this restriction, which in turn has led to ad-hoc
alternative mechanisms for lightweight objects (namely waifs, frobs,
and clones) that are clumsy and limited.  This can be solved by simply 
not allowing pointers to arbitrary objects to be created.

Dynamic behavior modification is somewhat clumsy in LPC: you can only
change code by recompiling an entire object, and if the object has
descendants, it and they must all be destructed and reloaded.  Some of 
this can be automated, but the granularity of updates is still too
coarse.  MOO and ColdC allow users to update individual methods and
fields on an object, and all descendants (and waifs and frobs)
automatically inherit the changes.

Cooperative multitasking of the sort implemented in MOO, ColdC, and
LPC has its benefits: you know exactly what part of your code is
atomic (namely, everything between calls to \texttt{suspend()}), so
you don't have to put semaphores on every public data field.  However,
there are two big disadvantages.  One is that every method you call
that you don't own might turn out to suspend when you didn't expect
it; this is usually solved by the convention that every utility
method that might suspend has \texttt{\_suspend} in its name.  However, 
the fatal problem with cooperative multitasking is the tick limit---it 
can be very tedious to always check to see how many ticks are left,
and if this check is left out, the code might fail with an uncatchable 
exception, which could cause all sorts of havoc with unfinished data
structures and the like.  In practice, it becomes almost as much work
to deal with the tick limit as it would be to deal with semaphores, and
as soon as you start suspending you need to worry about concurrency
anyway.  Preemptive multitasking is the only real solution.

Most of the built-in resource control features of MOO, ColdC, and LPC
deal with tick limits; with preemptive multitasking, this is
unnecessary, assuming the thread scheduler avoids starvation.  The
quota-based approach to memory consumption seems like the best model,
and powerful enough reflection capabilities allows this to be
implemented in a fairly straightforward manner, i.e.~by interposing a
quota check in calls to object creation primitives.

The MOO server handles persistence by periodic checkpoints of the
entire database; the Genesis and DGD drivers optimize the checkpoint
time by only keeping a cache of active objects in memory, at the
expense of the extra time needed to swap inactive objects into the
cache when they are referenced.  While this tradeoff may be desirable
for some purposes, I believe that MOO's model is sufficient for most
purposes: it's the job of the operating system to provide virtual
memory and take care of swapping to and from disk, and implementing an
additional swapping mechanism on top of this seems like diminishing
returns.  More sophisticated persistence mechanisms such as journaling
or compare-and-commit might be worth investigating, but I don't
consider it an important enough issue to spend time on it for my
thesis.  The issue of what to persist is another question; I think MOO
and LPC have the right idea, where everything (including tasks) is
saved to disk (ColdC doesn't save tasks).  In addition, LPC and ColdC
allow a user to mark fields that don't need to be saved; while this
can save some time and space in some instances, I don't see it as
being a crucial feature.

MOO, ColdC, and LPC all enable introspection and reflective invocation
in much the same way, with an ad hoc but fairly complete set of global
functions.  Intercession is handled similarly in MOO and LPC, by
allowing the global functions to be globally overridden; ColdC is
slightly more flexible, allowing functions to be bound to particular
objects which can then override the behavior.  In practice, though,
these schemes turn out to be virtually equivalent.  LPC has the added
advantage of providing kfuns that correspond to method calls, which
are only available as syntactic constructs in MOO and ColdC; this
gives users the important ability to intercede on every method call,
which allows a much greater range of customization of the
meta-behavior.  While the reflective global functions idea is more or
less sufficient, I think putting the reflective functions into a
meta-object protocol (where every object has a meta-object as well as
a parent) provides for easier and perhaps more disciplined reflection,
as you then gain all the advantages of object-oriented programming
when using reflection: encapsulation, inheritance, and polymorphism.

MOO's approach to security is fairly simple, but can be cumbersome to
use.  Programmers find themselves repeating the same code in verb
after verb: inspect \texttt{caller} and/or \texttt{caller\_perms()}
and raise an exception if they're not allowed to access the verb.
ColdC and LPC's \texttt{private} and
\texttt{protected}/\texttt{static} access keywords remove the need for 
most of this code, but conversely, neither of them allow access to
an object's fields from outside of the object, so the programmer
always has to write accessor methods, including code to check if the
caller has the right permissions.  LPC potentially allows this to be
abbreviated by interceding on method invocation, however, and I think
reflection (in particular, a powerful enough intercession mechanism)
is the key to making security easier to implement.

\section{A prototype next-generation collaborative language}

One approach to implementing a better collaborative language would be
to embed mechanisms for collaboration in an existing single-programmer
language.  Dynamically extensible languages such as Lisp or Smalltalk
and their descendants would be best suited for this.  This approach
gives all the benefits of a full-featured general purpose language
more or less for free; for example, you could write a simple program
in CLOS\cite{Lisp} that accepted connections on a socket and attached
a read-eval-print loop to each connection in a separate thread, along
with a small library of functions for publishing values to other
connections' environments, and you would have a collaborative
programming language with all the features described in section
\ref{features} except for resource control and security.
Unfortunately, these two are very difficult to integrate with an
existing language; for example, one can easily circumvent Common
Lisp's module encapsulation and reference an internal symbol in
another package by simply using two colons (i.e. \texttt{foo::bar}),
and CLOS maintains a global class namespace---most CLOS functions
allow the programmer to refer to a class by name (a symbol), rather
than by value.  One would have to write secure, multi-programmer-aware
versions of practically every function in the CLOS library in order to
be able to allow users to control access to their classes.

Java appears to be another good candidate for embedding collaborative
features into, and in fact the Safe Language Kernel (SLK)
project\cite{SLK} is working towards making Java into a collaborative
programming language (in fact, a community programming language---one
in which programmers might be sharing code and data in a single server
process but not necessarily collaborating towards the same goal).
Their J-Kernel\cite{J-Kernel} replaces the standard Java class loader
with one that rewrites the bytecode of classes as they are loaded and
interposes checks for security and (with JRes\cite{JRes}) resource
control in every method.  This approach seems to be workable, but
seems awkward to implement, somewhat fragile and error-prone, and also
inconvenient for the programmer, who has to reload a whole class (and
replace all the instances of the old class with the new class)
whenever a method or field is added or changed.  And with Java's
static typing, it seems like this could lead to a chain of
recompilations and updating of instances.  Perhaps if Java evolves to
be a little more dynamic, this approach will be more attractive, but
as it stand now it seems like a tough task to try to shoe-horn
collaborative features into the language.

An alternate approach is, of course, to implement a new language from
scratch.  This has the advantage of complete control over the
features; security and incremental update can be designed in from the
start.  It is also a good approach for pedagogical reasons: you can
define a very clean, simple prototype language without needing to deal 
with all the issues that would be required for a full-featured
practical language but are irrelevant to a collaborative programming
language.  However, even just implementing the features described in
section \ref{features} is quite a task!  Multitasking and persistence, 
in particular, can be very tricky to implement, and while they are
(practically speaking) required for a collaborative programming
language, their implementation has little to do with the
collaborative-ness.  (More importantly, I just don't have the
expertise needed to implement them in the time frame of my PhD!)

What I've decided on is a hybrid approach that gives me much of the
best of both worlds.  I have begun to implement a prototype language
embedded in Larceny, an implementation of Scheme created by Lars
Hansen, a fellow doctoral student at Northeastern.  In Scheme,
environments are truly encapsulated---values can only be accessed if
they are bound to a name in the local environment, or returned by a
procedure that the user already has access to---and can serve as the
basis of a simple security kernel, as Jonathan Rees has
outlined\cite{W7}.  The advantage of the Larceny implementation is
that it provides multithreading and persistence (in the form of heap
dumping and loading), as well as support for first-class environments.

\subsection{BOB}

So far, I've mainly concentrated on building an object system in
Larceny suitable for a collaborative programming language; I decided
to base it on the untyped subset of BeCecil\cite{BeCecil}, a
simplified (and formalized) version of Cecil\cite{Cecil}, hence the
(working) name ``BOB'', which stands for ``Based On BeCecil''.  The
BOB object system is, like BeCecil, centered around generic functions
(also known as multimethods) similar to those in CLOS and Dylan.  It
also provides a multiple-dispatch version of instance variables called
storage tables, elegantly unifying fields and methods into the same
lookup mechanism.  Unlike BeCecil, however, everything is dynamically
modifiable; methods and storage tables can be added to or deleted from
generic functions, and the object inheritance relation can be mutated,
all at runtime.  Figure~\ref{bob-syntax} summarizes the core syntax of
BOB.

\begin{figure}
\begin{grammar}
   \nt{definition} \: (define-object \nt{variable} \nt{parents})
\\  \> \| (define-gf \nt{variable})
\\ \nt{parents} \: \arbno{\nt{object}}
\\ \nt{object} \: \nt{expression}
\\
\\ \nt{expression} \: (add-method! \nt{gf} (\nt{spec formals}) \nt{body})
\\ \> \| (add-acceptor! \nt{gf} (\nt{spec formals}) \nt{rhs-variable} \nt{body})
\\ \> \| (add-storage! \nt{gf} (\nt{spec formals}) \nt{default})
\\ \> \| (\nt{gf} \arbno{\nt{actual arg}})
\\ \> \| (set! (\nt{gf} \arbno{\nt{actual arg}}) \nt{expression})
\\ \nt{gf} \: \nt{expression}
\\ \nt{spec formals} \:  \nt{variable} \| (\nt{variable} \nt{specializer})
\\ \nt{specializer} \: \nt{object}
\\ \nt{rhs-variable} \: \nt{variable}
\\ \nt{default} \: \nt{expression}
\\ \nt{actual arg} \: \nt{expression} \| (direct \nt{expression} \nt{parents})
\end{grammar}
\caption{The core syntax of BOB, as extensions to the Scheme syntax
grammar.}\label{bob-syntax}
\end{figure}

Objects are created with \texttt{define-object}, which makes a new
object with the given name and parent(s) and binds a name to the new
object in the current lexical scope.  If no parents are specified,
the new object inherits directly from \texttt{<object>}, the root of
the inheritance hierarchy.  Objects act as both classes and objects in 
class-based languages.  For example, we can define classes for points,
colored objects, and colored points with these definitions:

\begin{verbatim}
(define-object <point>)
(define-object <colored>)
(define-object <colored-point> <point> <colored>)
\end{verbatim}

Since there is no difference between the inheritance and instance-of
relationships, we can create an instance of \texttt{<point>} the same
way that we created a subclass:

\begin{verbatim}
(define-object my-point <point>)
\end{verbatim}

Generic functions are created with \texttt{define-gf}, which makes a
new generic function with the given name and binds the name to the new 
generic function in the current scope.  (Note that unlike CLOS, a
generic function does not have a fixed number of arguments; methods
with any number of arguments may be attached to a generic function.)
Methods are attached to a generic function with \texttt{add-method!},
which creates a method and adds it to the generic function by side-effect.
Each formal argument to a method has both a name and a specializer
object.  (A specializer object can be omitted if it is
\texttt{<object>}.)  The specializer objects determine which method is 
invoked when the generic function is applied; a method is applicable
only if all of the arguments are descendants of the corresponding
specializer objects.  For example, assuming generic
functions \texttt{x} and \texttt{y} have been defined to return the x
and y coordinates of a point, we can compare two points with these
methods:

\begin{verbatim}
(define-gf equal?)
(add-method! equal? ((p1 <point>) (p2 <point>))
  (and (equal? (x p1) (x p2)) (equal? (y p1) (y p2))))
(add-method! equal? ((i1 <number>) (i2 <number>))
  (= i1 i2))
\end{verbatim}

When the generic function \texttt{equal?} is applied to two point
objects, the first method is invoked, which in turn applies
\texttt{equals?} to the two pairs of coordinates; each of
these applications invokes the second method, which uses the Scheme
primitive function \texttt{=} to compare the two numbers.
Note that it would be an error to apply \texttt{equal?} to a point and 
a number; we could handle this case with a catch-all method that just
called the Scheme primitive \texttt{eq?}, which only returns true if
both arguments are the same:

\begin{verbatim}
(add-method! equal? (a b)
  (eq? a b))
\end{verbatim}

In this case, the method's formal specializers are both
\texttt{(<object>)}, so the method is applicable to all pairs of
arguments, such as a number and a point.  However, if \texttt{equal?}
is applied to two points, there are now two applicable methods; in
this case, the method that is more specific (i.e., whose specializer
objects are closest to the arguments in the inheritance graph) is
chosen, so the first method defined above is still chosen.

Now suppose we defined a method to compare colored objects, assuming a 
generic function \texttt{color} that retrieves the color value of a
colored object:

\begin{verbatim}
(add-method! equal? ((c1 <colored>) (c2 <colored>))
  (equal? (color c1) (color c2)))
\end{verbatim}

If we call \texttt{(equal?)} on two colored points, there are now three
applicable methods, and two of them (the ones comparing points and
comparing colored objects) are equally specific; in this case, an
error occurs.  The conflict can be resolved by adding a more specific
method that handled colored points:

\begin{verbatim}
(add-method! equal? ((cp1 <colored-point>) (cp2 <colored-point>))
  (and (equal? (direct cp1 <point>) (direct cp2 <point>))
       (equal? (direct cp1 <colored>) (direct cp2 <colored>))))
\end{verbatim}

The use of \texttt{direct} enables incremental inheritance, by
re-applying the generic function to the arguments but constraining the
applicable methods to those specific to the parent objects.

Now we show how to define the \texttt{x}, \texttt{y}, and
\texttt{color} generic functions, using storage tables.  A storage
table associates keys to values, where a key is a tuple of objects and
the value is the the value of the field attached to those objects.  If
the key consists of a single object, then this is equivalent to object
fields (also known as instance variables or slots) in most other OO
languages.  A generic function can contain both storage tables and
methods; reading a storage table uses the same syntax as applying a
method, with the application arguments forming the key that is looked
up in the table, and the value is returned.  Each storage table also
has a default value that is returned when a given key is not found.

\begin{verbatim}
(define-gf x)
(define-gf y)
(define-gf color)
(add-storage! x ((p <point>)) 0)
(add-storage! y ((p <point>)) 0)
(add-storage! color ((c <colored>)) 'black)
\end{verbatim}

After the above definitions, our \texttt{my-point} object now has
\texttt{x} and \texttt{y} fields, which can be accessed with
\texttt{(x my-point)} and \texttt{(y my-point)}; 
since \texttt{my-point} has not been associated with any value in
either storage table, they both return the default value, which is 0.
Storage tables are modified by using an extension of Scheme's
\texttt{set!} that instead of taking a variable on the left hand side, 
takes a generic function and a list of arguments:

\begin{verbatim}
(set! (x my-point) 3)
(set! (y my-point) 4)
\end{verbatim}

(The syntax is modeled after Common Lisp's \texttt{setf}.)  What assignment 
does is change the value associated with the tuple of arguments, or
add an association if it doesn't already exist in the storage table.

In order to be able to override a storage table with a setter method
without having to make the user switch between using the assignment
syntax and using method application syntax, there is a special kind of
method called an \texttt{acceptor} that is invoked with the same
syntax as assignment to a storage table:

\begin{verbatim}
(define-object <pos-point> <point>)
(add-acceptor! x ((p <pos-point>)) new-x
  (if (< new-x 0)
      (error "can't set negative x")
      (set! (x (direct p <point>)) new-x)))
\end{verbatim}

When an acceptor is assigned to, the value of the right hand side of
the assignment expression is bound to the variable specified after the
list of formal arguments (\texttt{new-x} in the above acceptor
definition), which can then be referred to in the body of the acceptor.

To construct anonymous objects (to play the role of instances from
class-based languages), we can simply define objects in a
local scope that gets discarded:

\begin{verbatim}
(define-gf make)
(add-method! make ((parent <object>) . args)
  (define-object anon parent)
  (apply initialize anon args)
  anon)
\end{verbatim}

We can then define methods on \texttt{initialize} that act like
constructors:

\begin{verbatim}
(define-gf initialize)
(add-method! initialize (obj . args)
  ; default: do nothing
  )
(add-method! initialize ((p <point>) (xx <number>) (yy <number>))
  (set! (x p) xx)
  (set! (y p) yy))
(add-method! initialize ((p <colored-point>) (xx <number>) (yy <number>) (c <symbol>))
  (initialize (direct p <point>) xx yy)
  (set! (color p) c))
\end{verbatim}

\subsection{Smud}

The driver for BOB is called Smud (short for ``Scheme mud'', although
it's not a mud---I'm still trying to think of a better name).  Smud
maintains a database of \emph{user} objects; each user object contains
a username, a password, and an environment.  When a user connects to a
Smud server with a valid username and password, a new thread is
created and the socket is connected to a read-eval-print loop in that
thread with the user object's environment.  There are two big
advantages to this arrangement: one is that each user has a separate
environment, and thus a separate name space; the other is that a
user's environment is persistent, so that you can disconnect from the
server and later reconnect and still have access to all the
environment definitions from the previous connection.

Initially, each user has an environment that consists of most of the
standard library definitions, as well as special versions of
user-specific definitions such as \texttt{current-input-port}.  Users
can communicate with each other by sending and publishing values; the
interface is based on the ``Shared Scheme'' portion of MUSEME, a
mud written in Scheme48\cite{MUSEME}, and is summarized in figure
\ref{shared-scheme}.

\begin{figure}
\begin{center}
\begin{grammar}
(send \nt{username} \nt{name} \nt{value}) \\
(share \nt{name} \nt{value}) \\
(get \nt{username} \nt{name}) \\
(unshare \nt{name}) \\
(forget \nt{username} \nt{name})
\end{grammar}
\end{center}
\caption{Communicating values between users in BOB.}\label{shared-scheme}
\end{figure}

Each user has an inbox and an outbox; \texttt{(send u n v)} associates 
\texttt{n} with \texttt{v} from the current user in \texttt{u}'s
inbox, and \texttt{(share n v)} associates \texttt{n} with \texttt{v}
in the current user's outbox.  \texttt{(get u n)} returns the value
associated with \texttt{n} from user \texttt{u} in the current user's
inbox, if it exists; otherwise it returns the value associated with
\texttt{n} in user \texttt{u}'s outbox (or throws an exception if it's 
not there either).  \texttt{unshare} and \texttt{forget} are used to
remove associations from the current user's outbox and inbox,
respectively.

For example, here is a simple bank account system (published by a user 
named \texttt{bank}:

\begin{verbatim}
(define-object <account>)
(define-gf balance)
(add-storage! balance ((acct <account>)) 0)
(define-gf transfer ((amt <number>) (from <account>) (to <account>))
  (if (> amt (balance from))
      (error "Insufficient funds")
      (begin
        (set! (balance from) (- (balance from) amt))
        (set! (balance to) (+ (balance to) amt)))))
(share <account> '<account>)
(share transfer 'transfer)
\end{verbatim}

Since the \texttt{balance} generic function is not shared, no one can
access account balances without going through the \texttt{transfer}
method.  A user can share his bank account object in various ways that 
protect access to it:

\begin{verbatim}
(define <account> (get 'bank '<account>))
(define acct (make <account>))
(send 'my-spouse 'acct acct)  ;; gives complete deposit and withdrawal access
(define-gf deposit ((amt <number>) (from <account>))
  ((get 'bank 'transfer) amt from acct))
(share 'deposit deposit)  ;; only gives deposit access
\end{verbatim}

The name used to send or share a value is usually a symbol, but may be 
any Scheme value; this allows a user to share a value with multiple users
(but not every user) by sharing it with a name that is a unique
object, and then sending that object to the users he wants to allow
access to the shared value.  The value shared may also be any Scheme
value, and will often be a procedure (or a generic function).

A value retrieved from another user may be assigned to a variable in
the current user's environment (or otherwise made ``local''), such as
\texttt{<account>} in the above example, or all references to that
value may be made through the inbox or outbox, such as
\texttt{transfer} in the above example; the latter case allows it to
be dynamically updated by the sending user, although even in the
former case the value may be dynamically updated if, for example, the
value is a procedure that refers to variables in the sending user's
environment, which can be altered by \texttt{set!}, or if the value is
a vector, which can be altered by \texttt{vector-set!}.

\section{Summary and Future Work}

What I've presented so far is the basic design for a new collaborative
programming language, as well as descriptions and a comparison of the
features of three existing languages.  My thesis is that this language
is a substantial improvement on these and other current languages.
Before I can demonstrate that, I need to make some improvements to the 
language, as well as survey additional languages and related systems,
and then I can compare the features of my language to the features of
the best of the existing ones.  In addition, my dissertation will
include formal semantics for subsets of my language and some of the
existing languages.  (Appendix \ref{FM} has one such semantics, for a
small subset of MOO.)

My strategy for expanding BOB and Smud is not simply to throw in a
bunch of features that are useful to collaborative languages, but to
add a reflection capability that is powerful enough to support
implementation of all of the other features as extensions to the base
language.  This has two big advantages: the base language can remain
very simple and elegant, and the various features can be implemented
several different ways in order to experiment with different design
ideas without having to change the whole language.  A particular
benefit of this latter feature is that different implementations can
exist simultaneously, so that, for example, two groups of users can
use two different security models for sharing code within each group,
and the groups could use a third security model for interacting
between groups.  My plan is to spend more effort on designing a good
reflection system, and to develop basic proof-of-concept
implementations for the various collaborative features without
spending too much time on any one in particular---many of them are
active research areas in their own right.

I have two ideas for how to do reflection in BOB.  One is to have a
meta-object protocol (MOP) which reifies programming constructs into
objects themselves, which lets the user extend the behavior of the
language by defining new methods that override the base behavior.  For
example, if there were a generic \texttt{<handler>} object, then a
\texttt{handler-applicable?} generic function might test a handler and
a list of arguments to determine if the handler were applicable;
you could then make a child of \texttt{<handler>},
\texttt{<private-handler>}, and attach a method to
\texttt{handler-applicable?} specialized on \texttt{<private-handler>} 
that returned false when the handler happened to be owned by another
user.  An alternative approach would be to use aspect-oriented
programming (AOP), which was designed as a restricted but safer
version of MOP; in AOP, a base program is combined with a set of
aspects that augment the base behavior in a way that crosscuts
multiple objects, and aspects can be selectively attached to different
sets of objects.  In the example above, you could define an aspect
that inserted code at the beginning of all private methods that
checked to see if the calling user were not the same as the owner, and
did a re-send to the next applicable method if so.

It's unclear to me what the advantages and disadvantages are of the
two approaches; AOP is meant to be safer, by not giving the user
access to certain parts of the base functionality, but it may turn out
not to be powerful enough to implement the collaborative features
needed.  I'm also not sure whether AOP is the best fit for a dynamic,
multiple-dispatch language; the only examples I am familiar with are
AspectJ\cite{AspectJ} and
Demeter/Java\cite{DemeterJava}\cite{DemeterJavaRD}, both of which
extend Java, whereas the main example of a MOP that I am familiar with
is from CLOS\cite{AMOP}, which is more akin to BOB.  This will be a
productive area of research, since the AOP field is currently very
active, and newer versions of AOP might be able to solve the problem.

Of the collaborative features I identified in section
\ref{features}, type safety, garbage collection, and multitasking
(multithreading) come for free by embedding the language in Larceny.
(Note that an object becomes garbage as soon as it has no pointers to
it in anyone's environment---there is no global namespace that holds
onto all objects regardless of other references, as there is in MOO,
ColdC, and LPC.)  A simple form of persistence also comes for free, in
that Larceny has a \texttt{dump-heap} procedure that saves the entire
state of the system to a file, which can be reloaded when starting
Larceny by giving the heap file name as a command line argument.  This
allows simple periodic checkpointing, similar to MOO, and I believe
this is sufficient for most needs (and far simpler to implement than
any alternative I can think of).  I have already addressed reflection,
available through the MOP.

This leaves resource control and security.  Controlling how much time
a user's task uses is much less important in a multithreading system
than in a cooperative multitasking system like MOO, ColdC, or LPC;
assuming the thread scheduler is fair, this is not an issue, although
a mechanism for adjusting thread priorities (using Larceny's ability
to set the interrupt timeslice) is worth exploring.  Controlling how
much memory a user allocates can easily be done through the MOP by
overriding the \texttt{make-object} function, as well as providing
protected versions of Scheme functions that allocate memory, such as
\texttt{cons} and \texttt{make-vector}.  Similarly, a simple
capability-based security can be implemented through the MOP using the
``seal'' idea from Rees's W7\cite{W7}; alternative systems that keep
track of a task's permissions (similar to MOO) would be interesting to
explore, though I have a less clear idea of how to implement that
through the MOP.

Other languages I intend to study include Muq\cite{Muq} (a mud that
actually has three different languages, based on C, Lisp, and Forth),
TinyMUSH\cite{MUSH} (a mud that allows programming based on scripting
of commands), POO\cite{POO} (a Python-based mud),
coolmud\cite{coolmud} (a distributed OO mud language, written by the
creator of MOO), FMPL\cite{FMPL} (a general-purpose collaborative
language somewhat like ColdC), and E\cite{E} (a scripting language for
secure, persistent, distributed programming).  There are also some
libraries for existing general-purpose languages that are useful from
a collaborative programming standpoint, such as SchMUSE\cite{SchMUSE}
(Scheme) and CL-HTTP\cite{CL-HTTP} (Common Lisp), as well as
inter-language systems like CORBA\cite{CORBA}.

\appendix
\section{FM: Featherweight MOO}
\label{FM}

\subsection{Purpose}

\subsection{Syntax}

\begin{verbatim}
P -> "return" E ";" P -> E ";" P

E -> "#-1" | "#0" | "#1"                        // constants
E -> "this" | "args" "[" INT "]"                // variables
E -> E ":" ID "(" [ EL ] ")"                    // verb invocation
E -> "create" "(" E ")"                         // object creation
E -> "set_verb_code" "(" E "," STR "," ESTR ")" // verb update

[Note: set_verb_code has different syntax and semantics in full MOO,
but a MOO database can be made to accept this syntax and act as
defined below with the help of a #0:bf_set_verb_code wrapper.]

EL -> E | E "," EL
ESTR -> E | STR

INT -> DIGIT+
STR -> "\"" CHAR* "\""
ID -> LETTER (LETTER|DIGIT)*
\end{verbatim}

\subsection{Semantics}

\begin{verbatim}
DB = Objnum -> (Objnum x Verbs)
Verbs = Str -> P
Objnum = Nat + { T }
Nat = set of natural numbers
Str = set of strings
\end{verbatim}

Objects are numbered with unique natural numbers; T is "top",
representing the top of the inheritance hierarchy (the parent of the
root object(s)), which is expressible as "\#$-$1" (object
negative-one).  A database (DB) is a partial function over objnums
such that DB(o) = <p, v> where p is the parent of o and v is the verb
map of o.  A verb map is a partial function mapping verb names
(strings) to programs.

\begin{verbatim}
db0 \in DB
db0(0) = <1, {}>
db0(1) = <T, {}>
\end{verbatim}

The starting database, db0, contains two objects, \#0 and \#1, the
latter being the parent of the former.  Neither object has any verbs.
Note that we could just as well start with one object (or none if
general objnums were expressible in the syntax), but this reflects the
standard initial MOO database where \#0 is the system object and \#1 is
the root object.

\begin{verbatim}
K -> halt
K -> <[]; p, K>
K -> <[]:id(e1...en), K>
K -> <v:id(v1..vi-1,[],ei+1...en), K>
K -> <create([]), K>
K -> <svc([],s,es), K>
K -> <svc(v,s,[]), K>
\end{verbatim}

K is the set of continuations.  A compound continuation is a
partly-evaluated expression with a hole in it, plus a next
continuation; thus, a continuation represents a stack of expressions
waiting to be completed.

\begin{verbatim}
Config = (P + E + Objnum) x K x DB
\end{verbatim}

A configuration is a triple whose first element is either a program or 
expression to be evaluated, or a value to be passed to the continuation.

\begin{verbatim}
eval : Config -> (Objnum x DB)
\end{verbatim}

The evaluation function maps configurations to configurations.  In the 
set of recursive definitions below, v stand for any element of Objnum,
db stands for any element of DB, e stands for any element of E, p stands 
for any element of P, k stands for any element of K, id stands for any 
element of ID, s stands for any element of STR.

Evaluations of programs and expressions:

\begin{verbatim}
eval(e; p, k, db) = eval(e, <[]; p, k>, db)
eval(return e, k, db) = eval(e, k, db)
eval(\#-1, k, db) = eval(T, k, db)
eval(\#0, k, db) = eval(0, k, db)
eval(\#1, k, db) = eval(1, k, db)
eval(e:id(e1...en), k, db) = eval(e, <[]:id(e1...en), k>, db)
eval(create(e), k, db) = eval(e, <create([]), k>, db)
eval(set_verb_code(e,s,es), k, db) = eval(e, <svc([],s,es), k>, db)
\end{verbatim}

Applications of continuations to values:

\begin{verbatim}
eval(v, halt, db) = <v, db>
eval(v, <[]; p, k>, db) = eval(p, k, db)
eval(v, <[]:id(e1...en), k>, db) = eval(e1, <v:id([],e2...en), k>, db)
eval(v, <v0:id(v1...vi-1, [], ei+1...en), k>, db)
    = eval(ei+1, <v0:id(v1...vi-1,v, [], ei+2...en), k>, db)
eval(v, <v0:id(v1...vn-1, []), k>, db)
    = eval(p[v0/this,v1/args[1]...vn-1/args[n-1],vn/args[n]], k, db)
    where lookup(db,v0,id) = p
    [In English: p is the program code of the verb named id on the v0
    object (or one of its ancestors); execute it, after replacing all
    occurences of "this" with v0 and "args[i]" with the corresponding vi.]
eval(v, <create([]), k>, db) = eval(vnew, k, db + {vnew:<v,{}>})
    where vnew = max(dom(db))+1
eval(v, <svc([],s,e), k>, db) = eval(e, <svc(v,s,[]), k>, db)
eval(v, <svc(v0,s,[]), k>, db) = eval(v0, k, dbnew)
    where dbnew = svc(db,v0,s,return v;)
    [This simulates a property with a verb that returns the value.]
eval(v, <svc([],s1,s2), k>, db) = eval(v, k, dbnew)
    where dbnew = svc(db,v,s1,parse(s2))

lookup : (DB x Objnum x ID) -> P
\end{verbatim}

The auxiliary function lookup takes an object and verbname and looks
up the inheritance chain of the object until it finds a verb
definition, returning the corresponding program.

\begin{verbatim}
lookup(db, v, id) = verbs(id) if v \in dom(db) and
                                 db(v) = <vp,verbs> and
                                 id \in dom(verbs)
lookup(db, v, id) = lookup(db, vp, id) if v \in dom(db) and
                                          db(v) = <vp,verbs> and
                                          id \not\in dom(verbs)

svc : (DB x Objnum x Str x P) -> DB
\end{verbatim}

The auxiliary function svc takes an object, verbname, and program and
adds the program as the verbname on the object, replacing the current
program if there is one.  Note that this does not go up the
inheritance chain; it always modifies the object itself.

\begin{verbatim}
svc(db, v, s, p) = db - {v:<vp,verbs>} + {v:<vp,verbsnew>}
    where verbsnew = verbs - s:pold + s:p
    if db(v) = <vp,verbs> and (verbs(s) = pold or s \not\in dom(verbs))

parse : Str -> P
\end{verbatim}

The auxiliary function parse converts strings to programs, using the
syntax grammar given in the previous section.


Thus, starting from db0, a stream of input programs can be evaluated
in sequence, producing a stream of results, modifying the database by
side-effect as it runs.  Note that "this" and "args[i]" are
meaningless in top-level input programs; they are evaluated by
substitution when a verb program is invoked.

\begin{verbatim}
eval(P1, halt, db) = <v, dbnew>
-------------------------------
run(db,P1,P2...) = v,run(dbnew,P2...)
\end{verbatim}

\bibliography{thesis}
\bibliographystyle{alpha}

\end{document}
