\documentclass{article}
\usepackage{amstex}
\usepackage{theorem}
\usepackage{fancyheadings}
\usepackage{program}

\setlength{\parindent}{0pt}	% Don't indent paragraphs.

%\addtolength{\voffset}{-0.25in}
%\addtolength{\textheight}{0.5in}
\addtolength{\hoffset}{-0.5in}	% Reduce left and right margins by 1/2 inch.
\addtolength{\textwidth}{1in}

\pagestyle{fancy}
\setlength{\headrulewidth}{0pt}

\lhead{COM 3350 Assignment 3}
\rhead{Doug Orleans (\texttt{dougo@@ccs.neu.edu})}

\theorembodyfont{\rmfamily}		% don't use italics
\theoremstyle{break}			% put a line break after the header
\newtheorem{exercise}{Exercise}
\theoremstyle{plain}
\newtheorem{subexercise}{}[exercise]
\renewcommand{\thesubexercise}{\alph{subexercise}.}
\newenvironment{answer}{\begin{quotation}\noindent}{\end{quotation}}

\newcommand{\sipser}{\textit{Sipser}}
\newcommand{\encoding}[1]{\ensuremath{\langle#1\rangle}}
\renewcommand{\qed}{~\ensuremath{\square}}
\newcommand{\Th}{\ensuremath{\operatorname{Th}}}
\newcommand{\Nat}{\ensuremath{\mathcal{N}}}
\newcommand{\tred}{\ensuremath{\leq_{\text{T}}}}
\newcommand{\ATM}{\ensuremath{A_{\textsf{TM}}}}
\renewcommand{\true}{\textsc{TRUE}}
\renewcommand{\false}{\textsc{FALSE}}
\newcommand{\defin}[1]{\textbf{\textit{#1}}}
\newcommand{\setname}[1]{\textit{#1}}
\renewcommand{\liminf}{\ensuremath{\lim_{n\to\infty}}}

\renewcommand{\AND}{\ \keyword{and}\ }
\newcommand{\ENDWHILE}{\untab\addtocounter{programline}{-1}}
\newcommand{\ENDIF}{\untab\untab\addtocounter{programline}{-1}}
\newcommand{\ACCEPT}{\keyword{accept}\ }
\newcommand{\REJECT}{\keyword{reject}\ }
\renewcommand{\keyword}[1]{\textbf{#1}} 
\renewcommand{\prognumstyle}{\textrm} 
\NumberProgramstrue 
\normalbaroutside 

\renewcommand{\a}{\stepcounter{subexercise}\textbf{\thesubexercise}\ }

\begin{document}

\begin{exercise}
Answer each part \true\ or \false.

\begin{tabular}{ll}
\textbf{(\sipser, 7.1)} \\
\a $2n=O(n)$. 		& \true.	\\
\a $n^2=O(n)$.		& \false.	\\
\a $n^2=O(n\log^2n)$.	& \false.	\\
\a $n\log n=O(n^2)$.	& \true.	\\
\a $3^n=2^{O(n)}$.	& \true; $3^n=2^{n\log_23}.$\\
\a $2^{2^n}=O(2^{2^n})$.& \true.	\\
\\
\textbf{(\sipser, 7.2)} \setcounter{subexercise}{0}\\
\a $n=o(2n)$.		& \false.	\\
\a $2n=o(n^2)$.		& \true.	\\
\a $2^n=o(3^n)$.	& \false.	\\
\a $1=o(n)$.		& \true.	\\
\a $n=o(\log n)$.	& \false.	\\
\a $1=o(1/n)$.		& \false.	\\
\end{tabular}
\end{exercise}

\begin{exercise}
Of the five symbols $O, o, \sim, \Theta, \Omega$, which are
transitive?  For example, if $f(n)=O(g(n))$ and $g(n)=O(h(n))$, does
$f(n)=O(h(n))$ hold?
\end{exercise}
\begin{align*}
\text{If }   \quad f(n)&\leq c_1g(n)\text{ for all }n>n_1 \\
\text{and }  \quad g(n)&\leq c_2h(n)\text{ for all }n>n_2,\\
\text{then } \quad f(n)&\leq c_1g(n)\leq c_1c_2h(n)
				    \text{ for all }n>\max(n_1,n_2).
\end{align*}
\begin{answer}
Thus $O$ is transitive, and similarly it's clear from the definitions
of $o$, $\Theta$, and $\Omega$ that they are also transitive.  Also,
\begin{align*}
\text{if }   \quad\liminf\frac{f(n)}{g(n)}&=1\\
\text{and }  \quad\liminf\frac{g(n)}{h(n)}&=1,\\
\text{then } \quad\liminf\frac{f(n)}{h(n)}
&=\liminf\frac{f(n)g(n)}{g(n)h(n)} \\
&=\liminf\frac{f(n)}{g(n)}\liminf\frac{g(n)}{h(n)} \\
&=1.
\end{align*}
Thus $\sim$ is also transitive.\qed
\end{answer}

\pagebreak
\begin{exercise}[\sipser, 7.12]\label{modexp}
Let
\begin{align*}
\setname{MODEXP}=\{\encoding{a,b,c,p}|~
&a,b,c,\text{ and $p$ are binary integers}\\
&\text{such that }a^b\equiv c\pmod{p}\}.
\end{align*}
Show that $\setname{MODEXP}\in$ P.  (Note that the most obvious algorithm
doesn't run in polynomial time.  Hint: Try it first where $b$ is a
power of 2.)
\end{exercise}
\begin{answer}
Let $b_i$ represent the $i$th binary digit of $b$.  Then the following
pseudo-code decides \setname{MODEXP}.
\begin{program}
m \gets 1
i \gets 1
a' \gets a \pmod{p}
\WHILE i \leq |length|(b) \AND b_i = 0
	i \gets i + 1
\ENDWHILE
\WHILE i \leq |length|(b)
	m \gets m * m \pmod{p}
	\IF b_i = 1
		\THEN m \gets m * a' \pmod{p}
	\ENDIF
\ENDWHILE
\IF m \equiv c \pmod{p}
	\THEN \ACCEPT
	\ELSE \REJECT
\ENDIF
\end{program}
This will involve at most $2|b|$ multiplications; I claim (without
proof, but it's a well-known algorithm) that two binary strings $x$
and $y$ can be multiplied in $O(|x||y|)$ time, and since $|m|$ and
$|a'|$ are at most $|p|$, each multiplication will be $O(|p|^2)$,
bringing the total time to $O(|b||p|^2)$.  The length of the input
is $n=O(|a|+|b|+|c|+|p|)$, thus the algorithm runs in $O(n^3)$ time,
and $\setname{MODEXP}\in$ P.\qed
\end{answer}

\begin{exercise}[\sipser, 7.13]
Show that P is closed under the star operation.  (Hint: On input
$y=y_1\cdots y_n$ for $y_i\in\Sigma$, build a table indicating for
each $i\leq j$ whether the substring $y_i\cdots y_j\in A^*$ for any
$A\in$ P.)
\end{exercise}
\begin{answer}
Given a language $A\in$ P, and an input string $y$ of length $n$, construct a
directed graph $G$ with nodes for every pair in the set
\begin{displaymath}
\{(0,0)\} \cup 
\{(i,j)|~1\leq i\leq j\leq n \text{ and } y_i\cdots y_j\in A\} \cup
\{(n+1,n+1)\}
\end{displaymath}
and edges from $(i,j)$ to $(j+1,k)$ for every $j<k\leq n+1$.
Constructing this graph requires deciding whether $n^2$ strings of
length at most $n$ are in $A$, thus the construction is polynomial in
$n$.  Now, deciding whether $y$ is in $A^*$ is equivalent to deciding
$\encoding{G,(0,0),(n+1,n+1)}\in\setname{PATH}$, which is polynomial
in the number of nodes in $G$, which is $O(n^2)$.  Thus $A^*\in$
P.\qed
\end{answer}

\begin{exercise}[\sipser, 7.16 (Part a)]
Let $G$ represent an undirected graph and let
\begin{align*}
\setname{SPATH}=\{\encoding{G,a,b,k}|~
&G\text{ contains a simple path of}\\
&\text{length at most $k$ from $a$ to $b$}\}.
\end{align*}
Show that $\setname{SPATH}\in$ P.
\end{exercise}
\begin{answer}
$G$ can be converted to an equivalent directed graph $G'$ in
polynomial time, by adding two directed edges for each undirected
edge.  Then we can run the algorithm for \setname{PATH}, but only loop
$k$ times.  If $G'$ contains a path from $a$ to $b$ of length at most
$k$, the modified algorithm will accept, otherwise it will reject.
The original algorithm runs in polynomial time, thus so does the
modified one, and thus \setname{SPATH} $\in$ P.\qed
\end{answer}

\begin{exercise}[\sipser, 7.20]
A \defin{permutation} on the set $\{1,\ldots,k\}$ is a one-to-one,
onto function on this set.  When $p$ is a permutation, $p^t$ means the
composition of $p$ with itself $t$ times.  Let
\begin{align*}
\setname{PERM-POWER}=\{\encoding{p,q,t}|~
&p=q^t \text{ where $p$ and $q$ are permutations}\\
&\text{on $\{1,\ldots,k\}$ and $t$ is a binary integer}\}.
\end{align*}
Show that $\setname{PERM-POWER}\in$ P.  (Note that the most obvious
algorithm doesn't run in polynomial time.  Hint: First try it where
$t$ is a power of 2).
\end{exercise}
\begin{answer}
Assume that $p$ and $q$ are sequences of $k$ binary integers between 1
to $k$; i.e.~$p=\encoding{p(1)p(2)\cdots p(k)}$.  Then
$|p|=|q|=O(k\lg k)$.  To compute $q^t$, use the algorithm from
Exercise~\ref{modexp} to compute $a^b$, with multiplication replaced
by composition. Applying a permutation $q$ takes $O(|q|)$ time: it
involves copying $k$ elements of length $O(\lg k)$.  Thus the total
computation takes $O(|q||t|)$ time.  The length of the input is
$n=O(|p|+|q|+|t|)$, thus this algorithm runs in $O(n^2)$ time, and
$\setname{PERM-POWER}\in$ P.\qed
\end{answer}
\end{document}
