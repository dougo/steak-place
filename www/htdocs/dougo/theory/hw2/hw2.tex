\documentclass{article}
\usepackage{amstex}
\usepackage{theorem}
\usepackage{fancyheadings}

\setlength{\parindent}{0pt}	% Don't indent paragraphs.

%\addtolength{\voffset}{-0.25in}
%\addtolength{\textheight}{0.5in}
\addtolength{\hoffset}{-0.5in}	% Reduce left and right margins by 1/2 inch.
\addtolength{\textwidth}{1in}

\pagestyle{fancy}
\setlength{\headrulewidth}{0pt}

\lhead{COM 3350 Assignment 2}
\rhead{Doug Orleans (\texttt{dougo@@ccs.neu.edu})}

\theorembodyfont{\rmfamily}		% don't use italics
\theoremstyle{break}			% put a line break after the header
\newtheorem{exercise}{Exercise}
\theoremstyle{plain}
\newtheorem{subexercise}{}[exercise]
\renewcommand{\thesubexercise}{(\alph{subexercise})}
\newenvironment{answer}{\begin{quotation}\noindent}{\end{quotation}}

\newcommand{\sipser}{\textit{Sipser}}
\newcommand{\encoding}[1]{\ensuremath{\langle#1\rangle}}
\newcommand{\qed}{~\ensuremath{\square}}
\newcommand{\Th}{\ensuremath{\operatorname{Th}}}
\newcommand{\Nat}{\ensuremath{\mathcal{N}}}
\newcommand{\tred}{\ensuremath{\leq_{\text{T}}}}
\newcommand{\ATM}{\ensuremath{A_{\textsf{TM}}}}
\newcommand{\set}[1]{\ensuremath{\{#1\}}}
\newcommand{\ruleset}[2]{\set{#1|~#2}}
\newcommand{\union}{\ensuremath{\cup}}

\begin{document}

\begin{exercise}[\sipser, 6.4]
Is the statement $\exists x \forall y [x+y=y]$ a member of
$\Th(\Nat,+)$?  Why or why not?  What about the statement $\exists x
\forall y [x+y=x]$?
\end{exercise}
\begin{answer}
The statement $\exists x \forall y [x+y=y]$ is true in the model
$(\Nat,+)$ with $x=0$; however, the statement
$\exists x \forall y [x+y=x]$ cannot be made true in the model
$(\Nat,+)$ because it's only true for $y=0$.\qed
\end{answer}

\begin{exercise}[\sipser, 6.9]
Give a model of the sentence
\begin{eqnarray*}
\phi_{\text{eq}} =
&       &\forall x [R_1(x,x)] \\
&\wedge &\forall x,y [R_1(x,y) \leftrightarrow R_1(y,x)] \\
&\wedge &\forall x,y,z [(R_1(x,y) \wedge R_1(y,z)) \to R_1(x,z).
\end{eqnarray*}
\end{exercise}
\begin{answer}
The model $(\Nat,=)$ satisfies this sentence.  (Any equivalence
relation will do.)\qed
\end{answer}

\begin{exercise}[\sipser, 6.10]
Give a model of the sentence
\begin{eqnarray*}
\phi_{\text{lt}} =
&	&\phi_{\text{eq}} \\
&\wedge &\forall x [R_1(x,x) \to \neg R_2(x,x)] \\
&\wedge &\forall x,y [\neg R_1(x,y) \to R_2(x,y) \oplus R_2(y,x)] \\
&\wedge &\forall x,y,z [(R_2(x,y) \wedge R_2(y,z)) \to R_2(x,z) \\
&\wedge &\forall x \exists y [R_2(x,y)].
\end{eqnarray*}
\end{exercise}
\begin{answer}
The model $(\Nat,=,>)$ satisfies this sentence.\qed
\end{answer}

\begin{exercise}[\sipser, 6.3]
Show that if $A\tred B$ and $B\tred C$ then $A\tred C$.
\end{exercise}
\begin{answer}
Let $M^B$ be the oracle Turing machine that decides $A$, and $M^C$ be
the oracle Turing machine that decides $B$.  Then $M^B$ can use $M^C$
as an oracle to decide $A$; thus $A\tred C$.\qed
\end{answer}

\begin{exercise}[\sipser, 6.7]
Show that for any two languages $A$ and $B$ a language $J$ exists
where $A\tred J$ and $B\tred J$.
\end{exercise}
\begin{answer}
Let $a$ and $b$ be symbols not in the alphabets of $A$ and $B$.  Let
$J$ be the language
\begin{displaymath}
\ruleset{aw}{w\in A} \union \ruleset{bw}{w\in B}.
\end{displaymath}
Then we can build a machine $M^J$ that accepts input $w$ iff $aw$ is
in $J$, thus deciding $A$, and a machine $N^J$ that accepts input $w$
iff $bw$ is in $J$, thus deciding $B$.  Hence both $A$ and $B$
Turing-reduce to $J$.\qed
\end{answer}

\pagebreak
\begin{exercise}[\sipser, 6.13]\label{desc}
Show how to compute the descriptive complexity of strings $K(x)$ with
an oracle for \ATM.
\end{exercise}
\begin{answer}
The set of Turing machines is countable, since each description is a
finite string of symbols from a finite alphabet.  Also, any language
over a finite alphabet is countable.  Thus we can construct a machine
that enumerates strings \encoding{M,w} in increasing order by length,
and for each one asks the oracle for \ATM\ whether $M$ followed by
$X$, a machine that accepts $x$, accepts $w$.  The first such string
for which the oracle answers yes will then be the minimal description
of $x$, and we can return $K(x)=$ the length of \encoding{M,w}.\qed
\end{answer}

\begin{exercise}[\sipser, 6.14]
Use the result of Exercise~\ref{desc} to give a function $f$ that is
computable with an oracle for \ATM, where for each $n$, $f(n)$ is an
incompressible string of length $n$.
\end{exercise}
\begin{answer}
We can construct a machine which, given $n$, enumerates each string of
length $n$, and for each one runs the machine constructed in
Exercise~\ref{desc} to find its descriptive complexity, halting when
it finds a string whose descriptive complexity is equal to its length
with that string on its tape.  Then the function $f$ is the function
computed by this machine.\qed
\end{answer}

\begin{exercise}[\sipser, 6.21]
In Corollary~4.15 we showed that the set of all languages is
uncountable.  Use this result to prove that languages exist that are
not recognizable by an oracle Turing machine with oracle for \ATM.
\end{exercise}
\begin{answer}
An oracle Turing machine is still finite, i.e.~it has a finite
description over a finite description language.  Thus there are only
countably many oracle Turing machines for a particular oracle;
therefore there are more languages than oracle Turing machines with
oracle for \ATM, so some must not be recognizable.\qed
\end{answer}
\end{document}
