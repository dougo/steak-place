\documentclass{article}
\usepackage{amstex}
\usepackage{amssymb}
\usepackage{theorem}
\usepackage{fancyheadings}
%\usepackage{program}

\setlength{\parindent}{0pt}	% Don't indent paragraphs.

\addtolength{\voffset}{-0.5in}
\addtolength{\textheight}{1in}
\addtolength{\hoffset}{-0.5in}	% Reduce left and right margins by 1 inch.
\addtolength{\textwidth}{1in}

\pagestyle{fancy}
\setlength{\headrulewidth}{0pt}

\lhead{COM 3350 Assignment 7}
\rhead{Doug Orleans (\texttt{dougo@@ccs.neu.edu})}

\theorembodyfont{\rmfamily}		% don't use italics
\theoremstyle{break}			% put a line break after the header
\newtheorem{exercise}{Exercise}
\theoremstyle{plain}
\newtheorem{subexercise}{}[exercise]
\renewcommand{\thesubexercise}{\alph{subexercise}.}
\newenvironment{answer}{\begin{quotation}\noindent}{\end{quotation}}

\newcommand{\sipser}{\textit{Sipser}}
\newcommand{\encoding}[1]{\ensuremath{\langle#1\rangle}}
\newcommand{\qed}{~\ensuremath{\square}}
\newcommand{\Th}{\ensuremath{\operatorname{Th}}}
\newcommand{\Nat}{\ensuremath{\mathcal{N}}}
\newcommand{\tred}{\ensuremath{\leq_{\text{T}}}}
\newcommand{\ATM}{\ensuremath{A_{\textsf{TM}}}}
\newcommand{\defin}[1]{\textbf{\textit{#1}}}
\newcommand{\sym}[1]{\ensuremath{\mathtt{#1}}}
\newcommand{\symbf}[1]{\textbf{\sym{#1}}}
\newcommand{\set}[1]{\ensuremath{\{#1\}}}
\newcommand{\enumlang}[1]{\set{\sym{#1}}}
\newcommand{\union}{\cup}
\newcommand{\setname}[1]{\textit{#1}}
\renewcommand{\liminf}{\ensuremath{\lim_{n\to\infty}}}
\newcommand{\TM}{\textsf{TM}}
\newcommand{\NTM}{\textsf{NTM}}
\newcommand{\DFA}{\textsf{DFA}}
\newcommand{\NFA}{\textsf{NFA}}
\renewcommand{\P}{\textrm{P}}
\newcommand{\NP}{\textrm{NP}}
\newcommand{\coNP}{\textrm{coNP}}
\newcommand{\emptystring}{\ensuremath{\boldsymbol\varepsilon}}

%\renewcommand{\AND}{\ \keyword{and}\ }
%\newcommand{\ENDWHILE}{\untab\addtocounter{programline}{-1}}
%\newcommand{\ENDIF}{\untab\untab\addtocounter{programline}{-1}}
%\newcommand{\ACCEPT}{\keyword{accept}\ }
%\newcommand{\REJECT}{\keyword{reject}\ }
%\renewcommand{\keyword}[1]{\textbf{#1}} 
%\renewcommand{\prognumstyle}{\textrm} 
%\NumberProgramstrue 
%\normalbaroutside 

\renewcommand{\a}{\stepcounter{subexercise}\textbf{\thesubexercise}\ }

\begin{document}

\begin{exercise}
\begin{subexercise}[\sipser, 9.2]
Prove that TIME($2^n$) $\subsetneq$ TIME($2^{2n}$).
\end{subexercise}
\begin{displaymath}
\lim_{n\to\infty}\frac{2^n}{2^{2n}/\lg 2^{2n}}
=\lim_{n\to\infty}\frac{2n2^n}{2^{2n}} 
=\lim_{n\to\infty}\frac{2n}{2^n} 
=\lim_{n\to\infty}\frac{2}{n2^{n-1}} 
=0
\end{displaymath}
\begin{answer}
Thus $2^n=o(2^{2n}/\lg 2^{2n})$, and since $2^{2n}$ is time
constructible, it follows from Corollary~9.11 that TIME($2^n$)
$\subsetneq$ TIME($2^{2n}$).\qed
\end{answer}

\begin{subexercise}[\sipser, 9.3]
Prove that NTIME($n$) $\subsetneq$ PSPACE.
\end{subexercise}
\begin{answer}
NTIME($n$) $\subseteq$ NSPACE($n$) $\subseteq$ SPACE($n^2$)
$\subsetneq$ SPACE($n^3$) $\subseteq$ PSPACE.\qed
\end{answer}
\end{exercise}

\begin{exercise}
\begin{subexercise}[\sipser, 9.6]
Prove that if $A\in\P$ then $\P^A=\P$.
\end{subexercise}
\begin{answer}
For any language $L\in\P^A$, we can replace writing a string to the
oracle tape with simply computing whether the string is in $A$, which
computation is in $\P$.  Thus $L\in\P$.  Also, for any $L\in\P$, we
can construct a polynomial-time oracle Turing machine that simply
ignores its oracle, making $L\in\P^A$.\qed
\end{answer}

\begin{subexercise}[\sipser, 9.9]
Show that if $\NP = \P^{\setname{SAT}}$ then $\NP = \coNP$.
\end{subexercise}
\begin{answer}
Since oracle Turing machines are deterministic, $\P^A$ is closed under
complementation for any language $A$.  Thus,
\begin{displaymath}
L\in\NP
\implies L\in\P^{\setname{SAT}}
\implies\overline L\in\P^{\setname{SAT}}
\implies\overline L\in\NP
\implies L\in coNP.
\end{displaymath}
Also, $L\in\coNP$ implies $\overline L\in\NP$, which by the above
reasoning implies $\overline L\in\coNP$, which implies
$L\in\NP$.\qed
\end{answer}
\end{exercise}

\begin{exercise}[\sipser, 9.7]
Give regular expressions with exponentiation that generate the
following languages over the alphabet $\enumlang{0,1}$.
\begin{subexercise}
All strings of length 500.
\end{subexercise}
\begin{answer}
$(\sym 0\union\sym 1)^{500}$
\end{answer}
\begin{subexercise}
All strings of length 500 or less.
\end{subexercise}
\begin{answer}
$(\sym 0\union\sym 1\union\emptystring)^{500}$
\end{answer}
\begin{subexercise}
All strings of length 500 or more.
\end{subexercise}
\begin{answer}
$(\sym 0\union\sym 1)^{500}(\sym 0\union\sym 1)^*$
\end{answer}
\begin{subexercise}
All strings of length different than 500.
\end{subexercise}
\begin{answer}
$(\sym 0\union\sym 1\union\emptystring)^{499}\union(\sym 0\union\sym 1)^{501}(\sym 0\union\sym 1)^*$
\end{answer}
\begin{subexercise}
All strings that contain exactly 500 \sym 1s.
\end{subexercise}
\begin{answer}
$(\sym 0^*\sym 1)^{500}\sym 0^*$
\end{answer}
\begin{subexercise}
All strings that contain at least 500 \sym 1s.
\end{subexercise}
\begin{answer}
$(\sym 0^*\sym 1)^{500}(\sym 0\union\sym 1)^*$
\end{answer}
\begin{subexercise}
All strings that contain at most 500 \sym 1s.
\end{subexercise}
\begin{answer}
$(\sym 0^*(\sym 1\union\emptystring))^{500}\sym 0^*$
\end{answer}
\begin{subexercise}
All strings of length 500 or more that contain a \sym 0\ in the 500th
position.
\end{subexercise}
\begin{answer}
$(\sym 0\union\sym 1)^{499}\sym 0(\sym 0\union\sym 1)^*$
\end{answer}
\begin{subexercise}
All strings that contain two \sym 0s that have at least 500 symbols
between them.
\end{subexercise}
\begin{answer}
$(\sym 0\union\sym 1)^*\sym 0(\sym 0\union\sym 1)^{500}\sym 0(\sym 0\union\sym 1)^*$
\end{answer}
\end{exercise}

\addtocounter{exercise}{-1}
\begin{exercise}[\sipser, 9.8]
If $R$ is a regular expression, let $R^{\set{m,n}}$ represent the
expression \[R^m\union R^{m+1}\union\cdots\union R^n.\]
Show how to implement the $R^{\set{m,n}}$ operator using the ordinary
exponentiation operator, but without ``$\cdots$''.
\end{exercise}
\begin{answer}
$R^m(R\union\emptystring)^{n-m}.$\qed
\end{answer}

\begin{exercise}[\sipser, 9.10]
Show the error in the following fallacious ``proof'' that P $\not=$ NP.
Assume that P = NP.  Then \setname{SAT} $\in$ P.  So, for some $k$,
\setname{SAT} $\in$ TIME($n^k$).  Because every language in NP is
polynomial time reducible to \setname{SAT}, NP $\subseteq$
TIME($n^k$).  Therefore P $\subseteq$ TIME($n^k$).  But, by the time
hierarchy theorem, TIME($n^{k+1}$) contains a language which isn't in
TIME($n^k$), which contradicts P $\subseteq$ TIME($n^k$).  Therefore P
$\not=$ NP.
\end{exercise}
\begin{answer}
Every language in NP is polynomial time reducible to \setname{SAT},
but this reduction might take $\Omega(n^k)$ time, which would dominate
the $O(n^k)$ time to decide \setname{SAT}.  Thus we can't deduce that
NP $\subseteq$ TIME($n^k$).\qed
\end{answer}

\begin{exercise}[\sipser, 9.11]
Show that the language \setname{MAX-CLIQUE} from problem~7.30 is in
$\NP^{\setname{SAT}}$.
\end{exercise}
\begin{answer}
Since both \setname{CLIQUE} and \setname{SAT} are NP-complete,
\setname{CLIQUE} $\leq_{\text{P}}$ \setname{SAT}.  Thus, we could
build an oracle Turing machine with an oracle for \setname{SAT} that,
given $\encoding{G,k}$, for each $i$ from $|G|$ down to $k$, asked its
oracle if $f(\encoding{G,i})\in\setname{SAT}$, where $f$ is a
polynomial-time computable function that reduces \setname{CLIQUE} to
\setname{SAT}.  The machine would accept if and only if the oracle
answered ``no'' for every $i>k$ and ``yes'' for $i=k$.  Since this
algorithm is a polynomial repetition of a polynomial time function,
it's actually in $\P^{\setname{SAT}}$ (which is of course a subset of
$\NP^{\setname{SAT}}$).\qed
\end{answer}
\end{document}
