\documentclass{article}
\usepackage{amstex}
\usepackage{theorem}
\usepackage{fancyheadings}

\setlength{\parindent}{0pt}	% Don't indent paragraphs.

%\addtolength{\voffset}{-0.25in}
%\addtolength{\textheight}{0.5in}
\addtolength{\hoffset}{-0.25in}	% Reduce left and right margins by 1/4 inch.
\addtolength{\textwidth}{0.5in}

\pagestyle{fancy}
\setlength{\headrulewidth}{0pt}

\lhead{COM 3350 Assignment 1}
\rhead{Doug Orleans (\texttt{dougo@@ccs.neu.edu})}

\theorembodyfont{\rmfamily}		% don't use italics
\theoremstyle{break}			% put a line break after the header
\newtheorem{exercise}{Exercise}
\theoremstyle{plain}
\newtheorem{subexercise}{}[exercise]
\renewcommand{\thesubexercise}{(\alph{subexercise})}

\newcommand{\sipser}{\textit{Sipser}}
\newcommand{\encoding}[1]{\ensuremath{\langle#1\rangle}}
\newcommand{\qed}{~\ensuremath{\square}}

\newenvironment{answer}{\begin{quotation}\noindent}{\end{quotation}}

\begin{document}

\begin{exercise}[\sipser, 6.1]
Give an example in the spirit of the recursion theorem of a program in
a real programming language (or a reasonable approximation thereof)
that prints itself out.
\end{exercise}
Here is a straightforward, but verbose, answer in Java:
\begin{verbatim}
class Self {
  static String add_escapes(String str) {
    String retval = "";
    char ch;
    for (int i = 0; i < str.length(); i++) {
      if ((ch = str.charAt(i)) == '\n') {
        retval += "\\n\" +\n\"";
      } else if (ch == '\"') {
        retval += "\\\"";
      } else if (ch == '\\') {
        retval += "\\\\";
      } else {
        retval += ch;
      }
    }
    return retval;
  }
  public static void main(String[] args) {
    System.out.println(B + add_escapes(B) + "\";}");
  }
  static String B = "class Self {\n" +
"  static String add_escapes(String str) {\n" +
"    String retval = \"\";\n" +
"    char ch;\n" +
"    for (int i = 0; i < str.length(); i++) {\n" +
"      if ((ch = str.charAt(i)) == '\\n') {\n" +
"        retval += \"\\\\n\\\" +\\n\\\"\";\n" +
"      } else if (ch == '\\\"') {\n" +
"        retval += \"\\\\\\\"\";\n" +
"      } else if (ch == '\\\\') {\n" +
"        retval += \"\\\\\\\\\";\n" +
"      } else {\n" +
"        retval += ch;\n" +
"      }\n" +
"    }\n" +
"    return retval;\n" +
"  }\n" +
"  public static void main(String[] args) {\n" +
"    System.out.println(B + add_escapes(B) + \"\\\";}\");\n" +
"  }\n" +
"  static String B = \"";}
\end{verbatim}
\noindent Here is a less verbose version that avoids the problem of
converting newlines:
\begin{verbatim}
class Self3 {
  public static void main(String[] args) {
    int i;
    for (i = 0; i < s.length; i++)
      System.out.println(s[i].replace(Character.toUpperCase('x'), '"'));
    for (i = 0; i < s.length; i++)
      System.out.println('"' + s[i] + '"' + ',');
    System.out.println("  };");
    System.out.println('}');
  }
  static String[] s = {
"class Self3 {",
"  public static void main(String[] args) {",
"    int i;",
"    for (i = 0; i < s.length; i++)",
"      System.out.println(s[i].replace(Character.toUpperCase('x'), 'X'));",
"    for (i = 0; i < s.length; i++)",
"      System.out.println('X' + s[i] + 'X' + ',');",
"    System.out.println(X  };X);",
"    System.out.println('}');",
"  }",
"  static String[] s = {",
  };
}
\end{verbatim}
\noindent Here is my first try at a shortest answer (344 chars) [enter
on a single line]:
\begin{verbatim}
class S{static char q='"';static String s=
"class S{static char q='';static String s=;
public static void main(String[]a){StringBuffer b=new StringBuffer(s);
b.insert(41,q+s+q);b.insert(23,q);System.out.println(b);}}";
public static void main(String[]a){StringBuffer b=new StringBuffer(s);
b.insert(41,q+s+q);b.insert(23,q);System.out.println(b);}}
\end{verbatim}
\noindent Here is the shortest answer I could come up with (331 chars)
[enter on a single line]:
\begin{verbatim}
class S{public static void main(String[]a){new S().p('"',"'",
"class S{public static void main(String[]a){new S().p('",
");}void p(char d,String q,String a,String b){
System.out.println(a+d+q+','+d+q+d+','+d+a+d+','+d+b+d+b);}}"
);}void p(char d,String q,String a,String b){
System.out.println(a+d+q+','+d+q+d+','+d+a+d+','+d+b+d+b);}}
\end{verbatim}
In contrast, this single line eval in MOO prints itself:
\begin{verbatim}
;;s=";player:tell(\";;s=\"+toliteral(s)+s)";player:tell(";;s="+toliteral(s)+s)
\end{verbatim}
\par$\square$\pagebreak
\begin{exercise}[\sipser, 6.5]
Describe two different Turing machines, $M$ and $N$, where, when
started on any input, $M$ outputs \encoding N and $N$ outputs
\encoding M.
\end{exercise}
\begin{answer}
This answer depends on the fact that two ``different'' machines may
have the same behavior, but differing implementations.  We define a
single Turing machine behavior $B$.

$B = $ ``On any input:
\begin{enumerate}
\item Obtain, via the recursion theorem, own description \encoding B.
\item If \encoding B has any useless states, construct a new machine
$B'$, which is $B$ with all useless states eliminated; print
\encoding{B'}; and halt.
\item Otherwise, construct a new machine $B''$, which is $B$ with one
useless state added; print \encoding{B''}; and halt.''
\end{enumerate}
(Note that discerning whether a machine has useless states is a simple
task of comparing the set of states $Q$ with the range of the $\delta$
function, and is thus a decidable problem.)

We then construct two machines that implement the behavior of $B$:
$M$, which has no useless states, and $N$, which is the output of $M$
and has one useless state, and which therefore will output \encoding
M.\qed
\end{answer}
\begin{exercise}[\sipser, 6.6]
In the fixed-point version of the recursion theorem (Theorem 6.6) let
the transformation $t$ be a function that interchanges the states
$q_{\text{accept}}$ and $q_{\text{reject}}$ in Turing machine
descriptions.  Give an example of a fixed point for $t$.
\end{exercise}
\begin{answer}
Any Turing machine that never halts for any input string will be a
fixed point for $t$; for example, a machine that moves right on any
input symbol and never enters $q_{\text{accept}}$ or
$q_{\text{reject}}$ will recognize the empty language whether or not
$q_{\text{accept}}$ and $q_{\text{reject}}$ are interchanged.\qed
\end{answer}
\end{document}
