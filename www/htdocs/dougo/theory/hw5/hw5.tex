\documentclass{article}
\usepackage{amstex}
\usepackage{theorem}
\usepackage{fancyheadings}
\usepackage{program}

\setlength{\parindent}{0pt}	% Don't indent paragraphs.

%\addtolength{\voffset}{-0.5in}
%\addtolength{\textheight}{1in}
%\addtolength{\hoffset}{-0.5in}	% Reduce left and right margins by 1/2 inch.
%\addtolength{\textwidth}{1in}

\pagestyle{fancy}
\setlength{\headrulewidth}{0pt}

\lhead{COM 3350 Assignment 5}
\rhead{Doug Orleans (\texttt{dougo@@ccs.neu.edu})}

\theorembodyfont{\rmfamily}		% don't use italics
\theoremstyle{break}			% put a line break after the header
\newtheorem{exercise}{Exercise}
\theoremstyle{plain}
\newtheorem{subexercise}{}[exercise]
\renewcommand{\thesubexercise}{\alph{subexercise}.}
\newenvironment{answer}{\begin{quotation}\noindent}{\end{quotation}}

\newcommand{\sipser}{\textit{Sipser}}
\newcommand{\encoding}[1]{\ensuremath{\langle#1\rangle}}
\renewcommand{\qed}{~\ensuremath{\square}}
\newcommand{\Th}{\ensuremath{\operatorname{Th}}}
\newcommand{\Nat}{\ensuremath{\mathcal{N}}}
\newcommand{\tred}{\ensuremath{\leq_{\text{T}}}}
\newcommand{\ATM}{\ensuremath{A_{\textsf{TM}}}}
\renewcommand{\true}{\textsc{TRUE}}
\renewcommand{\false}{\textsc{FALSE}}
\newcommand{\defin}[1]{\textbf{\textit{#1}}}
\newcommand{\setname}[1]{\textit{#1}}
\renewcommand{\liminf}{\ensuremath{\lim_{n\to\infty}}}
\newcommand{\TM}{\textsf{TM}}
\newcommand{\NTM}{\textsf{NTM}}
\newcommand{\emptystring}{\ensuremath{\boldsymbol\varepsilon}}

\renewcommand{\AND}{\ \keyword{and}\ }
\newcommand{\ENDWHILE}{\untab\addtocounter{programline}{-1}}
\newcommand{\ENDIF}{\untab\untab\addtocounter{programline}{-1}}
\newcommand{\ACCEPT}{\keyword{accept}\ }
\newcommand{\REJECT}{\keyword{reject}\ }
\renewcommand{\keyword}[1]{\textbf{#1}} 
\renewcommand{\prognumstyle}{\textrm} 
\NumberProgramstrue 
\normalbaroutside 

\renewcommand{\a}{\stepcounter{subexercise}\textbf{\thesubexercise}\ }

\begin{document}

\begin{exercise}[\sipser, 8.1]\label{tape}
Show that for any function $f:\Nat\to\Nat$, where $f(n)\geq n$, the
space complexity class SPACE($f(n)$) is the same whether you define
the class by using the single-tape \TM\ model or the two tape read-only
input \TM\ model.
\end{exercise}
\begin{answer}
In the two tape read-only input \TM\ model, we can simply copy the
input to the second tape, which uses $O(n)$ space, then use the second
tape as if it were a single-tape \TM.  If the single-tape algorithm is
in SPACE($f(n)$), then it uses $O(f(n))$ space and the total space
used by the two-tape algorithm will be $O(\max(n, f(n)))=O(f(n))$,
thus it's still in SPACE($f(n)$); similarly if it's not in
SPACE($f(n)$), then it uses $\omega(f(n))$ space, and the total space
used by the two-tape algorithm will be
$\omega(\max(n,f(n)))=\omega(f(n))$, thus it won't be in
SPACE($f(n)$).\qed
\end{answer}

\begin{exercise}
\begin{subexercise}[\sipser, 8.4]
Show that PSPACE is closed under the operations union,
complementation, and star.
\end{subexercise}
\begin{answer}
Union:
\begin{answer}
Let $L_1$ and $L_2$ be languages in PSPACE, and let $M_1$ and $M_2$ be
single-tape \TM s that decide them and take no more than polynomial
space.  We can construct a two tape read-only input tape \TM\ $M$ that
decides $L_1\union~L_2$.

\begin{answer}
$M =$ ``On input $w$:
\begin{enumerate}
\item Copy $w$ to the second tape.
\item Run $M_1$ on the second tape.  If it accepts, then accept.
\item Erase the second tape, and copy $w$ to it again.
\item Run $M_2$ on the second tape.  If it accepts, then accept,
otherwise reject.''
\end{enumerate}
\end{answer}
$M$ takes no more space than the maximum of $M_1$ and $M_2$; thus it
takes polynomial space, and by Exercise~\ref{tape}, its language
$L_1\union L_2$ is in PSPACE.\qed
\end{answer}

\noindent Complementation:
\begin{answer}
Let $L$ be a language in PSPACE, and let $M$ be a \TM\ that decides $L$
in polynomial space.  We can construct a new machine $M'$ that decides
$\overline L$ by simply switching the accept and reject states of $M$;
$M'$ takes the same amount of space as $M$, thus $\overline L$ is also
in PSPACE.\qed
\end{answer}

\pagebreak
\noindent Star:
\begin{answer}
Let $L$ be a language in PSPACE, and let $M$ be a \TM\ that decides $L$
in polynomial space.  We can construct a two tape \NTM\ $M'$ that
decides $L^*$.

\begin{answer}
$M' =$ ``On input $w$:
\begin{enumerate}
\item If $w=\emptystring$, accept.
\item Otherwise, nondeterministically split $w$ into a sequence of substrings
$w_1,\ldots,w_n$ where $1\leq n \leq |w|$.
\item For each string $w_i$:
	\begin{enumerate}
	\item Erase the second tape, and copy $w_i$ to it.
	\item Run $M$ on the second tape.
	\end{enumerate}
\item If $M$ ever rejects, reject, otherwise accept.
\end{enumerate}
\end{answer}
$M'$ nondeterministically decides $L^*$; step 2 takes $O(|w|)$ space,
and step 3 takes no more than the space that $M$ takes on $w$, since
each substring $w_i$ is no larger than $w$.  Thus $L^*\in$ NPSPACE $=$
PSPACE.\qed
\end{answer}
\end{answer}

\begin{subexercise}[\sipser, 8.6]\label{PSPACE-hard}
Show that any PSPACE-hard language is also NP-hard.
\end{subexercise}
\begin{answer}
Let $L$ be a PSPACE-hard language.  Every language in PSPACE is
polynomial-time reducible to $L$.  But every language in NP is also in
PSPACE; therefore, every language in NP is polynomial-time reducible
to $L$, or in other words, $L$ is NP-hard.\qed
\end{answer}
\end{exercise}

\begin{exercise}[\sipser, 8.9]
Show that, if every NP-hard language is also PSPACE-hard, then PSPACE
$=$ NP.
\end{exercise}
\begin{answer}
Assume every NP-hard language is also PSPACE-hard, and let $L$ be any
NP-complete language.  By our assumption, $L$ must be PSPACE-hard;
thus, every language in PSPACE can be reduced in polynomial time to
$L$.  This implies that every language in PSPACE is also in NP,
because we can just construct a \NTM\ that reduces to $L$ in
polynomial time and then decides $L$ in polynomial time (because
$L\in$ NP).  Thus PSPACE $\subseteq$ NP; also, we know that NP
$\subseteq$ PSPACE, so they must be equal.\qed
\end{answer}
\end{document}
