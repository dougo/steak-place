\documentclass{article}
\usepackage{amstex}
\usepackage{theorem}
\usepackage{fancyheadings}
\usepackage{program}

\setlength{\parindent}{0pt}	% Don't indent paragraphs.

\addtolength{\voffset}{-1in}
\addtolength{\textheight}{2in}
\addtolength{\hoffset}{-1in}	% Reduce left and right margins by 1 inch.
\addtolength{\textwidth}{2in}

\pagestyle{fancy}
\setlength{\headrulewidth}{0pt}

\lhead{COM 3350 Assignment 6}
\rhead{Doug Orleans (\texttt{dougo@@ccs.neu.edu})}

\theorembodyfont{\rmfamily}		% don't use italics
\theoremstyle{break}			% put a line break after the header
\newtheorem{exercise}{Exercise}
\theoremstyle{plain}
\newtheorem{subexercise}{}[exercise]
\renewcommand{\thesubexercise}{\alph{subexercise}.}
\newenvironment{answer}{\begin{quotation}\noindent}{\end{quotation}}

\newcommand{\sipser}{\textit{Sipser}}
\newcommand{\encoding}[1]{\ensuremath{\langle#1\rangle}}
\renewcommand{\qed}{~\ensuremath{\square}}
\newcommand{\Th}{\ensuremath{\operatorname{Th}}}
\newcommand{\Nat}{\ensuremath{\mathcal{N}}}
\newcommand{\tred}{\ensuremath{\leq_{\text{T}}}}
\newcommand{\ATM}{\ensuremath{A_{\textsf{TM}}}}
\renewcommand{\true}{\textsc{TRUE}}
\renewcommand{\false}{\textsc{FALSE}}
\newcommand{\defin}[1]{\textbf{\textit{#1}}}
\newcommand{\setname}[1]{\textit{#1}}
\renewcommand{\liminf}{\ensuremath{\lim_{n\to\infty}}}
\newcommand{\TM}{\textsf{TM}}
\newcommand{\NTM}{\textsf{NTM}}
\newcommand{\DFA}{\textsf{DFA}}
\newcommand{\NFA}{\textsf{NFA}}
\newcommand{\emptystring}{\ensuremath{\boldsymbol\varepsilon}}
\newcommand{\sym}[1]{\texttt{#1}}

\renewcommand{\AND}{\ \keyword{and}\ }
\newcommand{\ENDWHILE}{\untab\addtocounter{programline}{-1}}
\newcommand{\ENDIF}{\untab\untab\addtocounter{programline}{-1}}
\newcommand{\ACCEPT}{\keyword{accept}\ }
\newcommand{\REJECT}{\keyword{reject}\ }
\renewcommand{\keyword}[1]{\textbf{#1}} 
\renewcommand{\prognumstyle}{\textrm} 
\NumberProgramstrue 
\normalbaroutside 

\renewcommand{\a}{\stepcounter{subexercise}\textbf{\thesubexercise}\ }

\begin{document}

\begin{exercise}[\sipser, 8.5]
Show that NL is closed under the operations union, intersection, and
star.
\end{exercise}
\begin{answer}
Union: Given two \NTM s $M_1$ and $M_2$ such that $L(M_1)$ and
$L(M_2)$ are in NL, construct a \NTM\ $M$ that nondeterministically
chooses which of $M_1$ and $M_2$ to run on its input.  $M$ uses no
more space than $M_1$ and $M_2$, thus nondeterministically decides
$L(M_1)\cup L(M_2)$ in logarithmic space.
\end{answer}
\begin{answer}
Intersection: Given two \NTM s $M_1$ and $M_2$ such that $L(M_1)$ and
$L(M_2)$ are in NL, construct a \NTM\ $M$ that first runs $M_1$ on its
input, and rejects if $M_1$ rejects; otherwise, erases its work tape,
resets its heads, and runs $M_2$.  $M$ uses no more space than $M_1$
and $M_2$, thus nondeterministically decides $L(M_1)\cap L(M_2)$ in
logarithmic space.
\end{answer}
\begin{answer}
Star: Given a \NTM\ $M$ such that $L(M)$ is in NL, construct the
following \NTM.
\begin{quotation}
\noindent Let $M'=$ ``On input $w$,
\begin{enumerate}
\item If $w=\emptystring$, accept.
\item Otherwise, let $i=1$.
\item While $i\leq |w|$:
\begin{enumerate}
\item Nondeterministically choose $j$ between $i$ and $|w|$, inclusive.
\item Erase the work tape (except for $\encoding i\encoding j$), and
run $M$ on $w_{i\ldots j}$, rejecting if $M$ rejects.
\item Set $i=j+1$.
\end{enumerate}
\item Accept.''
\end{enumerate}
\end{quotation}
Storing $\encoding i\encoding j$ requires only $O(\log |w|)$
space; thus, $M'$ nondeterministically decides $L(M)^*$ in logarithmic
space.\qed
\end{answer}

\begin{exercise}[\sipser, 8.8]
Show that $A_{\NFA}$ is NL-complete.
\end{exercise}
\begin{answer}
We can reduce \setname{PATH} to it: given $\encoding{G,s,t}$ produce a
\NFA\ $M$ with one state per node in $G$, a transition labeled
$\emptystring$ for every edge in $G$, and with $s$ as the start state
and $t$ the only final state.  Now deciding whether $M$ accepts
$\emptystring$ decides whether there is a path from $s$ to $t$ in $G$;
since this translation requires only a constant amount of work space,
\setname{PATH} $\leq_{\text{L}} A_{\NFA}$, and since \setname{PATH} is
NL-complete, $A_{\NFA}$ is NL-hard.  To see that $A_{\NFA}$ is in NL,
the only work space needed while simulating a \NFA\ with a \NTM\ are
pointers to the current state and the current input symbol, which can
be recorded in logarithmic space.  Thus $A_{\NFA}$ is NL-complete.\qed
\end{answer}

\begin{exercise}[\sipser, 8.11]
Let $A$ be the language of properly nested parentheses.  For example,
\sym{(())} and \sym{(()(()))()} are in $A$, but \sym{)(} is
not.  Show that $A$ is in L.
\end{exercise}
\begin{answer}
We can construct a machine that simply scans the input linearly and
keeping a counter of the number of unclosed open parentheses,
i.e.~start at 0, increment whenever it sees a \sym(, and
decrement whenever it sees a \sym).  If the counter ever goes
negative, or if the counter is not at 0 when it reaches the end of
input, it rejects, otherwise it accepts.  Storing the counter takes
logarithmic space in the length of the input, thus $A$ is in L.\qed
\end{answer}

\begin{exercise}[\sipser, 8.12]
Let $B$ be the language of properly nested parentheses and brackets.
For example, \sym{([()()]()[])} is in $B$, but \sym{([)]} is
not.  Show that $B$ is in L.
\end{exercise}
\begin{answer}
The natural way to do this would be to keep a stack, pushing every
open symbol, and popping for every close symbol, making sure that the
popped open symbol matches the current close symbol; however, this
would require linear space.  For each symbol, we can calculate what
its depth on the stack would have been by simply keeping a counter, as
in the previous algorithm; thus we can simulate a stack by passing
over the input for $d = 0\ldots|w|/2$, keeping track of what would be
in the $d$th position of the stack as we go, and making sure that the
corresponding symbols at the $d$th level match.  This takes $O(|w|^2)$
time, but it only needs to keep two things in memory: $d$, which needs
logarithmic space, and the current symbol to match at the $d$th level,
which only needs a constant amount of space.  Thus this algorithm
decides $B$ in logarithmic space, meaning $B$ is in L.\qed
\end{answer}

\begin{exercise}[\sipser, 8.18]
A \textit{chain} is a sequence of strings $s_1,s_2,\ldots,s_k$,
wherein every string differs from the preceding one in exactly one
character.  For example the following is a chain of English words:
\begin{quote}
head, hear, near, fear, bear, beer, deer, deed, feed, feet, fret, free
\end{quote}
Let
\begin{align*}
\setname{DFACHAIN} = \{\encoding{M,s,t}|~
&M \text{ is a \DFA\ where $L(M)$ contains a} \\
&\text{chain of strings, starting with $s$, and ending with $t$.}\}
\end{align*}
Show that \setname{DFACHAIN} is in PSPACE.
\end{exercise}
\begin{answer}
We can construct a \NTM\ that decides \setname{DFACHAIN} using only
polynomial space:
\begin{quotation}
\noindent Let $A=$ ``On input $\encoding{M,s,t}$,
\begin{enumerate}
\item While $s \not= t$:
	\begin{enumerate}
	\item Nondeterministically choose a string $s'$ that differs
	from $s$ in exactly one character.  There are only a finite
	number of choices, since $s$ is finite and the alphabet is
	finite.
	\item Decide whether $M$ accepts $s'$,
	i.e.,~$\encoding{M,s'}\in A_{\DFA}$.  If not, reject. 
	\item Set $s=s'$.
	\end{enumerate}
\item Accept.''
\end{enumerate}
\end{quotation}
Since we need not store the whole chain, the only work space needed is
for $s'$, which is the same length as $s$, and the space needed by
deciding $A_{\DFA}$, which is logarithmic by Exercise~8.7.  (Note that
the while loop may not terminate, because the choice of $s'$ might
keep repeating a previous choice, but we can put an upper bound on the
number of iterations equal to the number of possible strings of length
$|s|$.)  $A$ nondeterministically decides \setname{DFACHAIN} in
polynomial space; thus, \setname{DFACHAIN} is in NPSPACE, which is
equal to PSPACE.\qed
\end{answer}

\begin{exercise}[\sipser, 8.20]
An undirected graph is \defin{bipartite} if its nodes may be divided
into two sets so that all edges go from a node in one set to a node in
the other set.  Show that a graph is bipartite if and only if it
doesn't contain a cycle that has an odd number of nodes.  Let
\begin{displaymath}
\setname{BIPARTITE}=\{\encoding{G}|~G\text{ is a bipartite graph}\}.
\end{displaymath}
Show that \setname{BIPARTITE} $\in$ NL.
\end{exercise}
\begin{answer}
Choose any node in $G$, and traverse $G$ depth-first from there,
putting nodes in alternate sets as you go.  If you hit a node that's
already been assigned to the same set as the one you assigned the
previous node to, then you know $G$ can't be bipartite; the only way
this can happen is if $G$ has a cycle of odd length.  Otherwise, if we
can repeat this traversal until all nodes are assigned successfully,
$G$ must be bipartite.  A full depth-first traversal, however, needs
more than logarithmic space; instead, we can decide
$\overline{\setname{BIPARTITE}}$ by nondeterministically following a
path through $G$, keeping track of the first node and the length of
the path, and accepting if and only if we come across the first node
again after an odd number of nodes.  (Again, this may not terminate,
but we can put an upper bound of $|G|+1$ nodes.)  The work space
needed is logarithmic in $|G|$, thus \setname{BIPARTITE} is in coNL;
since coNL = NL, \setname{BIPARTITE} is thus in NL.\qed
\end{answer}
\end{document}
