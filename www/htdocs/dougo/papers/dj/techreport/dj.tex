\documentclass{llncs}
%\documentclass{article}
\usepackage{html}
\usepackage{epsfig}

\newcommand{\defn}[1]{\textbf{#1}}
\newcommand{\code}[1]{\texttt{#1}}

% Reduce margins by 1/2 inch.
\addtolength{\voffset}{-0.5in}
\addtolength{\textheight}{1in}
\addtolength{\hoffset}{-0.5in}
\addtolength{\textwidth}{1in}

\title{DJ: Dynamic Adaptive Programming in Java}
\author{Doug Orleans and Karl Lieberherr}
\institute{Northeastern University, Boston, MA 02115, USA}

\begin{document}

\begin{htmlonly}
This paper is also available in
\htmladdnormallink{postscript}{dj.ps} form.
\end{htmlonly}

\maketitle

\noindent (This paper has been submitted to the Reflection 2001 conference.
It is also available as Northeastern University Tech Report NU-CCS-2001-02.)

\begin{abstract}
\noindent
Following the Law of Demeter, a programming style rule for loose
coupling between the structure and behavior concerns, can result in a
large number of small methods scattered throughout the program, which
can make it hard to understand the high-level picture of what a
program does.  Adaptive programming with traversal strategies and
adaptive visitors in DemeterJ avoids this problem while even better
supporting this loose coupling of concerns.  DJ is a pure-Java package
for adaptive programming that allows traversal strategies to be
constructed and interpreted dynamically at run-time.  Traversal
strategies can be used with adaptive visitors or in a generic
programming style by adapting them to the Java Collections framework.
The DJ package makes heavy use of Java reflection and we give the
highlights of this implementation.
\end{abstract}

\section{Introduction}

The Law of Demeter~\cite{karl-ian:soft1} is a style rule for OOP whose
goal is to reduce the behavioral dependencies between classes.  Its
primary form says that a method $M$ should only call methods (and
access fields) on objects which are \defn{preferred suppliers}:
immediate parts on \code{this}, objects passed as arguments to $M$,
objects which are created directly in $M$, and objects in global
variables (in Java, \code{public static} fields).  Limiting which
methods can call which other methods keeps programmers from encoding
too much information about the object model into a method, thus
loosening the coupling between the structure concern and the behavior
concern.

The drawback of following the Law of Demeter is that it can result in
a large number of small methods scattered throughout the program,
which can make it hard to understand the high-level picture of what a
program does.  \cite{wilde-etal:maint-support} studied three
medium-sized object-oriented systems and found that in all three
systems, 50\% of the methods were less than 2 C++ statements or 4
Smalltalk lines long.  The example in the next section also
demonstrates this effect.  The
DemeterJ~\cite{cse:preventive}~\cite{DemeterJ:00} software allows the
programmer to succinctly specify a \defn{traversal
strategy}~\cite{strategies-tr:LP97} in place of the many methods that
would be needed to traverse a complex object structure.  These
traversal methods are automatically generated at compile-time by a
preprocessor.  A traversal strategy describes a traversal at a high
level, only referring to the minimal number of classes in the
program's object model: the root of the traversal, the target classes,
and waypoints and constraints in between to restrict the traversal to
follow only the desired set of paths.  If the object model changes,
often the traversal strategy doesn't need to be changed; the traversal
methods can simply be re-generated in accordance with the new model,
and the behavior adapts to the new structure.  Programming with
traversal strategies is thus known as \defn{adaptive programming}
\cite{karl:demeter}.

With the addition of reflection to Java~\cite{java-reflection}, it
became possible to interpret a traversal strategy at runtime.  DJ
\cite{DJ:99} is a pure-Java package that provides this capability.
This makes it easier to add traversal strategies to a Java program,
because there is no need to modify the compilation process to run the
preprocessor or to convert the source code to the Demeter input
language syntax.  Instead traversal strategies can simply be expressed
as Java strings in ordinary Java code, or even constructed dynamically
from an external source not known at compile time.

\section{Example}

The example domain for this paper will be that of processing XML
Schema definitions~\cite{XML-schema}.  A simple task that one might
want to implement is checking a schema for undefined types.  This task
involves two traversals of the object structure representing the
schema definition: one to collect all the types defined in the schema,
and one to check each type reference to see if it's in the set of
defined types.  Figure~\ref{fig:UML} shows a UML class diagram that
represents a small subset of the XML Schema definition language;
figure~\ref{fig:LoD} shows the Java code for these classes, along with
the methods for these two traversals.  The methods
\code{getDefinedTypeNames} and \code{getUndefinedTypeNames} on class
\code{Schema} are the public interfaces to the traversals, and the
methods \code{addDef} and \code{addUndef} do the actual traversal,
building up the sets of type names.

\begin{figure}
\epsfig{file=UML.ps,height=\textwidth,width=4in,angle=270}
\caption{UML diagram for XML Schemas.}
\label{fig:UML}
\end{figure}

\begin{figure}[tbp!]
\begin{minipage}{3in}
{\small
\begin{verbatim}
import java.util.*;

class Schema {
  Attribute attrs[];
  SchemaItem items[];
  public Set getDefinedTypeNames() {
    Set def = new HashSet();
    addDef(def);
    return def;
  }
  public Set getUndefinedTypeNames() {
    Set undef = new HashSet();
    Set def = getDefinedTypeNames();
    addUndef(undef, def);
    return undef;
  }
  void addDef(Set def) {
    for (int i = 0; i < items.length; i++)
      items[i].addDef(def);
  }
  void addUndef(Set undef, Set def) {
    for (int i = 0; i < items.length; i++)
      items[i].addUndef(undef, def);
  }
}
abstract class SchemaItem {
  void addDef(Set def) { }
  void addUndef(Set undef, Set def) { }
}
abstract class TypeDef extends SchemaItem {
  Attribute attrs[];
  void addDef(Set def) {
    for (int i = 0; i < attrs.length; i++)
      attrs[i].addDef(def);
  }
}
class Attribute {
  String name;
  String value;
  void addDef(Set def) {
    if (name.equals("name"))
      def.add(value);
  }
  void addUndef(Set undef, Set def) {
    if (name.equals("type")
        && !def.contains(value))
      undef.add(value);
  }
}
\end{verbatim}
}
\end{minipage}
\begin{minipage}{3in}
{\small
\begin{verbatim}
class SimpleType extends TypeDef { }
class ComplexType extends TypeDef {
  SequenceGroup content;
  AttributeDecl adecls[];
  void addDef(Set def) {
    super.addDef(def);
    content.addDef(def);
  }
  void addUndef(Set undef, Set def) {
    content.addUndef(undef, def);
    for (int i = 0; i < adecls.length; i++)
      adecls[i].addUndef(undef, def);
  }
}
class SequenceGroup {
  Attribute attrs[];
  ElementDecl edecls[];
  void addDef(Set def) {
    for (int i = 0; i < edecls.length; i++)
      edecls[i].addDef(def);
  }
  void addUndef(Set undef, Set def) {
    for (int i = 0; i < edecls.length; i++)
      edecls[i].addUndef(undef, def);
  }
}
abstract class Decl extends SchemaItem {
  Attribute attrs[];
  void addUndef(Set undef, Set def) {
    for (int i = 0; i < attrs.length; i++)
      attrs[i].addUndef(undef, def);
  }
}
class AttributeDecl extends Decl {
}
class ElementDecl extends Decl {
  TypeDef typeDef;
  void addDef(Set def) {
    if (typeDef != null)
      typeDef.addDef(def);
  }
  void addUndef(Set undef, Set def) {
    if (typeDef != null)
      typeDef.addUndef(undef, def);
  }
}
\end{verbatim}
}
\end{minipage}
\caption{Following the Law of Demeter.}
\label{fig:LoD}
\end{figure}

Note that the Law of Demeter is strictly followed: each method only
refers to fields defined on the same class.  However, the overall
algorithm is lost in the noise of all the traversal methods.
The actual functional behavior is split between the \code{Schema} and
\code{Attribute} classes.  Moreover, even though each method only
refers to local fields, deciding whether to traverse a field requires
knowledge of the overall class structure: for example, in
\code{SequenceGroup}, the \code{addDef} method only needs to
traverse the \code{edecls} field because an element declaration may
include a type definition; if the object model were extended so that an
attribute declaration could also include a type definition, the
\code{addDef} method in \code{ComplexType} would have to be changed to
traverse the \code{adecls} field, even though nothing about
\code{ComplexType} itself changed.

Another way of implementing this example would be to use the Visitor
design pattern~\cite{gang-of-4}, by creating two classes
\code{TypeDefVisitor} and \code{DeclVisitor}, moving the traversal
methods into \code{visit} methods on those classes, and making
subclasses overriding \code{visit(Attribute)} to perform the behavior
of checking for defined and undefined types.  While this would
eliminate the scattering of traversal methods across the class
structure, the same set of traversal methods would need to be written,
and they would still need to be modified when the object model
changes.

\section{DJ}

DJ is a library of classes that make traversals like the previous
example much easier to define, understand, and maintain.
Figure~\ref{fig:DJ} shows an alternate implementation of the
\code{Schema} class that defines the two traversals succinctly
using the \code{ClassGraph} and \code{Visitor} classes from the
\code{edu.neu.ccs.demeter.dj} package.

\begin{figure}[bp!]
\begin{minipage}{\textwidth}
\begin{verbatim}
import java.util.*;
import edu.neu.ccs.demeter.dj.*;

class Schema {
  Attribute attrs[];
  SchemaItem items[];
  static final ClassGraph cg = new ClassGraph();
  public Set getDefinedTypeNames() {
    final Set def = new HashSet();
    cg.traverse(this, "from Schema via ->TypeDef,attrs,* to Attribute",
                new Visitor() {
                    void before(Attribute host) {
                      if (host.name.equals("name"))
                        def.add(host.value);
                    }
                  });
    return def;
  }
  public Set getUndefinedTypeNames() {
    final Set def = getDefinedTypeNames();
    final Set undef = new HashSet();
    cg.traverse(this, "from Schema via ->Decl,attrs,* to Attribute",
                new Visitor() {
                    void before(Attribute host) {
                      if (host.name.equals("type")
                          && !def.contains(host.value))
                        undef.add(host.value);
                    }
                  });
    return undef;
  }
}
\end{verbatim}
\end{minipage}
\caption{Using \code{traverse} from the DJ library.}\label{fig:DJ}
\end{figure}

A \code{ClassGraph} object is a simplified representation of a UML
\cite{rational:UML-LUG} class diagram; its nodes are types (classes and
primitive types) and its edges are (uni-directional) associations and
(bi-directional) generalizations.
%In Demeter terminology, a (uni-directional)
%association relationship is implemented as a \defn{construction} edge
%if its multiplicity is 1 (or 0..1) or a \defn{repetition} edge if its
%multiplicity is *; a generalization relationship is implemented as an
%\defn{inheritance} edge; and a specialization relationship is
%implemented as an \defn{alternation} edge.
The default \code{ClassGraph} constructor builds a graph object using
reflection from all the classes in the default package; a string
containing a package name can be provided as a constructor argument to
build a class graph from another package.  The
methods \code{addPackage(String~pkgname)} and \code{addClass(Class~cl)}
can be used to add other packages and classes to a class graph
object.  A traversal is done by calling the \code{traverse} method on a
\code{ClassGraph} object.  It takes three arguments: the root of the
object structure to be traversed; a string specifying the traversal
strategy to be used; and an \defn{adaptive visitor} object
describing what to do at points in the traversal.

A traversal strategy specifies the end points of the traversal, using
the \code{from} keyword for the source and the \code{to}
keyword for the target(s).  In between, any number of constraints can be
specified with \code{via} or \code{bypassing}.  The two
traversals in figure~\ref{fig:DJ} traverse from \code{Schema} to
\code{Attribute}; in other words, they visit attributes in a schema,
because type names appear in attribute values for both definitions and
references.  They differ in their constraints: to find the names of
types defined by the schema, the first traversal only looks at
attributes of type definitions (\code{TypeDef} objects); to find the
names of types referenced by the schema, the second traversal only
looks at attributes of declarations (\code{Decl} objects).  The
\code{->TypeDef,attrs,*} syntax is a pattern specifying the set of
association edges whose source is class \code{TypeDef} and whose
label (field name) is \code{attrs}; the asterisk means that an edge in
the set can have any target type.

Traversal strategy interpretation is done as described in
\cite{strategies-tr:LP97}, with a few modifications whose details will
be presented in a future paper.  The general idea is that at each
object in the traversal, those associations (including inherited
associations) which can possibly lead to a target object (subject to
any constraints specified in the traversal strategy) are traversed
sequentially; if an object is encountered which has no possible path
leading to a target object, the traversal returns to the previous step
in the traversal.  For example, in our XML Schema example, the
\code{items} field of \code{Schema} contains an array of
\code{SchemaItem} objects; this array may contain \code{TypeDef}
objects, since \code{TypeDef} is a subclass of \code{SchemaItem}, so
the elements of the array are traversed as part of the
\code{getDefinedTypes} traversal.  However, some of the elements may
be \code{AttributeDecl} objects, and there is no possible path to a
\code{TypeDef} object; if one of these elements is encountered in the
array, it is simply skipped over.  The \code{adecls} field of
\code{ComplexType} is never traversed at all, since it can only
contain an array of \code{AttributeDecl} objects.  Note that if the
\code{adecls} field were a \code{Vector} instead of an array, it
could contain objects of any type, and so DJ would have to traverse it
in case one of its elements were a \code{TypeDef} object or some
other object that could lead to a \code{TypeDef}.  If parameteric
polymorphism is added to Java, such as that proposed in~\cite{GJ},
this problem will be easier to avoid: the type of \code{adecls} could
be \code{List<AttributeDecl>} and DJ would know it could avoid it.

An adaptive visitor class is a subtype of the \code{Visitor} class in
the DJ package; it implements the Adaptive Visitor pattern described
in~\cite[pp.~426-427]{karl:demeter}.  The Adaptive Visitor pattern
differs from the Visitor pattern as presented in~\cite{gang-of-4}
in two ways: only a minimal set of methods needs to be defined, namely
those describing the functional behavior to be performed at points
along the traversal, rather than one method each for every class in
the traversal; and no \code{accept} methods need to be defined, nor
does traversal behavior need to be defined in the visitor methods.
These two differences result in a unit of behavior that can adapt both
to changes in the object model and changes in the traversal.

During a traversal with adaptive visitor $V$, when an object $o$ of type $T$ is
reached in the traversal, if there is a method on $V$ named
\code{before} whose parameter is type $T$, that method is called with
$o$ as the argument.  Then, each field on the object is traversed if
needed.  Finally, before returning to the previous object, if there is
a method on $V$ named \code{after} whose parameter is type $T$, that
method is called with $o$ as the argument.  The \code{Visitor}
subclasses defined inline in figure~\ref{fig:DJ} only define one
\code{before} method each, which is executed at \code{Attribute}
objects, the end point of the traversal.

DJ also provides support for generic programming
\cite{musser-stepanov:GL}: the \code{asList} method on
\code{ClassGraph} adapts an object structure and a traversal strategy
into a \code{List}, part of Java's Collections
framework~\cite{Collections}.  The object structure is viewed as a
collection of objects whose type is the target of the traversal
strategy; the collection's iterator performs the traversal
incrementally with each call to \code{next}.  Figure~\ref{fig:asList}
shows how to rewrite the previous example using \code{asList}.

\begin{figure}[tbhp!]
\begin{minipage}{\textwidth}
\begin{verbatim}
import java.util.*;
import edu.neu.ccs.demeter.dj.*;

class Schema {
  Attribute attrs[];
  SchemaItem items[];
  static final ClassGraph cg = new ClassGraph();
  public Set getDefinedTypeNames() {
    final Set def = new HashSet();
    List typeDefAttributes =
      cg.asList(this, "from Schema via ->TypeDef,attrs,* to Attribute");
    Iterator it = typeDefAttributes.iterator();
    while (it.hasNext()) {
      Attribute attr = (Attribute) it.next();
      if (attr.name.equals("name"))
        def.add(attr.value);
    }
    return def;
  }
  public Set getUndefinedTypeNames() {
    final Set def = getDefinedTypeNames();
    final Set undef = new HashSet();
    List declAttributes = 
      cg.asList(this, "from Schema via ->Decl,attrs,* to Attribute");
    Iterator it = declAttributes.iterator();
    while (it.hasNext()) {
      Attribute attr = (Attribute) it.next();
      if (attr.name.equals("type")
          && !def.contains(attr.value))
        undef.add(attr.value);
    }
    return undef;
  }
}
\end{verbatim}
\end{minipage}
\caption{Using the collection adaptor \code{asList}.}
\label{fig:asList}
\end{figure}

\section{Implementation Highlights}

In this section we present some highlights of the implementation of DJ
and some examples of interesting uses.

When the \code{ClassGraph} constructor is called, it creates a graph
object containing reflective information about all the classes in a
package.  In Java, however, there is no way to get a list of all
classes in a package; packages are just namespaces, not containers.
Moreover, the JVM only knows about classes that have already been
loaded, and it only loads classes when they are referenced.  Since a
class graph might be constructed before many of the classes in
the package have been referenced, the constructor has to discover
classes some other way: it searches the class path (provided by the
JVM as \code{System.getProperty("java.class.path")}) for all
\code{.class} files in subdirectories corresponding to the package
name.  For each class file that is found, it calls
\code{Class.forName()} with the class name, which causes the JVM to
load the class if it hasn't already been loaded.  If there are
classes that need to be added to a class graph that do not exist
as \code{.class} files in the class path, for example if they are
loaded from the network or constructed dynamically, they will need to
be added explicitly by calling \code{addClass()}.

A class graph may also be created from another class graph $G$ and a
traversal strategy $S$, forming the subgraph of classes and edges in
$G$ that would be traversed according to $S$.  This can be used to
remove unwanted paths from a class graph, such as backlinks, rather
than having to add \code{bypassing} constraints to every traversal
strategy.

The \code{traverse} method on \code{ClassGraph} is implemented in a
two-stage process: first, a \defn{traversal graph} is computed from
the class graph and the traversal strategy (which itself is converted
into a \defn{strategy graph}, whose nodes are the classes mentioned in
the traversal strategy and whose edges each have constraints attached
to that leg of the traversal); then, the object structure is
traversed, using information from the traversal graph to decide where
to go next at each step, and visitor methods are invoked as needed.
The traversal graph computation takes time proportional to the product
of the number of edges in the class graph and the number of edges in
the strategy graph; since the same traversal strategy is often reused
multiple times with the same class graph, the traversal graph can be
saved and reused without needing to be recomputed every time.  The
class \code{TraversalGraph} has a constructor that takes a traversal
strategy and a \code{ClassGraph} object, as well as methods
\code{traverse} and \code{asList}.  The traversal computation
algorithm is also available as a separate package, the AP Library
\cite{AP-Lib:99}.

At each step in a traversal, the fields and methods of the current
object, as well as methods on the visitor object, are inspected and
invoked by reflection.  Some of this reflective overhead could be
avoided by generating a new class (at run-time) that invokes the
appropriate fields and methods directly; this is planned for a future
addition to DJ.  Other applications of partial evaluation to speed up
the traversal may be possible as well.

The implementation of \code{asList} is somewhat trickier than regular
traversal: the list iterator must return in the middle of the
traversal whenever a target object is reached, and then resume where
it left off when \code{next} is called again.  An earlier version
created an ad-hoc continuation-like object that was saved and restored
at each iteration, but this was error-prone and not very efficient;
the current version uses a separate Java thread as a coroutine,
suspending and resuming at each iteration.  An additional provided method
\code{gather} can be used to copy all the target objects into an
\code{ArrayList}, which is faster still, but the list returned by
\code{asList} has the advantage that calls to \code{set} on the
iterator can replace target objects in the original object structure.

Java's reflection system, unlike other meta-object protocols
\cite{AMOP}, has no mechanism for \defn{intercession}: there is no way
to make a new subclass of \code{Class} that behaves differently for
certain meta-operations such as method invocation.  However, DJ's
\code{Visitor} class does allow a limited form of intercession.  It
has the method \code{before(Object~obj, Class~cl)} (and corresponding
\code{after}), which is invoked by the \code{ClassGraph.traverse}
method at each traversal step; it looks for a method named
\code{before} with a single parameter whose type is the class
represented by \code{cl}, and invokes it with \code{obj} as argument.
This method can be overridden by a subclass to perform more dynamic
behavior based on the reified class object of the object being
traversed.  Figure~\ref{fig:intercession} shows a simple
pretty-printing visitor that uses this technique, along with a method
on class \code{Schema} that uses it.

\begin{figure}[tbhp!]
\begin{minipage}{\textwidth}
\begin{verbatim}
class XMLPrinter extends Visitor {
  public XMLPrinter(Map map) { this.map = map; }
  Map map;
  String indent = "";
  public void before(Object obj, Class cl) {
    String elementName = (String) map.get(cl);
    if (elementName != null) {
      System.out.println(indent + "<" + elementName + ">");
      indent += "  ";
    }
  }
  public void after(Object obj, Class cl) {
    String elementName = (String) map.get(cl);
    if (elementName != null) {
      indent = indent.substring(2);
      System.out.println(indent + "</" + elementName + ">");
    }
  }
}

class Schema {
  ...
  void print() {
    Map map = new HashMap();
    map.put(Schema.class, "schema");
    map.put(SimpleType.class, "simpleType");
    map.put(ComplexType.class, "complexType");
    map.put(ComplexTypeContent.class, "complexContent");
    map.put(SequenceGroup.class, "sequence");
    map.put(ElementDecl.class, "element");
    map.put(AttributeDecl.class, "attribute");
    Visitor v = new XMLPrinter(map);
    XSD.cg.traverse(this, "from Schema to *", v);
  }
}
\end{verbatim}
\end{minipage}
\caption{Using visitor method intercession.}
\label{fig:intercession}
\end{figure}

Note that the \code{XMLPrinter} visitor class is generic, in that it
makes no mention of any of the XML Schema classes but is parameterized
by a mapping of classes to element names.  

\section{Related Work}

DJ is closely related to DemeterJ~\cite{DemeterJ:00}, a preprocessing
tool that takes a \defn{class dictionary file} (containing a textual
representation of a UML class diagram, with syntax directives for
parsing and printing object structures) and some \defn{behavior files}
(containing regular Java methods to be attached to the classes in the
class dictionary, plus traversal method specifications, visitor
methods, and \defn{adaptive methods} that connect a traversal with a
visitor class) and generates plain Java code for those classes with
traversal methods attached (similar to the code in
Figure~\ref{fig:LoD}) along with a parser and some custom visitors
such as for printing, copying, or comparing object structures.  DJ
shares the same traversal strategy language and traversal graph
algorithms as DemeterJ, but does no code generation and is a pure-Java
library.

Besides being easier to use with existing Java code, DJ has a few
other advantages compared to DemeterJ.  One is the ability to traverse
classes for which the programmer does not have source code, or is not
able or willing to modify the source code; for example, one might
traverse parts of Java's Swing library of GUI widgets.  DJ can
traverse public accessor methods, or may even use private methods and
fields if the JVM's security manager allows reflective access to
private parts (which is often the case outside of applets).  Another
feature of DJ which does not exist in DemeterJ is the ability to work
with subgraphs of a class graph; in DemeterJ, all traversals are
computed in the context of the whole class graph defined in the class
dictionary, but in DJ you can create new class graphs by selecting a
subgraph with a traversal strategy.  In addition, DJ allows components
to be more generic, by taking class graphs, traversal strategies, or
classes to be visited as run-time parameters.  These latter two
advantages are due to the reification of concepts which only exist at
compile-time in DemeterJ as first class objects in DJ.

An Adaptive Object-Model~\cite{AOM} is an object model that is
interpreted at run-time.  If an object model is changed, the system
changes its behavior.  Java's object model can't be changed at
run-time (other than dynamic class loading) but DJ interprets the
object model when doing traversals.

DJ's \code{Visitor} class is similar to the reflective visitor
described in~\cite{java-tip:reflective-visitor} and the
\code{Walkabout} class described in~\cite{palsberg:jay}.  However,
neither of these allows for customized traversals.

Java OQL, is the binding of OQL (Object Query Language) from ODMG 2.0
\cite{cattell:odmg2.0} to Java, treats query specifications much like
DJ treats traversal strategy specifications.  An \code{OQLQuery}
object can be constructed from a string describing a query; the query
can then be executed by calling the \code{execute()} method on the
\code{OQLQuery} object.  Queries are either compiled dynamically at
run-time or interpreted.  An example of a query is:
\begin{verbatim}
OQLQuery query = new OQLQuery("select p.getSpouse from p in persons");
Set spouses = (Set) query.execute();
\end{verbatim}
For an adaptive version of OQL, see~\cite{harrison:aql-94}.

DJ has some connections with aspect-oriented programming (AOP)
\cite{aop:ecoop97}.  An adaptive visitor is a specialized aspect: it
says what behavior should happen at certain principled points in the
execution of a traversal.  A traversal strategy can also be considered
an aspect: it adds crosscutting behavior whose implementation would
ordinarily require scattering methods across the class structure.
More details about the aspectual nature of DJ are in
\cite{aspectual-methods}.

\section{Summary and Future Work}

We have presented DJ, a pure-Java library supporting dynamic adaptive
programming.  DJ makes it easier to follow the Law of Demeter,
loosening the coupling between the structure and behavior concerns and
adapting to changes in the object model.  It is more flexible and
dynamic than the preprocessing approach taken by DemeterJ, by
interpreting traversal strategies at run-time and using reflection to
traverse object structures with adaptive visitors.

The Demeter team is currently working on a number of enhancements to
DJ.  One is the ability to write visitor methods that get executed
whenever certain edges in the class graph are executed (currently,
visitor method execution depends only on the class of the object being
traversed).  Another is the ability to write \code{around} methods on
visitors that are executed in place of a traversal step and get passed
a thunk that can be executed to continue the traversal.  Both of these
enhancements would allow visitors to have more control over the
behavior that gets executed during a traversal.

AspectJ \cite{aop:ecoop2001}, an implementation of AOP for Java, makes
available to advice an object that represents the current join point
with information such as the current type and the control-flow history
of past join points.  A similar object could be added to DJ that would
be passed to visitor methods so that they could access the history of
objects being traversed, in order to carry information from one point
of the traversal to others.

There are many optimizations that could be made to traversal
interpretation.  Currently fields and methods in the object structure
and methods on visitors are looked up using reflection every time they
are traversed; this information could be cached after the first
lookup.  Some of the reflection could be avoided altogether by
generating a new class at run-time that hard-coded the field accesses
and method executions involved in a traversal; although generating a
class would be a one-time performance hit, overall performance would
be much better if the same visitor and traversal strategy were used
together multiple times.

There is room for improvement in the capability for manipulating class
graph objects, such as being able to selectively remove edges or
recognizing special kinds of accessor methods such as a
\code{getFoo(int~i)} method that retrieved the \code{i}th member of a
collection of \code{Foo} objects.  Also, the ability to join, merge,
or intersect multiple traversal strategies would help make traversal
strategies more reusable.

\section{Acknowledgements}

Josh Marshall designed and implemented the early versions of DJ.
Johan Ovlinger designed and implemented the predecessor to DJ, TAO
(Traversals As Objects), and suggested the run-time generation of a
class to avoid reflective overhead.  Pengcheng Wu is implementing
several enhancements to DJ, including \code{around} visitor methods
and visitor methods on edges.  Thanks to Lars Hansen for
suggesting the use of threads as coroutines for the \code{asList}
iterator.  The XML Schema example is a simplified version of an
XML-Schema-to-Java tool being developed by Adak Prasenjit.

\bibliography{biblio,dj}
\bibliographystyle{plain}

\end{document}
