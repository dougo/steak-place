\documentclass[landscape]{slides}
\special{papersize=11in,8.5in}
\usepackage{amsmath}

\title{Incremental Programming with \\ Extensible Decisions}
\author{Doug Orleans \\
College of Computer Science\\
Northeastern University \\
\texttt{dougo@ccs.neu.edu}
}
\begin{document}

\maketitle

\sloppy % don't try so hard to justify the lines

\newcommand{\defn}[1]{\textbf{#1}}
\newcommand{\code}[1]{\texttt{#1}}
\newcommand{\slidetitle}[1]{\begin{center}\textbf{#1}\end{center}}

\begin{slide}
  \slidetitle{Incremental Programming}

  ``The construction of new program components by specifying how they
  differ from existing components.'' [Cook \& Palsberg, OOPSLA 1989]

  A language that supports incremental programming:
  \begin{itemize}
    \item allows good separation of concerns
    \item reduces code duplication
    \item improves extensibility and reuse
  \end{itemize}
\end{slide}

\begin{slide}
  \slidetitle{Extensible Decisions}

  In OOP, whenever a message is sent, a decision occurs (dynamic
  dispatch), but the branches of the decision are specified
  separately: methods that correspond to the message signature.

  Incremental programming is supported because methods can override
  each other, but only through inheritance.  Different branches of an
  extensible decision must involve different classes.
\end{slide}

\begin{slide}
  \slidetitle{Design Patterns}

  Many behavioral design patterns [Gamma et al, 1994] are essentially
  workarounds for this constraint of OOP.  For example, the State
  pattern implements state-based dispatch with one class per state.

  But\ldots
  \begin{itemize}
    \item Objects must delegate all state-dependent messages to their
          state objects: runtime overhead and a coding burden.
    \item The state objects must be manually kept up to date when the
          state condition changes: no implicit states.
  \end{itemize}
\end{slide}

\begin{slide}
  \slidetitle{A Different Solution}

  Change the language to make programs easier to express instead of
  changing your program to fit the constraints of the language.

  Two approaches to allowing incremental programming without requiring
  inheritance:
  \begin{itemize}
    \item Aspect-Oriented Programming (AOP) [Kiczales et al, 1997]
    \item Predicate Dispatching [Ernst, Kaplan, \& Chambers, ECOOP 1998.]
  \end{itemize}
\end{slide}

\begin{slide}
  \slidetitle{Aspect-Oriented Programming (AspectJ)}

  Incremental programming for crosscutting concerns: an aspect can
  override behavior in other classes without using inheritance.

  \begin{itemize}
    \item Each piece of advice has a \defn{pointcut} which specifies
          when the advice is applicable to a message send.
    \item Pointcuts can be arbitrary boolean expressions.
  \end{itemize}
\end{slide}

\begin{slide}
  \slidetitle{Predicate Dispatching}
  
  A form of dynamic dispatch that unifies and generalizes the dispatch
  mechanisms found in many programming languages, including OO single
  and multiple dispatch.

  \begin{itemize}
    \item Each method implementation has a \defn{predicate} which
          specifies when the method is applicable to a message send.
    \item Predicates can be arbitrary boolean expressions.
  \end{itemize}
\end{slide}

\begin{slide}
  \slidetitle{Advantages of AspectJ Over Predicate Dispatching}

  \begin{itemize}
    \item Pointcuts can access more information about a message send
          than just the receiver and arguments:
    \begin{itemize}
      \item message signature
      \item control flow history
      \item location of message send code
    \end{itemize}
    \item Method combination: \code{before}/\code{after} advice,
          \code{proceed()} in \code{around} advice
  \end{itemize}
\end{slide}

\begin{slide}
  \slidetitle{Advantages of Predicate Dispatching Over AspectJ}

  \begin{itemize}
    \item More elegant model, natural generalization of OO dispatch.
          Not tacked onto the side of Java.
    \item Method overriding is based on logical implication:
    \begin{itemize}    
      \item A method with predicate $p_1$ overrides a method with
            predicate $p_2$ if $p_2$ is true in all cases where $p_1$ is 
            true.
      \item Example: $\operatorname{is-a}(x,A) \implies
            \operatorname{is-a}(x,B)$ (where $A$ is a subclass of $B$)
      \item I.e., subclass methods override superclass methods, just
            like in OOP.
    \end{itemize}
  \end{itemize}
\end{slide}

\begin{slide}
  \slidetitle{Fred}

  A new programming language that takes the best from both worlds.

  Fred's dispatch mechanism unifies those of AOP and OOP languages,
  and provides uniform support for incremental programming whether the
  concerns implemented by the components are crosscutting or not.

  \begin{itemize}
    \item Behavior is specified as \defn{branches}, each of which has
          a condition predicate over \defn{decision points}.
    \item Decision points capture message signature, message
          arguments, source branch, previous decision point.
  \end{itemize}
\end{slide}

\begin{slide}
  \slidetitle{Fred (continued)}

  \begin{itemize}
    \item Overriding is based on logical implication of predicates,
          but \code{around} branches always override plain branches.
    \item \code{follow-next-branch} allows method combination.
    \item Syntactic sugar allows more declarative syntax for common
          kinds of branches, to look more like multimethods or
          advice.
    \item Prototype implementation is embedded in MzScheme.
  \end{itemize}
\end{slide}

\begin{slide}
  \slidetitle{OOP in Fred (1/3)}

  \begin{verbatim}
(define-class person () (fname lname))
(define-msg full-name)
(define-branch
  ;; condition predicate:
  (lambda (dp) (and (eq? (dp-msg dp) full-name)
                    (= (length (dp-args dp)) 1)
                    (is-a? (car (dp-args dp)) person)))
  ;; body:
  (lambda (dp) (let ((this (car (dp-args dp))))
                 (string-append (get-fname this) " "
                                (get-lname this)))))\end{verbatim}
\end{slide}

\begin{slide}
  \slidetitle{OOP in Fred (2/3)}

With sugar:
\begin{verbatim}
(define-branch (&& (call full-name) (args person))
  (with-args (this)
    (string-append (get-fname this) " " (get-lname this))))
\end{verbatim}

More sugar:
\begin{verbatim}
(define-method full-name (this) & (is-a? this person)
    (string-append (get-fname this) " " (get-lname this)))
\end{verbatim}

Still more sugar:
\begin{verbatim}
(define-method full-name ((person this))
    (string-append (get-fname this) " " (get-lname this)))
\end{verbatim}
\end{slide}

\begin{slide}
  \slidetitle{OOP in Fred (3/3)}
  \begin{verbatim}
(define-class knight (person) ())
(define-method full-name ((knight this))
  (string-append "Sir " (follow-next-branch)))
\end{verbatim}

Example output:
\begin{verbatim}
> (define gandalf (make knight "Ian" "McKellen"))
> (full-name gandalf)
"Sir Ian McKellen"
\end{verbatim}
\end{slide}

\begin{slide}
\slidetitle{AOP in Fred (1/2)}
\begin{verbatim}
(define-around
  (lambda (dp) (is-a? (car (dp-args dp)) person))
  (with-msg (msg)
    (printf "Received message ~a.~n" (msg-name msg))
    (follow-next-branch)))
\end{verbatim}

With sugar:
\begin{verbatim}
(define-before (args person ..)
  (with-msg (msg)
    (printf "Received message ~a.~n" (msg-name msg))))
\end{verbatim}
\end{slide}

\begin{slide}
\slidetitle{AOP in Fred (2/2)}
\begin{verbatim}
(define-before (&& (args person ..) (! in-full-name-cflow?))
  (with-msg-and-args (msg this . rest)
    (printf "~a received message ~a.~n"
            (full-name this) (msg-name msg))))

(define (in-full-name-cflow? dp)
  (let ((prev (dp-previous dp)))
    (and prev (or (eq? (dp-msg prev) full-name)
                  (in-full-name-cflow? prev)))))
\end{verbatim}
With sugar:
\begin{verbatim}
(define in-full-name-cflow? (cflowbelow (call full-name)))
\end{verbatim}
\end{slide}

\begin{slide}
  \slidetitle{Future Work}

  \begin{itemize}
    \item Compare with other AOP models (composition filters, hyperslices,
          mixin layers, aspectual collaborations, QoS regions)
    \item Variable binding in predicates (like AspectJ's context
          exposure in pointcuts)
    \item Customized branch overriding relationships
    \item Extensible predicates
    \item Modularization, using MzScheme's units [Flatt \& Felleisen,
          PLDI 1998]
  \end{itemize}
\end{slide}

\end{document}

